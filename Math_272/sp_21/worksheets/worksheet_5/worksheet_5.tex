%%%%%%%%%%%%%%%%%%%%%%%%%%%%%%%%%%%%%%%%%%%%%%%%%%%%%%%%%%%%%%%%%%%%%%%%%%%%%%%%%%%%
% Document data
%%%%%%%%%%%%%%%%%%%%%%%%%%%%%%%%%%%%%%%%%%%%%%%%%%%%%%%%%%%%%%%%%%%%%%%%%%%%%%%%%%%%
\documentclass[12pt]{article} %report allows for chapters
%%%%%%%%%%%%%%%%%%%%%%%%%%%%%%%%%%%%%%%%%%%%%%%%%%%%%%%%%%%%%%%%%%%%%%%%%%%%%%%%%%%%
\usepackage{preamble}

\begin{document}

\begin{center}
   \textsc{\large MATH 272, Worksheet 5}\\
   \textsc{Coordinate systems.}
\end{center}
\vspace{.5cm}

\begin{problem}
Provide an explicit or implicit description (both if possible) of the following regions in $\R^3$ in Cartesian coordinates, cylindrical coordinates, and spherical coordinates.
    \begin{enumerate}[(a)]
        \item A solid box with side lengths $a$, $b$, and $c$.
        \item A solid cylinder of height $h$.
        \item A solid ball of radius $R$.
    \end{enumerate}
\end{problem}

\vspace*{1cm}
\begin{problem}
Likewise, provide an explicit or implicit description (both if possible) of the following surfaces in $\R^3$ in Cartesian coordinates, cylindrical coordinates, and spherical coordinates.
    \begin{enumerate}[(a)]
        \item A the surface of a box with side lengths $a$, $b$, and $c$.
        \item A the surface of a cylinder of height $h$ including endcaps.
        \item A the surface of a ball of radius $R$ (i.e., the sphere).
    \end{enumerate}
\end{problem}


\vspace*{1cm}
\begin{problem}
One can also take a look at curves in each coordinate system.  Plot the following curves by hand. Play around with different choices of parameterizations. Hold certain variables constant. See how this affects your curve!
\begin{enumerate}[(a)]
    \item $x(t)=t$, $y(t)=t$, $z(t)=t$.
    \begin{itemize}
        \item What happens if we hold $x(t)=C$ constant? 
        \item What if we held both $x(t)=C_1$, and $y(t)=C_2$ constant?
        \item Choose some other functions (including constants) for $x(t)$, $y(t)$, and $z(t)$, and plot these as well.
    \end{itemize}
    \item $\rho(t)=t$, $\theta(t)=t$, $z(t)=t$.
    \begin{itemize}
        \item What happens if we hold $\rho(t)=C$ constant? 
        \item What if we held both $\rho(t)=C_1$, and $\theta(t)=C_2$ constant? Or held $\rho(t)$ and $z(t)$ constant?
        \item Choose some other functions (including constants) for $\rho(t)$, $\theta(t)$, and $z(t)$, and plot these as well.
    \end{itemize}
    \item $r(t)=t$, $\theta(t)=t$, $\phi(t)=t$.
    \begin{itemize}
        \item What happens if we hold $r(t)=C$ constant? 
        \item What if we held both $r(t)=C_1$, and $\theta(t)=C_2$ constant? Or held $r(t)$ and $\phi(t)$ constant?
        \item Choose some other functions (including constants) for $\rho(t)$, $\theta(t)$, and $z(t)$, and plot these as well.
    \end{itemize}
\end{enumerate}
\end{problem}

\vspace*{1cm}
\begin{problem}
Integrate the following functions in their relevant coordinate systems over the given region $\Omega$.
\begin{enumerate}[(a)]
    \item $f(x,y,z) = xy+yz$ over $\Omega$ which is the solid unit cube.
    \item $f(\rho,\theta,z) = \frac{z}{\rho}\sin(\theta)$ over $\Omega$ which is the cylinder of radius $1$ and of height $2$. Align this cylinder so the $z$-axis runs through the core of the cylinder and so the height is split at the $xy$-plane.
    \item $f(r,\theta,\phi) = \frac{1}{r^2}\sin(\theta)\sin(\phi)$ over the unit ball centered at the origin.
\end{enumerate}
\end{problem}

\vspace*{1cm}
\begin{problem}
    Consider a description of the plane $\R^2$ in both Cartesian coordinates $(x,y)$ and polar coordinates $(r,\theta)$.  Recall the coordinate transformations
    \begin{align*}
        x(r,\theta)&= r\cos \theta\\
        y(r,\theta)&= r\sin \theta.
    \end{align*}
    Let $f(x,y) = \frac{1}{\sqrt{1-x^2-y^2}}$ and let us attempt to integrate
    \[
    \iint_{\Sigma} f(x,y)d\Sigma,
    \]
    where $\Sigma$ is the unit disk defined by $x^2+y^2\leq 1$.
    \begin{enumerate}[(a)]
        \item Convert $f(x,y)$ in Cartesian coordinates to a function $f(r,\theta)$ in polar coordinates.
        \item Note that we can convert the area form $d\Sigma=dxdy$ in Cartesian coordinates to an area form in polar coordinates.  Think of the function
        \[
        \vec{\textrm{Pol}}(r,\theta) = \begin{pmatrix} x(r,\theta) \\ y(r,\theta) \end{pmatrix},
        \]
        as the function that coverts Cartesian coordinates into polar coordinates.  Then, 
        \[
        [J]_{\vec{\textrm{Pol}}} = \begin{pmatrix} \frac{\partial x}{\partial r} & \frac{\partial x}{\partial \theta} \\ \frac{\partial y}{\partial r} & \frac{\partial y}{\partial \theta} \end{pmatrix},
        \]
        is the Jacobian of this transformation. The magnitude of determinant of a matrix describes the stretching that the matrix does to the space at a point, and thus the magnitude of the determinant will be a function that depends on each point that describes the local stretching behavior.
        
        Compute $[J]_{\vec{\textrm{Pol}}}$ and compute the determinant of this matrix as well. Simplify this expression as much as possible.
        \item Now, to find the area form in polar coordinates, we simply take
        \[
        \left|\det\left([J]_{\vec{\textrm{Pol}}}\right)\right|drd\theta.
        \]
        Confirm that you have the area form $rdrd\theta$.
        \item Draw a picture of a small segment (sides $dr$ and $d\theta$) in the plane.  Can you see why the area of this segment depends on the radius? Can you also see why it does \underline{not} depend on the angle?
        \item We know we have this correct if 
        \[
        \int_{0}^{2\pi} \int_0^R rdrd\theta = \pi R^2,
        \]
        since this is the area contained inside a circle of radius $R$.  Show that the above integral is true.
        \item Now, set up the integral posed initially in polar coordinates and evaluate this integral using a change of variables (\underline{do not use WolframAlpha}).
    \end{enumerate}
\end{problem}

\vspace*{1cm}
\begin{problem}
    Repeat a very similar argument to show that the volume element $d\Omega$ in cylindrical coordinates is $\rho d\rho d\theta dz$.  \emph{Hint: The coordinate transformations for $x$ and $y$ are analogous.  But, we have the addition of the $z$-coordinate, but this is identical to the Cartesian $z$-coordinate.}
\end{problem}

\vspace*{1cm}
\begin{center}
The following three problems are designed to show you how one can derive the unit vector fields in each coordinate system. It is important to remember that the gradient vector field points in the direction of greatest increase for a function -- this will be essential in the understanding of why we define these quantities in the way that we do.
\end{center}

\vspace*{1cm}
\begin{problem}
In Cartesian coordinates, one can consider the curves given when two of the Cartesian coordinates are held constant. That is,
\begin{itemize}
    \item Define $\curvegamma_x$ by $x(t)=t$, $y(t)=y_0$, and $z(t)=z_0$;
    \item Define $\curvegamma_y$ by $x(t)=x_0$, $y(t)=t$, and $z(t)=z_0$;
    \item Define $\curvegamma_z$ by $x(t)=x_0$, $y(t)=y_0$, and $z(t)=t$.
\end{itemize}
If we consider the unit tangent vectors to these curves, we will recover our basis vectors $\xhat$, $\yhat$, and $\zhat$ at whichever point we wish.  Let's see how.
\begin{enumerate}[(a)]
    \item Compute the tangent vectors $\tangentgamma_x$, $\tangentgamma_y$, and $\tangentgamma_z$.  Normalize these vectors if need be.
    \item For example, does choosing values of $t$, $y_0$, and $z_0$ for $\tangentgamma_x(t)$ change the tangent vector? That is, do the tangent vectors change based on the point at which they are based? Make sure to consider all the different tangent vectors!
    \item To see what these unit vectors $\xhat$, $\yhat$, and $\zhat$ look like, we can also compute, for example,
        \[
        \xhat(x,y,z) = \frac{\grad x(x,y,z)}{\left| \grad x(x,y,z)\right|}.
        \]
        Note that $x(x,y,z)=x$. This is just saying the $x$-position only depends on the $x$-value of the point we are at.
    \item Show that these vectors are orthogonal at every point $(x,y,z)$.
    \item Plot the vector fields $\xhat(x,y,z)$, $\yhat(x,y,z)$, and $\zhat(x,y,z)$.
\end{enumerate}
\end{problem}

\vspace*{1cm}
\begin{problem}
* In cylindrical coordinates, one can consider the curves given when two of the cylindrical coordinates are held constant.  That is, 
\begin{itemize}
    \item Define $\curvegamma_\rho$ by $\rho(t)=t$, $\theta(t)=\theta_0$, and $z(t)=z_0$;
    \item Define $\curvegamma_\theta$ by $\rho(t)=\rho_0$, $\theta(t)=t$, and $z(t)=z_0$;
    \item Define $\curvegamma_z$ by $\rho(t)=\rho_0$, $\theta(t)=\theta_0$, and $z(t)=t$.
\end{itemize}
If we consider the unit tangent vectors to these curves, we will recover a new set of basis vectors $\rhohat$, $\thetahat$, and $\zhat$.  In fact, these basis vectors depend on the point at which they are based, and hence they are actually defining a field of vectors.  One should emphasize this by putting $\rhohat(\rho,\theta,z)$, $\thetahat(\rho,\theta,z)$, and $\zhat(\rho,\theta,z)$.
\begin{enumerate}[(a)]
    \item Compute the tangent vectors $\tangentgamma_\rho$, $\tangentgamma_\theta$, and $\tangentgamma_z$.  Normalize these vectors if need be.
    \item For example, does choosing values of $t$, $\theta_0$, and $z_0$ for $\tangentgamma_\rho(t)$ change the tangent vector? That is, do the tangent vectors change based on the point at which they are based? Make sure to consider all the different tangent vectors!
    \item To find out what these vectors $\rhohat$, $\thetahat$, and $\zhat$ look like in terms of the unit vectors $\xhat$, $\yhat$ and $\zhat$, we can also compute, for example,
    \[
    \rhohat(x,y,z) = \frac{\grad \rho(x,y,z)}{\left| \grad \rho(x,y,z) \right|}.
    \]
    Show that
    \begin{align*}
        \rhohat(x,y,z) &= \frac{x}{\sqrt{x^2+y^2}}\xhat + \frac{y}{\sqrt{x^2+y^2}} \yhat,\\
        \thetahat(x,y,z) &= \frac{-y}{\sqrt{x^2+y^2}} \xhat + \frac{x}{\sqrt{x^2+y^2}} \yhat,\\
        \zhat(x,y,z) &= \zhat.
    \end{align*}
    \item Show that these vectors are orthogonal at every point $(x,y,z)$.
    \item Plot the vector fields $\rhohat(x,y,z)$, $\thetahat(x,y,z)$, and $\zhat(x,y,z)$.
\end{enumerate}
\end{problem}

\vspace*{1cm}
\begin{problem}
* In spherical coordinates, one can consider the curves given when two of the spherical coordinates are held constant.  That is, 
\begin{itemize}
    \item Define $\curvegamma_r$ by $r(t)=t$, $\theta(t)=\theta_0$, and $\phi(t)=\phi_0$;
    \item Define $\curvegamma_\theta$ by $r(t)=r_0$, $\theta(t)=t$, and $\phi(t)=\phi_0$;
    \item Define $\curvegamma_\phi$ by $r(t)=r_0$, $\theta(t)=\theta_0$, and $\phi(t)=t$.
\end{itemize}
If we consider the unit tangent vectors to these curves, we will recover a new set of basis vectors $\rhohat$, $\thetahat$, and $\phihat$.  In fact, these basis vectors depend on the point at which they are based, and hence they are actually defining a field of vectors.  One should emphasize this by putting $\rhat(r,\theta,\phi)$, $\thetahat(r,\theta,\phi)$, and $\phihat(r,\theta,\phi)$.
\begin{enumerate}[(a)]
    \item Compute the tangent vectors $\tangentgamma_r$, $\tangentgamma_\theta$, and $\tangentgamma_\phi$.  Normalize these vectors if need be.
    \item For example, does choosing values of $t$, $\theta_0$, and $\phi_0$ for $\tangentgamma_r(t)$ change the tangent vector? That is, do the tangent vectors change based on the point at which they are based? Make sure to consider all the different tangent vectors!
    \item To find out what these vectors $\rhat$, $\thetahat$, and $\phihat$ look like in terms of the unit vectors $\xhat$, $\yhat$ and $\zhat$, we can also compute, for example,
    \[
    \rhat(x,y,z) = \frac{\grad r(x,y,z)}{\left| \grad r(x,y,z) \right|}.
    \]
    Show that
    \begin{align*}
        \rhat(x,y,z) &= \frac{x}{\sqrt{x^2+y^2+z^2}}\xhat + \frac{y}{\sqrt{x^2+y^2+z^2}} \yhat + \frac{z}{\sqrt{x^2+y^2+z^2}} \zhat,\\
        \thetahat(x,y,z) &= \frac{-y}{\sqrt{x^2+y^2}} \xhat + \frac{x}{\sqrt{x^2+y^2}} \yhat,\\
        \phihat(x,y,z) &= \frac{xz}{\sqrt{x^2+y^2}(x^2+y^2+z^2)}\xhat + \frac{yz}{\sqrt{x^2+y^2}(x^2+y^2+z^2)}\yhat + \frac{-\sqrt{x^2+y^2}}{x^2+y^2+z^2}\zhat
    \end{align*}
    \item Show that these vectors are orthogonal at every point $(x,y,z)$.
    \item Plot the vector fields $\rhat(x,y,z)$, $\thetahat(x,y,z)$, and $\phihat(x,y,z)$.
\end{enumerate}
\end{problem}




\vspace*{1cm}
\begin{center} 
The following four problems are used to determine how the gradient operator is determined in various coordinate systems.
\end{center}

\vspace*{1cm}
\begin{problem} *
    Suppose that we have the vector $\vecv = x\xhat + y\yhat$ in the plane $\R^2$.  If I move this vector infinitesimally in each component, then how much length is swept out? That is, what is the differential $d\vecv$? We can compute this differential by taking,
    \[
    d\vecv = \frac{\partial \vecv}{\partial x}dx+\frac{\partial \vecv}{\partial y}dy.
    \]
    Then we can refer to, for example, the value
    \[
    h_x = \left| \frac{\partial \vecv}{\partial x} \right|,
    \]
    as the scale factor for $x$.
    \begin{enumerate}[(a)]
        \item Show that in Cartesian coordinates that
        \[
        d\vecv = h_x \xhat dx + h_y \yhat dy = \xhat dx + \yhat dy.
        \]
        In other words, show the scale factors $h_x$ and $h_y$ are one.
        \item In general, the gradient $\grad$ is computed by swapping $dx$ for $\frac{\partial}{\partial x}$ and likewise $dy$ for $\frac{\partial}{\partial y}$.  We also invert the scale factors. Thus,
        \[
        \grad = \frac{1}{h_x} \xhat \frac{\partial }{\partial x} + \frac{1}{h_y} \yhat \frac{\partial }{\partial y} = \xhat \frac{\partial }{\partial x} + \yhat \frac{\partial }{\partial y}
        \]
    \end{enumerate}
\end{problem}

\vspace*{1cm}
\begin{problem} *
    Likewise, suppose that we have the vector $\vecv = x\xhat + y\yhat$ in the plane $\R^2$.  Now, use the polar coordinates
    \begin{align*}
    x &= \rho \cos \theta\\
    y &= \rho \sin \theta
    \end{align*}
    If I move this vector infinitesimally in each component, then how much length is swept out? That is, what is the differential $d\vecv$? We can compute this differential by taking,
    \[
    d\vecv = \frac{\partial \vecv}{\partial \rho}d\rho+\frac{\partial \vecv}{\partial \theta}d\theta.
    \]
    Then we can refer to, for example, the value
        \[
        h_\rho = \left| \frac{\partial \vecv}{\partial \rho} \right|,
        \]
        as the scale factor for $\rho$.
    \begin{enumerate}[(a)]
        \item Show that in polar coordinates that
        \[
        h_r = 1 \qquad \textrm{and} \qquad h_\theta = \rho.
        \]
        This means that
        \[
        d\vecv =  \rhohat d\rho + r\thetahat d\theta.
        \]
        \item Argue why the gradient in polar coordinates is then
        \[
        \grad = \rhohat \frac{\partial}{\partial \rho} + \thetahat \frac{1}{\rho}\frac{\partial}{\partial \theta}.
        \]
    \end{enumerate}
\end{problem}

\vspace*{1cm}
\begin{problem}
    * Repeat the previous arguments for cylindrical and spherical coordinates.
\end{problem}


%\vspace*{1cm}
%\begin{centering} 
%The following two problems derive the divergence and Laplacian in cylindrical and spherical coordinates.
%\end{centering}

\begin{problem}
    Consider a description of the plane $\R^2$ in both Cartesian coordinates $(x,y)$ and polar coordinates $(r,\theta)$.  Recall the coordinate transformations
    \begin{align*}
        x(r,\theta)&= r\cos \theta\\
        y(r,\theta)&= r\sin \theta.
    \end{align*}
    Let $f(x,y) = \frac{1}{\sqrt{1-x^2-y^2}}$ and let us attempt to integrate
    \[
    \iint_{\Sigma} f(x,y)d\Sigma,
    \]
    where $\Sigma$ is the unit disk defined by $x^2+y^2\leq 1$.
    \begin{enumerate}[(a)]
        \item Convert $f(x,y)$ in Cartesian coordinates to a function $f(r,\theta)$ in polar coordinates.
        \item Note that we can convert the area form $d\Sigma=dxdy$ in Cartesian coordinates to an area form in polar coordinates.  Think of the function
        \[
        \vec{\textrm{Pol}}(r,\theta) = \begin{pmatrix} x(r,\theta) \\ y(r,\theta) \end{pmatrix},
        \]
        as the function that coverts Cartesian coordinates into polar coordinates.  Then, 
        \[
        [J]_{\vec{\textrm{Pol}}} = \begin{pmatrix} \frac{\partial x}{\partial r} & \frac{\partial x}{\partial \theta} \\ \frac{\partial y}{\partial r} & \frac{\partial y}{\partial \theta} \end{pmatrix},
        \]
        is the Jacobian of this transformation. The magnitude of determinant of a matrix describes the stretching that the matrix does to the space at a point, and thus the magnitude of the determinant will be a function that depends on each point that describes the local stretching behavior.
        
        Compute $[J]_{\vec{\textrm{Pol}}}$ and compute the determinant of this matrix as well. Simplify this expression as much as possible.
        \item Now, to find the area form in polar coordinates, we simply take
        \[
        \left|\det\left([J]_{\vec{\textrm{Pol}}}\right)\right|drd\theta.
        \]
        Confirm that you have the area form $rdrd\theta$.
        \item Draw a picture of a small segment (sides $dr$ and $d\theta$) in the plane.  Can you see why the area of this segment depends on the radius? Can you also see why it does \underline{not} depend on the angle?
        \item We know we have this correct if 
        \[
        \int_{0}^{2\pi} \int_0^R rdrd\theta = \pi R^2,
        \]
        since this is the area contained inside a circle of radius $R$.  Show that the above integral is true.
        \item Now, set up the integral posed initially in polar coordinates and evaluate this integral using a change of variables (\underline{do not use WolframAlpha}).
    \end{enumerate}
\end{problem}

\vspace*{1cm}
\begin{problem}
    Repeat a very similar argument to show that the volume element $d\Omega$ in cylindrical coordinates is $\rho d\rho d\theta dz$.  \emph{Hint: The coordinate transformations for $x$ and $y$ are analogous.  But, we have the addition of the $z$-coordinate, but this is identical to the Cartesian $z$-coordinate.}
\end{problem}




\end{document}

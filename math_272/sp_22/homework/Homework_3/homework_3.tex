%%%%%%%%%%%%%%%%%%%%%%%%%%%%%%%%%%%%%%%%%%%%%%%%%%%%%%%%%%%%%%%%%%%%%%%%%%%%%%%%%%%%
% Document data
%%%%%%%%%%%%%%%%%%%%%%%%%%%%%%%%%%%%%%%%%%%%%%%%%%%%%%%%%%%%%%%%%%%%%%%%%%%%%%%%%%%%
\documentclass[12pt]{article} %report allows for chapters
%%%%%%%%%%%%%%%%%%%%%%%%%%%%%%%%%%%%%%%%%%%%%%%%%%%%%%%%%%%%%%%%%%%%%%%%%%%%%%%%%%%%
\usepackage{preamble}
\newcommand{\grad}{\boldsymbol{\vec{\nabla}}}
\newcommand{\curvegamma}{\boldsymbol{\vec{\gamma}}}
\newcommand{\tangentgamma}{\boldsymbol{\dot{\vec{\gamma}}}}
\newcommand{\normalgamma}{\boldsymbol{\ddot{\vec{\gamma}}}}
\newcommand{\vecfieldE}{\boldsymbol{\vec{E}}}
\newcommand{\rhat}{\boldsymbol{\hat{r}}}
\newcommand{\thetahat}{\boldsymbol{\hat{\theta}}}
\newcommand{\phihat}{\boldsymbol{\hat{\phi}}}
\newcommand{\rhohat}{\boldsymbol{\hat{\rho}}}
\newcommand{\unitvec}{\boldsymbol{\hat{n}}}
\newcommand{\vecfieldB}{\boldsymbol{\vec{B}}}
\newcommand{\vecfieldJ}{\boldsymbol{\vec{J}}}
\newcommand{\vecfieldF}{\boldsymbol{\vec{F}}}
\newcommand{\vecfieldV}{\boldsymbol{\vec{V}}}
\newcommand{\vecfieldU}{\boldsymbol{\vec{U}}}

\begin{document}

\begin{center}
   \textsc{\large MATH 272, Homework 3}\\
   \textsc{Due February 17$^\textrm{th}$}
\end{center}
\vspace{.5cm}


\begin{problem}
Show that for any smooth (more than twice differentiable) fields $f(x,y,z)$ and $\vecfieldV(x,y,z)$ that
\begin{enumerate}[(a)]
	\item $\grad \times \left(\grad f\right)=\boldsymbol{\vec{0}}$;
	\item $\grad \cdot \left(\grad \times \vecfieldV\right)=0$.
\end{enumerate}
\end{problem}

\begin{problem}
	Let 
	\[
	\vecfieldU(x,y,z) = \begin{pmatrix} -y \\ x \\ 0 \end{pmatrix} \qquad \textrm{and} \qquad \vecfieldV(x,y,z) = \begin{pmatrix} 2x \\ 2y \\ 2z \end{pmatrix},
	\] 
	be vector fields.  
	\begin{enumerate}[(a)]
		\item Explain why there exists no potential function $\phi(x,y,z)$ for the vector field $\vecfieldU$.
		\item Explain why there does exist a potential function $\phi(x,y,z)$ for the field $\vecfieldV$.
		\item Compute the potential function for $\vecfieldV$.
	\end{enumerate}
\end{problem}

\begin{problem}
Consider the two dimensional scalar field $T(x,y)=x+y$ that describes the temperature on the square plate $\Omega$ given by the set $0\leq x,y \leq 1$.  Compare the two answers you get!
\begin{enumerate}[(a)]
	\item Compute the integral
	\[
	\int_\Omega T d\Omega.
	\]
	\item Let $\curvegamma$ be the curve that traverses the boundary of the square plate in the counterclockwise direction.  Compute
	\[
	\int_{\curvegamma} T d\curvegamma. 
	\]
\end{enumerate}
\end{problem}

\begin{problem}
Let $f(x,y,z)=2xy+e^{xz}+\sin(y)$ be a scalar field. Integrate $f$ over the triangular prism $\Omega$ defined by taking the half triangle of the unit square in the $xy$-plane satisfying $x\leq y$ and with height 4 above the $xy$-plane.
\end{problem}

\begin{problem}
Consider $f(x,y)= 3x^4+x^3-18x^2y^2-3xy^2+3y^4$. 
\begin{enumerate}[(a)]
    \item Show $\Delta f = 0$.
    \item Find the surface normal to the graph of $f$.
\end{enumerate}
\end{problem}

\begin{problem} 
Parameterize the following either implicitly or explicitly. In Cartesian coordinates, find the parameterization of the normal vector as well.
\begin{enumerate}[(a)]
	\item The plane perpendicular to the vector $\vecv = \xhat + \yhat + \zhat$ passing through the point $(1,1,1)$.
	\item The upper half of the unit disk in the $xy$-plane.
	\item The surface of the unit sphere in $\R^3$.
\end{enumerate}
\end{problem}

\begin{problem}
Consider the following vector field
\[
\vecfieldE(x,y,z) = \begin{pmatrix} \frac{x}{(x^2+y^2+z^2)^{3/2}} \\ \frac{y}{(x^2+y^2+z^2)^{3/2}} \\ \frac{z}{(x^2+y^2+z^2)^{3/2}} \end{pmatrix},
\]
which models the electric field of a proton (in units of of charge $q=1$) placed at the origin.
\begin{enumerate}[(a)]
	\item Show that $\vecfieldE(x,y,z) = - \grad \phi(x,y,z)$ where $\phi(x,y,z) = \frac{1}{\sqrt{x^2+y^2+z^2}}$.  We refer to $\phi(x,y,z)$ as the electrostatic potential (or voltage).
	\item Let $\Omega$ be a box with side lengths two centered at the origin.  Compute the total flux of $\vecfieldE$ through the surface of the box $\Sigma$. That is,
	\[
	\int_\Sigma \vecfieldE \cdot \unitvec d\Sigma.
	\]
	\item Using the provided argument, one can compute
	\[
	\int_\Omega \grad \cdot \vecfieldE d\Omega.
	\]
	\begin{itemize}
		\item Compute $\grad \cdot \vecfieldE$ and note that this is zero everywhere except at $(x,y,z)=(0,0,0)$.
		\item Note that the two integrals in this problem are equal. This is known as the \emph{divergence theorem} and it is a special case of a more general theorem called \emph{Stokes' theorem} which generalizes the fundamental theorem of calculus. Is it true that $\grad \cdot \vecfieldE = 0$ everywhere?
	\end{itemize}
	\item Does the total flux depend on the size or shape of the box?
\end{enumerate}
\end{problem}




\end{document}
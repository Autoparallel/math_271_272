%%%%%%%%%%%%%%%%%%%%%%%%%%%%%%%%%%%%%%%%%%%%%%%%%%%%%%%%%%%%%%%%%%%%%%%%%%%%%%%%%%%%
% Document data
%%%%%%%%%%%%%%%%%%%%%%%%%%%%%%%%%%%%%%%%%%%%%%%%%%%%%%%%%%%%%%%%%%%%%%%%%%%%%%%%%%%%
\documentclass[12pt]{article} %report allows for chapters
%%%%%%%%%%%%%%%%%%%%%%%%%%%%%%%%%%%%%%%%%%%%%%%%%%%%%%%%%%%%%%%%%%%%%%%%%%%%%%%%%%%%
\usepackage{preamble}
\newcommand{\curvegamma}{\boldsymbol{\vec{\gamma}}}
\newcommand{\tangentgamma}{\boldsymbol{\dot{\vec{\gamma}}}}
\newcommand{\normalgamma}{\boldsymbol{\ddot{\vec{\gamma}}}}
\newcommand{\rhat}{\boldsymbol{\hat{r}}}
\newcommand{\thetahat}{\boldsymbol{\hat{\theta}}}
\newcommand{\phihat}{\boldsymbol{\hat{\phi}}}
\newcommand{\rhohat}{\boldsymbol{\hat{\rho}}}
\newcommand{\vecfieldB}{\boldsymbol{\vec{B}}}
\newcommand{\vecfieldJ}{\boldsymbol{\vec{J}}}
\newcommand{\vecfieldF}{\boldsymbol{\vec{F}}}
\newcommand{\vecx}{\boldsymbol{\vec{x}}}

\newcommand{\veclaplace}{\boldsymbol{\vec{\Delta}}}

\begin{document}

\begin{center}
   \textsc{\large MATH 272, Homework 5}\\
   \textsc{Due April 6$^\textrm{th}$}
\end{center}
\vspace{.5cm}

\begin{problem}
\textbf{(5 pt.)} Take the set up from the previous problem, but let us modify the initial conditions and boundary conditions slightly. Instead, we have
\[
\begin{cases}
\frac{\partial}{\partial t} u(x,t) = \frac{\partial^2}{\partial x^2} u(x,t) - 1, & \textrm{in $(0,1)$},\\
u(0,t)=0 \textrm{~and~} u(1,t)=0, & \text{as boundary conditions},\\
u(x,0) = -(2x-1)^2+1, & \textrm{as the initial condition}.
\end{cases}
\]
We want to discover how we can possibly solve problems with more general initial conditions. If you pay attention to the work in 3, you will find the initial conditions were chosen in a very contrived manner. This is not ideal if we want to solve a problem in general!

Taking a look at Example ``Particular Solution to the 1D Heat Equation" in the notes. Notice that it is of the form
\[
u_n(x,t) = e^{-n^2\pi^2 t} \sin(n \pi x),
\]
is a solution for all integers $n$. In fact, given a constant $A_n$, $A_n u_n$ is a solution as well!
\begin{enumerate}[(a)]
    \item \textbf{(1 pt.)} Can you recreate the initial condition $u(x,0)$ with a single $A_n u_n(x,0)$?
    \item \textbf{(1 pt.)} Can you recreate the initial condition with a finite linear combination of $u_n(x,0)$?
    \item \textbf{(2 pt.)} Suppose that we can take an infinite linear combination
    \[
    \sum_{n=1}^\infty A_n u_n.
    \]
    Show that
    \[
    \sum_{n=1}^\infty -\frac{16\left(\left(-1\right)^{n}-1\right)}{\pi^{3}n^{3}}u_n(x,0) = u(x,0)
    \]
    by plotting both $u(x,0)$ and the sum (up to a large $N$) simultaneously. It is worthwhile to steadily increase the upper bound of the sum to see this convergence! For your own sake (and for correctness of the problem), plot your functions only on the domain we care about.
    \item \textbf{(1 pt.)} Comment on this. Do you think this is something we can do in general for any initial condition?
\end{enumerate}
\end{problem}


\begin{problem}
\textbf{(12 pt.)} Previously we studied the time-independent Schr\"odinger equation. Now, we can take a look at the time-dependent version given by
\[
\left( H - i\hbar \frac{\partial}{\partial t} \right) \Psi(x,t) = 0,
\]
where $H$ is the Hamiltonian operator.  

Consider the situation for the free particle in the 1-dimensional box of length $1$ so that $V(x)=0$ and $\Psi(0,t)=0=\Psi(1,t)$. Hence, the problem
\[
\begin{cases}
\left( -\frac{\hbar}{2m} \frac{\partial ^2}{\partial x^2} - i\hbar \frac{\partial}{\partial t} \right) \Psi(x,t) = 0 & \textrm{in the region $\Omega = [0,1]$}\\
\Psi(0,t)=0=\Psi(1,t), & \textrm{as boundary conditions}.
\end{cases}
\]
Note that for each $n$ we have a normalized state
\[
\psi_n(x,t) = \sqrt{2} e^{-i \frac{E_n}{\hbar} t} \sin(n \pi x)
\]
where 
\[
E_n = \frac{n^2 \pi^2 \hbar^2}{2m}
\]
is the energy eigenvalue.
\begin{enumerate}[(a)]
    \item \textbf{(1 pt.)} Argue that a linear combination (superpostion) of states is a solution by noting that the operator
	\[
	\mathcal{L} = -\frac{\hbar}{2m} \frac{\partial ^2}{\partial x^2} - i\hbar \frac{\partial}{\partial t}
	\]
	is linear. \emph{Hint: think about kernels and subspaces.}
    \item \textbf{(2 pt.)} For a single state $\psi_n(x,t)$, show that the probability amplitude
    \[
\left|\psi_n(x,t)\right|^2
    \]
    is independent of $t$. This shows that the states $\psi_n$ are \emph{stationary} since their probability amplitude does not depend on time.
	\item \textbf{(3 pt.)} When a particle is in superposition, then the probability amplitude can change over time. Consider the superposition wavefunction
	\[
	\Psi = \frac{1}{\sqrt{2}} \left( \psi_1 + \psi_2 \right).
	\]
	Show that the probability amplitude can be written as
	\[
	|\Psi|^2 = 2 \cos(\omega_R t)\sin(\pi x)\sin(2\pi x)+ \sin(\pi x)^2 +\sin(2\pi x)^2
	\]
	and determine $\omega_R$ in terms of the energy eigenvalues. Note that $\omega_R$ is called the \emph{Rabi frequency} and this occurs in two-state systems.
	\item \textbf{(1 pt.)} Quickly verify that the total probability amplitude for the superposition wavefunction $\Psi$ is 1. \emph{Hint: please just use the fact that the states are orthonormal!}
	\item \textbf{(2 pt.)} Make an animation of the probability amplitude on the given domain $\Omega$ on Desmos. (For credit: show your Desmos code or show a few slices of the probability amplitude over time.)
	\item \textbf{(1 pt.)} Can you interpret your answer in part (d) as motion of the particle? Explain.
	\item \textbf{(2 pt.)} Given an initial wave function $\Psi(x,0)$, how can you determine $\Psi(x,t)$? \emph{Hint: consider what we discuss in Problem 1. Can you take an infinite linear combination of states? If so, how do you determine the coefficients?}
\end{enumerate}
\end{problem}


\begin{problem}
\textbf{(7 pt.)} Maxwell's equations are given as
\begin{align*}
\grad \cdot \vecfieldB &= 0  & \grad \cdot \vecfieldE &= \frac{\rho}{\epsilon_0}\\
\grad \times \vecfieldB -\mu_0 \epsilon_0\frac{\partial \vecfieldE}{\partial t}&=\mu_0 \vecfieldJ & \grad \times \vecfieldE + \frac{\partial \vecfieldB}{\partial t} &= \zerovec
\end{align*}
\begin{enumerate}[(a)]
    \item \textbf{(2 pt.)} Look up each of the terms in the equations above and describe them.
    \item \textbf{(2 pt.)} Describe what each equation is saying and why these are PDEs.
    \item \textbf{(2 pt.)} In the absence of all charges we will have $\vecfieldJ=\zerovec$ and $\rho=0$.  Using that and the following two facts
    \[
    \veclaplace \vecfieldV = \grad (\grad \cdot \vecfieldV) - \grad \times (\grad \times \vecfieldV) \qquad \textrm{and} \qquad \grad \times \frac{\partial \vecfieldV}{\partial t} = \frac{\partial}{\partial t} (\grad \times \vecfieldV),
    \]
    derive the vector wave equations for light
    \[
    \left( - \veclaplace + \mu_0 \epsilon_0 \frac{\partial^2 }{\partial t^2}\right) \vecfieldE = \zerovec
    \]
    and
    \[
    \left( - \veclaplace + \mu_0 \epsilon_0 \frac{\partial^2 }{\partial t^2}\right) \vecfieldB = \zerovec
    \]
    \item \textbf{(1 pt.)} From the equations you derived, determine the wave speed of light in the vacuum, $c_0$.
\end{enumerate}
\end{problem}




\end{document}
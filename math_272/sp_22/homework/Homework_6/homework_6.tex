%%%%%%%%%%%%%%%%%%%%%%%%%%%%%%%%%%%%%%%%%%%%%%%%%%%%%%%%%%%%%%%%%%%%%%%%%%%%%%%%%%%%
% Document data
%%%%%%%%%%%%%%%%%%%%%%%%%%%%%%%%%%%%%%%%%%%%%%%%%%%%%%%%%%%%%%%%%%%%%%%%%%%%%%%%%%%%
\documentclass[12pt]{article} %report allows for chapters
%%%%%%%%%%%%%%%%%%%%%%%%%%%%%%%%%%%%%%%%%%%%%%%%%%%%%%%%%%%%%%%%%%%%%%%%%%%%%%%%%%%%
\usepackage{preamble}
\newcommand{\innprod}[2]{\left\langle #1, #2\right\rangle}
\begin{document}

\begin{center}
   \textsc{\large MATH 272, Homework 6}\\
   \textsc{Due April 29$^\textrm{th}$}
\end{center}
\vspace{.5cm}



\begin{problem} \textbf{(11 pts)} Are you sure you understand what constitutes a vector space? What about an inner product? Let's see a few examples. Please work through each part of the question.

Given an a vector space $V$ with an inner product $\innprod{-}{-}$, we can always define a \emph{norm} (or \emph{energy}) by taking $v\in V$ and putting
\[
\|v\|^2 = \innprod{v}{v}.
\]
\emph{P.S. both l's seen here are short for Henri Lebesgue.}
\begin{enumerate}[(a)]
\item \textbf{(3 pts)} (Finite dimensional inner product space) Consider the vector space $\C^3$ with the Hermitian inner product $\innprod{-}{-}$.
\begin{itemize}
\item Compute the norm of 	
\[
		\vecu = \begin{pmatrix} i \\ 1 \\ 0 \end{pmatrix}.
	\]
\item Compute the norm of
\[
\vecv = \begin{pmatrix} 1 \\ -i \\ 0 \end{pmatrix}.
\]
\item Compute the inner product $\innprod{\vecu}{\vecv}$.
\item Provide an example of a basis for $\C^3$. 
\end{itemize}

\item \textbf{(4 pts)} (Countably infinite dimensional inner product space) Consider the space of \emph{square summable} (\emph{finite energy}) sequences $\ell^2(\C)$. That is, an element of the $\ell^2(\C)$ is a sequence of complex numbers $\{a_n\}_{n=0}^\infty$ with an inner product defined by
\[
\innprod{\{a_n\}}{\{b_n\}} = \sum_{n=0}^\infty a_n^* b_n.
\]
and we require the vectors to have finite energy which gives us the definition:
\[
\ell^2(\C) \coloneqq \{ \{a_n\} ~|~ a_n \in \C ~\textrm{such that $\|\{a_n\}\|<\infty$}\}.
\]
\begin{itemize}
\item Show that the sequence $a_n = \frac{1}{n+1}$ is in $\ell^2(\C)$. What is its norm? \emph{Please use WolframAlpha!}
\item Show that the sequence $b_n = \frac{1}{2^n}$ is in $\ell^2(\C)$. What is its norm? 
\item Compute the inner product $\innprod{\{a_n\}}{\{b_n\}}$. \emph{Please use WolframAlpha!}
\item Provide an example of a basis for $\ell^2(\C)$.
\end{itemize}

\item \textbf{(4 pts)} (Functional inner product space) Consider the space of \emph{square integrable} (\emph{finite energy}) functions $L^2(\Omega)$ on the region $\Omega=[0,1]$. That is, an element of the $L^2(\Omega)$ is a complex-valued function $f \colon [0,1] \to \C$ of complex numbers with an inner product defined by
\[
\innprod{f}{g} = \int_\Omega f^*g d\Omega = \int_0^1 f^*(x) g(x)dx.
\]
and we require the vectors to have finite energy which gives us the definition:
\[
L^2(\Omega) \coloneqq \{ f\colon \Omega \to \C ~|~ \textrm{such that $\|f\|<\infty$}\}.
\]
\begin{itemize}
\item Show that the function $f(x)= e^{i2\pi x}$ is in $L^2(\Omega)$. What is its norm? 
\item Show that the sequence $g(x) = \sin(2\pi x)$ is in $L^2(\Omega)$. What is its norm?
\item Compute the inner product $\innprod{\{a_n\}}{\{b_n\}}$. \emph{Please use WolframAlpha!}
\end{itemize}
\end{enumerate}
\end{problem}

\vspace*{1cm}
\begin{problem} \textbf{(4 pts)} Consider the real function $f(x)=1 \in L^2(\Omega)$ on the domain $\Omega = [0,L]$.
	\begin{enumerate}[(a)]
		\item \textbf{(1 pts)} What is the norm of $f$, $\|f\|$?
		\item \textbf{(1 pts)} Normalize $f(x)$.
		\item \textbf{(2 pts)} Find a nonzero normalized polynomial of degree $\leq 1$ that is orthogonal to $f(x)$.
	\end{enumerate}
\end{problem}

\vspace*{1cm}
\begin{problem}
\textbf{(6 pts)} A wavefunction $\Psi$ for a particle in the 1-dimensional box $\Omega=[0,1]$ is a member of the space of finite energy functions $L^2(\Omega)$. Recall that $\Psi$ could be written as a superposition of normalized states
	\[
	\psi_n(x) = \sqrt{2} \sin\left(n\pi x\right).
	\]
	That is,
	\[
	\Psi(x) = \sum_{n=1}^\infty a_n \psi_n(x),
	\]
	for some choice of the coefficients $a_n$.
	\begin{enumerate}[(a)]
		\item \textbf{(3 pts)} Let $a_n = \frac{\sqrt{6}}{n\pi}$. Show that $\Psi(x)$ is normalized. \emph{Hint: first, use orthogonality of the states $\psi_n(x)$ to your advantage. Then you will need to know what an infinite series evaluates to. Use a tool like WolframAlpha to evaluate this series.}
		\item \textbf{(2 pts)} Note that we can approximate $\Psi(x)$ by taking a finite sum approximation up to some chosen $N$ by
		\[
			\Psi(x) \approx \sum_{n=1}^N a_n \psi_n(x).
		\]
		Plot the approximation of $\Psi(x)$ for $N=1,5,25,50,100$.  
\item \textbf{(1 pts)} Describe the wave function $\Psi$.
		\end{enumerate}
\end{problem}

\vspace*{1cm}
\begin{problem}
\textbf{(9 pts)}  When making a measurement of the position of the particle, we will use the \emph{position operator} $x$.  This is the same as the variable $x$ in the original problem statement, but it is also an operator! Similarly, we could measure the momentum of a particle using the \emph{momentum operator $p$}. The potential $V(x)$ is a function of the position operator and it, itself, is an operator. Lastly, I should mention the \emph{Hamiltonian operator} $H=\frac{p^2}{2M} + V$. 

What I mean here by operator is that the operators defined above are linear transformations $\mathcal{L} \colon L^2(\Omega) \to L^2(\Omega)$. \emph{Actually, I am lying to you. It is true that $x\colon L^2(\Omega) \to L^2(\Omega)$ but you have to be careful which spaces you are talking about when it comes to the momentum operator $p$. Do not worry, this understanding that the underlying space is $L^2(\Omega)$ is good enough!}
   \begin{enumerate}[(a)]
		\item \textbf{(1 pts)} True or false. A self-adjoint operator has a real-valued spectrum.
   		\item \textbf{(1 pts)} Show that the position operator $x$ is self-adjoint.
   		\item \textbf{(2 pts)} We can compute the expected position of a particle with wavefunction $\Psi(x)$ by computing
   		\[
   		\mathbb{E}[x]=\innprod{\Psi}{x\Psi}.
   		\]
   		Let $\Psi(x) = \psi_1(x)$, compute $\mathbb{E}[x]$. This value $\mathbb{E}[x]$ tells you where we expect to find the particle on average. 
        \item \textbf{(1 pts)} In fact, any real valued function $V(x)$ of the position operator $x$ is also self-adjoint. Make a quick argument on why this must be true.
		\item \textbf{(2 pts)} We define the \emph{momentum operator} $p = -i\hbar \frac{d}{dx}$. Using integration by parts, show that this operator is self-adjoint.
		\item \textbf{(2 pts)} Argue that the Hamiltonian operator is self-adjoint. \emph{Hint: look at how $H$ is defined. Don't show more work than you need to. This part should be short and sweet.}
   	\end{enumerate}
\end{problem}
\begin{remark}
The fact that all measurements in quantum mechanics are self-adjoint operators motivates the Dirac bra-ket notation which looks like this:
\[
\mathbb{E}[x] = \langle \Psi \vert ~x~ \vert \Psi \rangle.
\]
This is because $x$ can act on either side!
\end{remark}

\vspace*{1cm}
\begin{problem}
\textbf{(6 pts)} Let's explore two subspaces of $L^2(\Omega)$ for $\Omega = [0,1]$. One built by exponentials and one built by sines and cosines.
	\begin{enumerate}[(a)]
		\item \textbf{(3 pts)} Show that the set of functions $\{1,\sqrt{2}\cos(2n \pi x), \sqrt{2}\sin(2n \pi x)\}$ for integers $n\geq 1$ are orthonormal in $L^2(\Omega)$ with $\Omega = [0,1]$.
		\item \textbf{(3 pts)} Recall Euler's formula: $e^{ix} = \cos(x)+i\sin(x)$.  Argue that 
\[
\operatorname{Span}\{e^{i2n\pi x} ~|~ n \in \Z\} \qquad \textrm{and} \qquad \operatorname{Span}\{1,\sqrt{2}\cos(2n \pi x), \sqrt{2}\sin(2n \pi x) ~|~ m,n \in \Z, ~ n\geq 1\}
\]
are the same subspace. \emph{Hint: can you just show that a given $e^{i2n\pi x}$ corresponds to a pair $\cos(2n \pi x)$ and $\sin(2n \pi x)$ using Euler's formula? Also, you should use the fact that sine is an odd function and cosine is an even function.}
	\end{enumerate}
\end{problem}
\vspace*{1cm}

\begin{problem}
\textbf{(10 pts)} It turns out that the set of complex exponentials $\{e^{i2n\pi x} ~|~ n\in \Z\}$ is an orthonormal basis for $L^2(\Omega)$ for $\Omega = [0,1]$ (and thus by the previous problem, so are the sines and cosines). This is the foundational insight of the Fourier transform/series. Let us denote by $\phi_n(x) = e^{i2n\pi x}$.

Let $f \in L^2(\Omega)$, then the \emph{Fourier transform} of $f$ is the function $\hat{f} \colon \Z \to \C$ defined by
\[
\hat{f}(n) = \innprod{f}{\phi_n}.
\]
Then the \emph{Fourier series of $f$} is the series
\[
f(x) = \sum_{n=-\infty}^\infty \hat{f}(n) \phi_n(x)
\]
\begin{enumerate}[(a)]
\item \textbf{(2 pts)} Let $V$ be an $n$-dimensional vector space. Recall that you can write a vector $v \in V$ in terms of an orthonormal basis $v_1,\dots,v_n$ for $V$ by
\[
v = \sum_{j=1}^n \langle v,v_j\rangle v_j.
\]
Explain why the Fourier series is the same concept just for an infinite dimensional vector space.
\item \textbf{(2 pts)} Compute the Fourier transform of the function 
\[
f(x) = \begin{cases} 0, & 0\leq x \leq \frac{1}{4} \\
	1, &  \frac{1}{4} \leq x \leq \frac{3}{4} \\
 0, & \frac{3}{4}\leq x \leq 1 \end{cases}
\]
\emph{Please note that you will want to consider the term $\hat{f}(0)$ separately from the others.}
\item \textbf{(1 pts)} Write out the Fourier series of $f$. 
\item \textbf{(2 pts)} Convert the Fourier series of $f$ into a series of sine and cosine functions using Euler's formula. \emph{Hint: you can also just use this from the beginning and put}
\[
\hat{f}(n) = \langle f,\phi_n\rangle = \int_0^1 f(x)\cos(2n\pi x) dx + i \int_0^1 f(x)\sin(2n\pi x)dx.
\]
\item \textbf{(2 pts)} Plot an approximation of the Fourier series of sine and cosine functions up to $N=1,3,5,10,50,100$ only on the domain $\Omega$. Please graph your approximations to the original $f$. Describe what is happening with your approximations.
\item \textbf{(1 pts)} What happens if you plot your Fourier series over all of $\R$?
\end{enumerate}
\end{problem}

\vspace*{1cm}
\begin{problem}
\textbf{(10 pts)} One advantage of the Fourier transform is that we can use it to solve differential equations in a new way that also allows us to consider far more general forcing terms. The basic idea is that the Fourier transform converts a differential equation into an algebraic equation. Let us see how this works.

First, let me say that we will be working over $L^2(\Omega)$ with $\Omega = [0,1]$. Recall the Dirac delta $\delta(x)$ which satisfies the properties
\[
\int_0^1 \delta(x-x_0) dx = 1 \qquad \textrm{and} \qquad \int_0^1 \delta(x-x_0) f(x) dx = f(x_0)
\]
whenever $x_0 \in \Omega$. You can imagine the Dirac delta $\delta(x- x_0)$ as a probability distribution where all the mass is located at a single point $x_0$.
\begin{enumerate}[(a)]
\item \textbf{(2 pts)} Show that for any $f$ satisfying the Dirichlet boundary conditions $f(0)=0$ and $f(1)=0$ that for $n\neq 0$
\[
\left\langle \frac{d}{dx} f , \phi_n \right\rangle = i2 \pi n \hat{f}(n).
\]
\emph{Hint: use integration by parts.}
\item \textbf{(2 pts)} Consider the Poisson (elastic deformation) problem
\[
\begin{cases}
\frac{d^2}{dx^2} f(x) = \delta(x-x_0) \\
f(0)=0=f(1) & \textrm{as boundary conditions.}
\end{cases}
\]
Apply the Fourier transform to both sides to show that we have for $n\neq 0$
\[
\hat{f}(n) =  -\frac{e^{i2n\pi x_0}}{4\pi^2 n^2}.
\]
\item \textbf{(2 pts)} Hence, the solution to the problem is just the Fourier series
\[
f(x)=c_0 + c_1 x + \sum_{n=-\infty}^{-1} -\frac{e^{in \pi x_0}}{4\pi^2 n^2} \phi_n + \sum_{n=1}^\infty -\frac{e^{in \pi x_0}}{4\pi^2 n^2} \phi_n.
\]
Let $x_0=1/2$ and convert this Fourier series to a real valued Fourier series in terms of sines and cosines. 
\item \textbf{(1 pts)} Determine the constants $c_0$ and $c_1$ using the boundary conditions on $f$.
\item \textbf{(2 pts)} Plot your approximation to the Fourier series for $N=1,3,5,10,50,100$. 
\item \textbf{(1 pts)} Explain your result given the following interpretation: You can imagine that $\delta(x-1/2)$ is a point mass of mass 1 placed exactly at $x=1/2$ on your elastic rod $\Omega$. \emph{Hint: does this look what what you'd imagine the deformation to look like?}
\end{enumerate}
\end{problem}


\begin{problem}
\textbf{(Bonus 10 pts.)} Show that the Fourier transform is an isomorphism between $\ell^2(\C)$ and $L^2(\Omega)$ where $\Omega=[0,1]$. For an extra 5 points, argue that this is true if $\Omega$ is any closed interval.
\end{problem}


\end{document}
%%%%%%%%%%%%%%%%%%%%%%%%%%%%%%%%%%%%%%%%%%%%%%%%%%%%%%%%%%%%%%%%%%%%%%%%%%%%%%%%%%%%
% Document data
%%%%%%%%%%%%%%%%%%%%%%%%%%%%%%%%%%%%%%%%%%%%%%%%%%%%%%%%%%%%%%%%%%%%%%%%%%%%%%%%%%%%
\documentclass[12pt]{article} %report allows for chapters
%%%%%%%%%%%%%%%%%%%%%%%%%%%%%%%%%%%%%%%%%%%%%%%%%%%%%%%%%%%%%%%%%%%%%%%%%%%%%%%%%%%%
\usepackage{preamble}

\begin{document}

\begin{center}
   \textsc{\large MATH 271, Worksheet 8}\\
   \textsc{Linear transformations, matrices, and linear systems.}
\end{center}
\vspace{.5cm}

\begin{problem}
    Note that any linear transformation $T\colon \R^m \to \R^n$ is fully understood by its action on the vectors
    \[
        \xhat_1 = \begin{pmatrix} 1 \\ 0 \\ 0 \\ \vdots \\ 0 \end{pmatrix}, \quad \xhat_2 = \begin{pmatrix} 0 \\ 1 \\ 0 \\ \vdots \\ 0 \end{pmatrix}, ~\dots \quad, \xhat_m = \begin{pmatrix} 0 \\ 0 \\ 0 \\ \vdots \\ 1 \end{pmatrix},
    \]
    and note that all these vectors $\xhat_j \in \R^m$. In particular, we have
    \begin{align*}
        T(\xhat_1) &= \vecv_1\\
        T(\xhat_2) &= \vecv_2\\
                    & \vdots\\
        T(\xhat_m) &= \vecv_m,\\
    \end{align*}
    where the vectors $\vecv_j \in \R^n$ and as such can be written as column vectors with $n$ entries.
    \begin{enumerate}[(a)]
        \item As per usual, let $\xhat = \begin{pmatrix} 1 \\ 0 \end{pmatrix}$ and let $\yhat = \begin{pmatrix} 0 \\ 1 \end{pmatrix}$. Let $A \colon \R^2 \to \R^2$ be given by
        \[
            A(\xhat) = 5 \xhat + 6\yhat = \begin{pmatrix} 5 \\ 6 \end{pmatrix}
        \]
        and
        \[
            A(\xhat) = 2 \xhat - 3 \yhat = \begin{pmatrix} 2 \\ -3 \end{pmatrix}.
        \]
        If I wanted to transform an arbitrary vector $\vecu = u_1 \xhat + u_2 \yhat = \begin{pmatrix} u_1 \\ u_2 \end{pmatrix}$, how can I use the definition of $A$ acting on unit vectors?
        \item Determine a matrix of numbers $[A] = \begin{pmatrix} a_{11} & a_{12} \\ a_{21} & a_{22} \end{pmatrix}$ that captures this linear transformation through matrix-vector multiplication. 
        \item How do the columns of $[A]$ relate to $A(\xhat)$ and $A(\yhat)$?
        \item Now, how can I think of $[A]\vecu$ as describing a linear combination of the columns of $[A]$?
    \end{enumerate}
\end{problem}

\begin{problem}
    Repeat the steps in Problem 1 but with the transformation $B \colon \R^2 \to \R^3$ given by
    \[
        B(\xhat) = \xhat + \yhat + \zhat \qquad \textrm{and} \qquad B(\yhat) = -\xhat + \yhat - \zhat.
    \]
\end{problem}

\begin{problem}
    Repeat the steps in Problem 1 but with the transformation $C\colon \R^3 \to \R^2$ given by
    \[
        C(\xhat) = \yhat, \qquad C(\yhat) = \xhat, \qquad C(\zhat) = \xhat.
    \]
\end{problem}

\begin{problem}
Let
\vspace*{.25cm}
\[
[M]= \begin{pmatrix} 1 & 2\\ 0 & 1 \end{pmatrix} \quad 
[P]=\begin{pmatrix} 1 \\ 1 \end{pmatrix} \quad 
[Q]=\begin{pmatrix} 2 & 1 \end{pmatrix} \quad
[R] = \begin{pmatrix} 1 & 1 \\ 1 & 0 \end{pmatrix} \quad
[S]=\begin{pmatrix} 3 & 3 & 3 \\ 3 & 3 & 3 \end{pmatrix}
\]
\vspace*{.25cm}
\begin{enumerate}[(a)]
    \item Compute the following matrix products (when possible) and state which multiplications are not possible.
\[
[M][M], \qquad [P][P], \qquad [Q][P], \qquad [M][S], \qquad [S][M].
\]
    \item Compute the following:
    \begin{enumerate}[i.]
        \item $[A]=[P][Q]$;
        \item $[B]=[Q]^T[P]^T$. Is this equal to $([P][Q])^T$?
        \item $[C] = [M][R] - [R][M]$. Do these matrices commute?
    \end{enumerate}
\end{enumerate}
\end{problem}

\begin{problem}
    The linear transformation $H\colon \R^2 \to \R^2$ given by
    \[
        H(\xhat) = \yhat \qquad \textrm{and} \qquad H(\yhat) = \xhat,
    \]
    has some nice properties. 
    \begin{enumerate}[(a)]
        \item In some sense, $H$ is the square root of $1$ in that $H^2 = H \circ H = 1$. Show that this is true.
        \item Write down a matrix representation for $H$ and denote it by $[H]$.
        \item Consider a linear combination of matrices 
        \[
            [\eta] = x[I] + y[H],
        \]
        where $[I]$ is the $2\times 2$ identity matrix. Compute $[\eta]^2$.  
    \end{enumerate} 
\end{problem}

\begin{problem} 
Consider the system of linear equations:
\begin{align*}
    x + 2y &= 3\\
    x + y  &= 3
\end{align*}
\begin{enumerate}[(a)]
    \item Write this system in the form:
    \[
    [A]\vec{\boldsymbol{x}} = \vec{\boldsymbol{y}}
    \]
    \item Row reduce to find a the solution $\vec{\boldsymbol{x}}$.
\end{enumerate}
\end{problem}

\begin{problem}
Consider the system of linear equations:
\begin{align*}
    3x+2y+0z&=5\\
    1x+1y+1z&=3\\
    0x+2y+2z&=4.
\end{align*}
\begin{enumerate}[(a)]
    \item Write the augmented matrix $M$ for this system of equations.
    \item Use row reduction to get the augmented matrix in row-echelon form.
    \item Determine the solution to the system of equations.
\end{enumerate}
\end{problem}

\begin{problem}
Consider the equation
\[
\begin{pmatrix} 1 & 3 & 4 \\ 2 & 9 & 9 \\ 1 & 5 & 5 \end{pmatrix} \begin{pmatrix} x \\ y \\ z \end{pmatrix} = \begin{pmatrix} 8 \\ 20 \\ 11 \end{pmatrix}.
\]
Does this equation have a solution or not? If so, determine the solution.
\end{problem}

\begin{problem} 
Consider the matrix
\[
[A] = \begin{pmatrix} 3 & 1 \\ 6 & 2 \end{pmatrix}.
\]
Determine the nullspace of $[A]$.
\end{problem}


\end{document}
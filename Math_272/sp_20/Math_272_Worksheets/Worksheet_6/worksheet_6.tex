%%%%%%%%%%%%%%%%%%%%%%%%%%%%%%%%%%%%%%%%%%%%%%%%%%%%%%%%%%%%%%%%%%%%%%%%%%%%%%%%%%%%
% Document data
%%%%%%%%%%%%%%%%%%%%%%%%%%%%%%%%%%%%%%%%%%%%%%%%%%%%%%%%%%%%%%%%%%%%%%%%%%%%%%%%%%%%
\documentclass[12pt]{article} %report allows for chapters
%%%%%%%%%%%%%%%%%%%%%%%%%%%%%%%%%%%%%%%%%%%%%%%%%%%%%%%%%%%%%%%%%%%%%%%%%%%%%%%%%%%%
\usepackage{preamble}

\newcommand{\curvegamma}{\boldsymbol{\vec{\gamma}}}
\newcommand{\tangentgamma}{\boldsymbol{\dot{\vec{\gamma}}}}
\newcommand{\normalgamma}{\boldsymbol{\ddot{\vec{\gamma}}}}
%\newcommand{\forcevec}{\boldsymbol{\vec{F}}}
\newcommand{\rvec}{\boldsymbol{\vec{r}}}
\newcommand{\grad}{\boldsymbol{\vec{\nabla}}}
\usepackage{multicol}

\begin{document}

\begin{center}
   \textsc{\large MATH 272, Worksheet 6}\\
   \textsc{Curves, scalar fields, and calculus.}
\end{center}
\vspace{.5cm}

\begin{problem}
    Consider the curve 
    \[
    \curvegamma(t) = \begin{pmatrix} (5+3\cos(8t))\cos(t) \\ (5+3\cos(8t))\sin(t) \\ 3\sin(8t) \end{pmatrix}.
    \]
    \begin{enumerate}[(a)]
        \item Plot this curve from $t=0$ to $t=2\pi$.
        \item Compute the tangent (velocity) vector $\tangentgamma(t)$.
        \item Compute the normal (acceleration) vector $\normalgamma(t)$.
        \item Compute the following 
        \[
        \int_{\curvegamma} \left| \normalgamma(t) \right| dt,
        \]
        which is closely related to the total curvature of the curve.  Indeed, this would be the total force applied to an object of mass $m=1$.
    \end{enumerate}
\end{problem}

\begin{problem}~
\begin{enumerate}[(a)]
\item Write an equation for a curve
\[
\curvegamma \colon \R \to \R^3,
\]
satisfying:
\begin{itemize}
    \item Starts with $\curvegamma(0)=\begin{pmatrix} 0 \\ 0 \\ 0 \end{pmatrix}$.
    \item Ignoring the $z$-component, makes a spiral emanating from the origin.
    \item Moves upward at a constant rate in the $z$-direction.
\end{itemize}
Plot this curve that you made to verify that it is correct.
\item Find the tangent vector $\tangentgamma(t)$ to this curve.
\item Find the normal vector $\normalgamma(t)$ to this curve.
\end{enumerate}
\end{problem}

\begin{problem} Consider the two dimensional scalar function 
\[
f(x,y)=e^{\frac{xy}{x^2+y^2-1}}.
\]  
    \begin{enumerate}[(a)]
        \item Plot the graph of this function $(x,y,f(x,y))$ for $x^2+y^2<1$.  
        \item Plot the graph for $x^2+y^2>1$.  
        \item Compute $\frac{\partial f}{\partial x}$ and $\frac{\partial f}{\partial y}$.
        \item Compute the gradient vector field $\grad f$.  
        \item Plot the vector field $\grad f$. 
    \end{enumerate}
\end{problem}

\begin{problem}~
\begin{enumerate}[(a)]

\item    Write the equation for a scalar function 
    \[
    f\colon \R^2 \to \R,
    \]
    satisfying,
    \begin{itemize}
        \item Has positive $\frac{\partial f}{\partial x}$ everywhere.  
        \item Has negative $\frac{\partial f}{\partial y}$ everywhere.  
    \end{itemize}
    (\emph{Hint: it may help to try adding single variable functions together. That is, let $f(x,y)=u(x)+v(y)$.})
    \item Find the gradient of the function you chose.
\end{enumerate}
\end{problem}

\begin{problem}
    Using your curve $\curvegamma$ from Problem 2 with time starting at $t_0=0$ and $t_1=2\pi$ and your scalar field $f$ from Problem 4, compute the following. 
    \[
    \int_{\curvegamma} f(\gamma)d\curvegamma = \int_{t_0}^{t_1} f(\gamma(t))\left|\tangentgamma(t)\right|dt.
    \]
\end{problem}

\begin{problem} 
We have briefly discussed the idea of \emph{work} (change in energy) before and wrote
\[
W = \forcevec \cdot \rvec,
\]
where $\forcevec$ was a constant force and $\rvec$ was a straight line displacement.

Now, we can write the real version of this. The work done on a particle moving along a curve $\curvegamma(t)$ that starts at time $t_0$ and ends at time $t_1$ experiencing a (spatially dependent) force field $\forcevec(x,y,z)$ is
\[
W = \int_{\curvegamma} \forcevec(\curvegamma) \cdot d\curvegamma = \int_{t_0}^{t_1} \forcevec(\curvegamma(t))\cdot \tangentgamma(t) dt.
\]
Compute the work given the following
\[
\forcevec(x,y,z) = \begin{pmatrix} x^2 \\ y \\ \sqrt{z} \end{pmatrix} \qquad \textrm{and} \qquad \curvegamma(t) = \begin{pmatrix} t \\ t^2 \\ t^3 \end{pmatrix}.
\]
\end{problem}

\begin{problem}
For the following functions, plot the level sets for $c=-1$, $c=0$, and $c=1$. 
\begin{multicols}{2}
\begin{enumerate}[(a)]
    \item For just $c=1$, plot the level set for $E(x,y,z) = \frac{x^2}{25} + \frac{y^2}{16} + \frac{z^2}{9}$.
    \item $f(x,y,z) = xyz$.
    \item $g(x,y,z) = e^x-y^2-z^2$.
    \item $h(x,y,z) = \sin(x)+\cos(y)-\tanh(z)$.
    \item $p(x,y,z) = \sin^2(x)+\sin^2(y)-\frac{1}{2}\sin(z)$.
    \item $q(x,y,z) = x^2+xy+y^2+sin(yz)$.
    \item One of your own choosing.
\end{enumerate}
\end{multicols}
\end{problem}


\begin{problem}
    Compute the integral of the scalar fields given in Problem 7 over the following regions.
  \begin{multicols}{2}
  \begin{enumerate}[(a)]
      \item $\Omega_1$ is the unit cube.
      \item $\Omega_2$ is the unit cube along with the rectangular prism given by $2<x<3$, $2<y<4$, and $-1<z<1$.
      \item $\Omega_3$ is the portion of the unit cube left over after slicing diagonally by the plane given by the equation $x+y+z=\frac{1}{2}$. Specifically, take the region left \underline{under} the plane and in the unit cube.
  \end{enumerate}
  \end{multicols}  
\end{problem}




\end{document}

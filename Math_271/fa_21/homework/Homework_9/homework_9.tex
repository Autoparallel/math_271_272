%%%%%%%%%%%%%%%%%%%%%%%%%%%%%%%%%%%%%%%%%%%%%%%%%%%%%%%%%%%%%%%%%%%%%%%%%%%%%%%%%%%%
% Document data
%%%%%%%%%%%%%%%%%%%%%%%%%%%%%%%%%%%%%%%%%%%%%%%%%%%%%%%%%%%%%%%%%%%%%%%%%%%%%%%%%%%%
\documentclass[12pt]{article} %report allows for chapters
%%%%%%%%%%%%%%%%%%%%%%%%%%%%%%%%%%%%%%%%%%%%%%%%%%%%%%%%%%%%%%%%%%%%%%%%%%%%%%%%%%%%
\usepackage{preamble}


\newcommand{\vhat}{\boldsymbol{\hat{v}}}
\newcommand{\ehat}{\boldsymbol{\hat{e}}}
\newcommand{\Span}{\operatorname{Span}}

\begin{document}

\begin{center}
   \textsc{\large MATH 271, Homework 9}\\
   \textsc{Due December 3$^\textrm{rd}$}
\end{center}
\vspace{.5cm}

\begin{problem}
Compute the following:
\begin{enumerate}[(a)]
    \item
    \[
    [A]=\begin{pmatrix} 1& 1& 1 \end{pmatrix}
    \begin{pmatrix} 2\\ 1\\ 3 \end{pmatrix}.
    \]
    \item
    \[
    [B]=\begin{pmatrix} 1& 2& 3& 4\\ 5& 6& 7& 8\\ 9& 10& 11& 12\end{pmatrix}
    \begin{pmatrix} 3& 2\\ 2& 3\\ 3& 2\\ 2& 3\end{pmatrix}
    \]
    \item Take
    \[
    [M]=\begin{pmatrix} 10& 15\\ 20& 10 \end{pmatrix}
    \]
    and
    \[
    [N]=\begin{pmatrix} 1 & 2\\ 2& 1\end{pmatrix}.
    \]
    Compute $[M][N]$ and $[N][M]$ to see that matrices do not commute in general.
\end{enumerate}
\end{problem}

\begin{problem}
A linear transformation $T\colon \R^3 \to \R^3$ is given by the matrix
\[
[T]= \begin{pmatrix}
1& 2& 0\\
2& 1& 2\\
0& 2& 1
\end{pmatrix}.
\]
\begin{enumerate}[(a)]
    \item Compute how $T$ transforms the standard basis elements for $\R^3$. That is, find
    \[
    T(\ehat_1), \qquad
    T(\ehat_2), \qquad
    T(\ehat_3)
    \]
    and relate these values to the columns of $[T]$.
    \item Is the transformed basis $T(\ehat_1)$, $T(\ehat_2)$, and $T(\ehat_3)$ linearly independent? Do these vectors form a basis for $\R^3$?
    \item If we apply this linear transformation to the unit cube (that is, all points who have $(x,y,z)$ coordinates with $0\leq x \leq 1$, $0\leq y \leq 1$, and $0\leq z \leq 1$), what will the volume of the transformed cube be? (\emph{Hint: use the determinant.})
\end{enumerate}
\end{problem}

\begin{problem}~
\begin{enumerate}[(a)]
    \item Show that for any $2\times 2$-matrix that the sign of the determinant changes if either a row or column is swapped. \emph{Note: this is true for square matrices of any size}.
    \item Show that for any $2\times 2$-matrix that multiplying a column by a constant scales the determinant by that constant as well. \emph{Note: this is true for square matrices of any size.}
    \item Show that for any $2\times 2$-matrix that adding a scalar multiple one column to the other will not change the determinant. \emph{Note: this is true in broader generality. In fact, adding linear combinations of columns to another column will not change the determinant.}
    \item Using these facts, argue why a square matrix with columns that are linearly dependent must have a determinant of zero.
\end{enumerate}
\end{problem}

\begin{problem}
Consider the equation
\[
[A]\vecv = \zerovec,
\]
where
\[
[A] = \begin{pmatrix} 0 & 1 & 0 \\ 1 & 0 & 1 \\ 0 & 1 & 0 \end{pmatrix}.
\]
\begin{enumerate}[(a)]
    \item Are the columns of $[A]$ linearly independent or dependent? Explain.
    \item What vector(s) $\vecv$ satisfy this equation? In other words, what is $\ker[A]$?
    \item Using what you found above, what must $\det[A]$ be equal to? \emph{Hint: you do not need to compute the determinant!}
\end{enumerate}
\end{problem}

\begin{problem}
Compute the following.
\begin{enumerate}[(a)]
    \item
    \[
    \det[A]=\left| \begin{array}{ccc}
    -3& 1 & 5\\
    -3& 4 & 2\\
    -3& 2 & 1
    \end{array}\right|
    \]
    \item
    \[
    \det[B]=\left| \begin{array}{ccc}
    1& 2& 3\\
    4& 5& 6\\
    7& 8& 9
    \end{array}\right|
    \]
    \item Compute $\det([A][B])$ using properties of the determinant. \emph{Hint: this should be very quick to do. Do not compute the product of the matrices $[A]$ and $[B]$!}
    \item Compute $\tr([C])$ and $\tr([D])$ where
    \[
[C]=\begin{pmatrix} 1 & 0 & 2 \\ 2  & 1 & 3 \\ -2 & -2 & 0 \end{pmatrix} \qquad \textrm{and} \qquad [D]=\begin{pmatrix} -3 & 1 & 1 \\ 2 & -2 & 4 \\ -1 & -1 & -1 \end{pmatrix}.
\]
    \item Compute $\tr([C][D])$ and compare it to $\tr([D][C])$.
\end{enumerate}
\end{problem}



\begin{problem}
Consider some linear transformation $T\colon \R^n \to \R^m$.  Let $\vecv_1, \dots, \vecv_k$ be vectors in the kernel $\ker(T)$.
\begin{enumerate}[(a)]
    \item Show that the span of these vectors is also in the kernel of $T$.
    \item How many linearly independent vectors can be in the kernel? Give bounds using $n$ and $m$.
\end{enumerate}
\end{problem}

\begin{problem}~
    Suppose that the operator $T\colon V \to V$ has a nonzero kernel (e.g., some vector $\vecv$ other than $\zerovec$ is in the kernel). Prove that $T$ has no inverse. \emph{Hint: this means you can construct a vector that is not in the image of $T$!}
\end{problem}

\begin{problem}
The previous problem will be very helpful for these two parts.
\begin{enumerate}[(a)]
    \item Let $T\colon V\to V$ be an operator such that $\det [T]=0$. Explain why there exists a solution to the homogeneous equation $S\vecu = \zerovec.$
    \item Suppose $S\colon V \to V$ is another operator such that $\det [S] \neq 0$. Explain why there exists a solution to the inhomogeneous equation $S\vecv = \vecw$ for any $\vecw \in V$.
\end{enumerate}
\end{problem}

\begin{problem}
Prove that the eigenvectors with eigenvalue 0 of an operator $T\colon V \to V$ correspond to vectors in the kernel of $T$.
\end{problem}

\begin{problem}
Consider the linear operator $J\colon \R^2 \to \R^2$ defined by
\[
J(\ehat_1) = \ehat_2 \qquad \textrm{and} \qquad J(\ehat_2) = -\ehat_1.
\]
\begin{enumerate}[(a)]
    \item Show that operator polynomial
    \[
    P(J) \coloneqq J^2 + I \colon \R^2 \to \R^2
    \]
    annihilates $\R^2$. Or, said another way, show that every $\vecv \in \R^2$ is in the kernel of $P(J)$.
    \item Show that the characteristic polynomial of $J$ is
    \[
    p(\lambda) = \lambda^2 + 1.
    \]
    Does this coincide with $P(J)$? \emph{If need be, use your matrix representation $[J]$ from the previous homework.}
    \item Compute the eigenvalues $\lambda_1$ and $\lambda_2$ of $J$.
    \item Compute the corresponding eigenvectors of $J$.
    \item If we don't allow for complex scalars, $J$ has no eigenvalues. However, $J^2$ does have only real eigenvalues. Using (a), show that $J$ has eigenvalue $\lambda=-1$ with eigenvectors $\ehat_1$ and $\ehat_2$.
    \item (Bonus) Can you argue that any nonzero rotation of $\R^2$ must have imaginary eigenvalues?
\end{enumerate}
\end{problem}

\begin{problem} For this problem, we will consider eigenvectors of three operators that act on the space of analytic functions $C^\omega(\C)$. Your goal should be to realize that these correspond to differential equations you have seen before.
\begin{enumerate}[(a)]
    \item Take the operator $\frac{d}{dx}\colon C^\omega(\C)\to C^\omega(\C)$. Show that the exponential $e^{kx}\in C^\omega(\C)$ is an eigenvector (or eigenfunction) with eigenvalue $k$. Write down the corresponding ODE. \emph{Hint: just by doing the problem properly, you will probably write down the ODE.}
    \item Take the operator $\frac{d^2}{dx^2}\colon C^\omega(\C)\to C^\omega(\C)$. Show that there are two eigenfunctions $e^{i\omega x}$ and $e^{-i\omega x}$ with eigenvalue $-\omega^2$. Write down the corresponding ODE. \emph{Hint: just by doing the problem properly, you will probably write down the ODE.}
    \item Take the operator $x\frac{d}{dx} \colon C^\omega(\C) \to C^\omega(\C)$. Find the eigenfunctions to this operator using the fact that this corresponds to a separable ODE.
\end{enumerate}
\end{problem}

\begin{problem} Consider the Legendre polynomials
    \[
        B_L = \left\{f_0 = \sqrt{\frac{1}{2}}, ~ f_1 = \sqrt{\frac{3}{2}}x, ~ f_2 = \sqrt{\frac{5}{8}} (1-3x^2),~ f_3=\sqrt{\frac{63}{8}}\left(x-\frac{5x^3}{3}\right) \right\}
    \]
which form a basis for $P_3(\C)$.
\begin{enumerate}[(a)]
    \item For polynomials $f,g\in P_3(\C)$, define an inner product
    \[
\langle g,h\rangle \coloneqq \int_{-1}^1 gh^* dx.
\]
    Show (or find in the text or previous homeworks) evidence that the basis $B_L$ is orthonormal with respect to this inner product.
    \item Consider the operator
    \[
    \mathcal{L} \coloneqq (1-x^2)\frac{d^2}{dx^2} - 2x \frac{d}{dx} \colon P_3(\C) \to P_3(\C).
    \]
    Show that $\mathcal{L}$ is linear.
    \item Show that each Legendre polynomial $f_i$ is an eigenvector (or eigenfunction) of $\mathcal{L}$. What are the corresponding eigenvalues? How do these eigenvalues correspond to the $m$ that appears in Legendre's equation (see the section in our text).
\end{enumerate}
\end{problem}




\end{document}
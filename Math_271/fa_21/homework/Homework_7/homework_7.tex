%%%%%%%%%%%%%%%%%%%%%%%%%%%%%%%%%%%%%%%%%%%%%%%%%%%%%%%%%%%%%%%%%%%%%%%%%%%%%%%%%%%%
% Document data
%%%%%%%%%%%%%%%%%%%%%%%%%%%%%%%%%%%%%%%%%%%%%%%%%%%%%%%%%%%%%%%%%%%%%%%%%%%%%%%%%%%%
\documentclass[12pt]{article} %report allows for chapters
%%%%%%%%%%%%%%%%%%%%%%%%%%%%%%%%%%%%%%%%%%%%%%%%%%%%%%%%%%%%%%%%%%%%%%%%%%%%%%%%%%%%
\usepackage{preamble}

\begin{document}

\begin{center}
   \textsc{\large MATH 271, Homework 7}\\
   \textsc{Due November 3$^\textrm{rd}$}
\end{center}
\vspace{.5cm}

\begin{problem}
Consider the following vectors in the real plane $\R^2$. We let
\[
\vecu = 1\xhat + 2\yhat \qquad \textrm{and} \qquad \vecv = -3\xhat+ 3\yhat.
\]
\begin{enumerate}[(a)]
    \item What is the dimension of the vector space $\R^2$? Explain.
    \item Draw both $\vecu$ and $\vecv$ in the plane and label the origin.
    \item Draw the vector $\vecw = \vecu+\vecv$ in the plane.
    \item Draw the subspace spanned by $\vecu$.
\end{enumerate}
\end{problem}
\vspace*{0.5cm}

\begin{problem}
Let a mass $m_1$ weighing $1kg.$ be placed at $\vecr_1=2\xhat -3 \yhat -\zhat$ and a mass $m_2$ of $2kg.$ be placed at $\vecr_2 = 4\yhat -2\zhat$.  Where must a mass $m_3$ of $3kg.$ be placed so that the center of mass is at the origin $\zerovec$?
\end{problem}
\vspace*{0.5cm}

\begin{problem}
Which of the following are linear transformations? For those that are not, which properties of linearity (the properties (i) and (ii) in our notes) fail? Show your work.
\begin{enumerate}[(a)]
    \item $T_a \colon \R \to \R$ given by $T_a(x)=\frac{1}{x}$.
    \item $T_b \colon \R^3 \to \R^2$ given by
    \[
    T_b \begin{pmatrix} x\\ y\\ z \end{pmatrix}
    = \begin{pmatrix} x\\ y \end{pmatrix}.
    \]
    \item $T_c \colon \R \to \R^3$ given by
    \[
    T_c(t)=\begin{pmatrix} t\\ t^2\\ t^3 \end{pmatrix}.
    \]
    \item $T_d \colon \R^2 \to \R^3$ given by
    \[
    T_d \begin{pmatrix} x\\ y \end{pmatrix}
    = \begin{pmatrix} x+y\\ x+y\\ x+y \end{pmatrix}.
    \]
\end{enumerate}
\end{problem}
\vspace*{0.5cm}

\begin{problem}~
\begin{enumerate}[(a)]
    \item We can reflect an arbitrary vector in the plane by defining a function that reflects the basis vectors and extending the function with linearity. Let $R\colon \R^2 \to \R^2$ be a function be defined by
    \[
    R(\xhat) = -\xhat \qquad \textrm{and} \qquad R(\yhat)=\yhat.
    \]
    Let $\vecv = \alpha_1 \xhat + \alpha_2 \yhat$ and let
    \[
    R(\vecv) = \alpha_1 R(\xhat) + \alpha_2 R(\yhat).
    \]
    Show that $R$ reflects the vector $\vecu = 1\xhat + 2\yhat$ about the $y$-axis and draw a picture.
    \item We can rotate a vector in the plane by first rotating the basis vectors $\xhat$ and $\yhat$. Define a linear function $J\colon \R^2 \to \R^2$ defined by
    \[
    J(\xhat)=\yhat \qquad \textrm{and} \qquad J(\yhat)=-\xhat.
    \]
    \noindent Show that $J$ rotates $\vecu$ by $\pi/2$ in the counterclockwise direction and draw a picture.
\end{enumerate}
\end{problem}
\vspace*{0.5cm}

\begin{problem}
Let $S$ be the set of general solutions to the following second order homogeneous linear differential equation
\[
x''+f(t)x'+g(t)x=0.
\]
Show that this set $S$ is a vector space over the field of complex numbers.
\end{problem}
\vspace*{0.5cm}

\begin{problem}
Let $P_3(\mathbb{C})$ be the vector space of polynomials of degree at most 3 with coefficients in $\C$ with variable $x$. For example,
\[
f(x)= x^2+1 \in P_3(\mathbb{C}).
\]
\begin{enumerate}[(a)]
    \item Write down a basis for $P_3(\mathbb{C})$.
    \item What is the dimension of the vector space $P_3(\mathbb{C})$.
    \item Let $\frac{d}{dx} \colon P_3(\mathbb{C}) \to P_3(\mathbb{C})$. Argue that $\frac{d}{dx}$ is a linear transformation.
    \item What is the kernel (nullspace) of $\frac{d}{dx}$? What is the image (range) of $\frac{d}{dx}$?
\end{enumerate}
\end{problem}


\end{document}
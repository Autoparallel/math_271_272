%%%%%%%%%%%%%%%%%%%%%%%%%%%%%%%%%%%%%%%%%%%%%%%%%%%%%%%%%%%%%%%%%%%%%%%%%%%%%%%%%%%%
% Document data
%%%%%%%%%%%%%%%%%%%%%%%%%%%%%%%%%%%%%%%%%%%%%%%%%%%%%%%%%%%%%%%%%%%%%%%%%%%%%%%%%%%%
\documentclass[12pt]{article} %report allows for chapters
%%%%%%%%%%%%%%%%%%%%%%%%%%%%%%%%%%%%%%%%%%%%%%%%%%%%%%%%%%%%%%%%%%%%%%%%%%%%%%%%%%%%
\usepackage{preamble}

\begin{document}

\begin{center}
   \textsc{\large MATH 271, Homework 8}\\
   \textsc{Due November 8$^\textrm{th}$}
\end{center}
\vspace{.5cm}

\begin{problem}
Let a mass $m_1$ weighing $1kg.$ be placed at $\vecr_1=2\xhat -3 \yhat -\zhat$ and a mass $m_2$ of $2kg.$ be placed at $\vecr_2 = 4\yhat -2\zhat$.  Where must a mass $m_3$ of $3kg.$ be placed so that the center of mass is at the origin $\zerovec$?
\end{problem}

\begin{problem}
Which of the following are linear transformations? For those that are not, which properties of \emph{linearity} (the properties (i) and (ii) in our notes) fail? Show your work. 
\begin{enumerate}[(a)]
    \item $T_a \colon \R \to \R$ given by $T_a(x)=\frac{1}{x}$.
    \item $T_b \colon \R^3 \to \R^2$ given by
    \[
    T_b \begin{pmatrix} x\\ y\\ z \end{pmatrix}
    = \begin{pmatrix} x\\ y \end{pmatrix}.
    \]
    \item $T_c \colon \R \to \R^3$ given by
    \[
    T_c(t)=\begin{pmatrix} t\\ t^2\\ t^3 \end{pmatrix}.
    \]
    \item $T_d \colon \R^2 \to \R^3$ given by
    \[
    T_d \begin{pmatrix} x\\ y \end{pmatrix}
    = \begin{pmatrix} x+y\\ x+y\\ x+y \end{pmatrix}.
    \]
\end{enumerate}
\end{problem}

\begin{problem}
Write down the matrix for the following linear transformation $T\colon \R^3 \to \R^3$:
\[
T\begin{pmatrix} x\\ y\\ z \end{pmatrix}
= \begin{pmatrix} x+y+z\\ 2x\\ 3y + z \end{pmatrix}.
\]
\end{problem}

\begin{problem}
A linear transformation $T\colon \R^3 \to \R^3$ is given by the matrix
\[
[T]= \begin{pmatrix}
1& 2& 0\\
2& 1& 2\\
0& 2& 1
\end{pmatrix}.
\]
\begin{enumerate}[(a)]
    \item Compute how $T$ transforms the standard basis elements for $\R^3$. That is, find
    \[
    T(\xhat), \qquad
    T(\yhat), \qquad 
    T(\zhat).
    \]
    This gives a nice interpretation of matrix vector multiplication as linear combinations of the column vectors that make up a matrix.
    \item If we apply this linear transformation to the unit cube (that is, all points who have $(x,y,z)$ coordinates with $0\leq x \leq 1$, $0\leq y \leq 1$, and $0\leq z \leq 1$), what will the volume of the transformed cube be? (\emph{Hint: the determinant of this matrix $[T]$ provides us this information.})
\end{enumerate}
\end{problem}


\begin{problem}
Solve the following equation.
\[
\begin{pmatrix} 1 & 1 & 1 \\ 1 & 2 & 1 \\ 1 & 2 & 2 \end{pmatrix} \begin{pmatrix} x \\ y \\ z \end{pmatrix} = \begin{pmatrix} 6 \\ 8 \\ 11 \end{pmatrix}.
\]
\end{problem}



\end{document}
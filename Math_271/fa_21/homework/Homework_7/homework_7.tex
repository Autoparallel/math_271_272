%%%%%%%%%%%%%%%%%%%%%%%%%%%%%%%%%%%%%%%%%%%%%%%%%%%%%%%%%%%%%%%%%%%%%%%%%%%%%%%%%%%%
% Document data
%%%%%%%%%%%%%%%%%%%%%%%%%%%%%%%%%%%%%%%%%%%%%%%%%%%%%%%%%%%%%%%%%%%%%%%%%%%%%%%%%%%%
\documentclass[12pt]{article} %report allows for chapters
%%%%%%%%%%%%%%%%%%%%%%%%%%%%%%%%%%%%%%%%%%%%%%%%%%%%%%%%%%%%%%%%%%%%%%%%%%%%%%%%%%%%
\usepackage{preamble}

\begin{document}

\begin{center}
   \textsc{\large MATH 271, Homework 7}\\
   \textsc{Due November 2$^\textrm{st}$}
\end{center}
\vspace{.5cm}

\begin{problem}
Let $S$ be the set of general solutions $x(t)$ to the following homogeneous linear differential equation 
\[
x''+f(t)x'+g(t)x=0.
\]
Show that this set $S$ is a vector space over the complex numbers by doing the following. Let $x(t),y(t),z(t) \in S$ be solutions to the above equation and let $\alpha, \beta \in \C$ be complex scalars.
\begin{enumerate}[(a)]
    \item Write down the eight requirements for $S$ to be a vector space.  
    \item Identify the objects that play the role of $\zerovec \in S$ and $1\in \C$. \emph{Hint: Don't overthink this, it is just a formality.} 
    \item Show that $\alpha x(t) + \beta y(t) \in S$. That is, show that a superposition of solutions is also a solution. \emph{Hint: We have shown this before.}
\end{enumerate}
\end{problem}

\begin{problem}
Consider the following vectors in the real plane $\R^2$. We let
\[
\vecu = 1\xhat + 2\yhat \qquad \textrm{and} \qquad \vecv = -3\xhat+ 3\yhat.
\]
\begin{enumerate}[(a)]
    \item Draw both $\vecu$ and $\vecv$ in the plane and label the origin.
    \item Draw the vector $\vecw = \vecu+\vecv$ in the plane.
    \item Find the area of the parallelogram generated by $\vecu$ and $\vecv$.
\end{enumerate}
\end{problem}

\begin{problem}
Consider the following vectors in space $\R^3$
\[
\vecu = 1\xhat + 2\yhat + 3\zhat \qquad \textrm{and} \qquad \vecv = -2\xhat +1\yhat -2\zhat.
\]
\begin{enumerate}[(a)]
    \item Compute the dot product $\vecu\cdot \vecv$. 
    \item Compute the cross product $\vecu \times \vecv$.
    \item Compute the lengths $\|\vecu\|$ and $\|\vecv\|$ using the dot product.
    \item Compute the angle between vectors $\vecu$ and $\vecv$. 
    \item Compute the projection of $\vecu$ in the direction of $\vecv$. 
\end{enumerate}
\end{problem}

\begin{problem}~
\begin{enumerate}[(a)]
    \item We can reflect a vector in the plane by first reflecting basis vectors. Let $R\colon \R^2 \to \R^2$ be a function be defined by 
    \[
    R(\xhat) = -\xhat \qquad \textrm{and} \qquad R(\yhat)=\yhat.
    \]
    Let $\vecv = \alpha_1 \xhat + \alpha_2 \yhat$ and let
    \[
    R(\vecv) = \alpha_1 R(\xhat) + \alpha_2 R(\yhat).
    \]
    When this is the case, we call the function $R$ \underline{linear}.\\
    \noindent Show that $R$ reflects the vector $\vecu = 1\xhat + 2\yhat$ about the $y$-axis and draw a picture.
    \item We can rotate a vector in the plane by first rotating the basis vectors $\xhat$ and $\yhat$. Define a \underline{linear} function $T\colon \R^2 \to \R^2$ defined by
    \[
    T(\xhat)=\yhat \qquad \textrm{and} \qquad T(\yhat)=-\xhat.
    \]
    \noindent Show that $T$ rotates $\vecu$ by $\pi/2$ in the counterclockwise direction and draw a picture.
\end{enumerate}
\end{problem}

\end{document}
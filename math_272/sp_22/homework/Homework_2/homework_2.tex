%%%%%%%%%%%%%%%%%%%%%%%%%%%%%%%%%%%%%%%%%%%%%%%%%%%%%%%%%%%%%%%%%%%%%%%%%%%%%%%%%%%%
% Document data
%%%%%%%%%%%%%%%%%%%%%%%%%%%%%%%%%%%%%%%%%%%%%%%%%%%%%%%%%%%%%%%%%%%%%%%%%%%%%%%%%%%%
\documentclass[12pt]{article} %report allows for chapters
%%%%%%%%%%%%%%%%%%%%%%%%%%%%%%%%%%%%%%%%%%%%%%%%%%%%%%%%%%%%%%%%%%%%%%%%%%%%%%%%%%%%
\usepackage{preamble}

\newcommand{\curvegamma}{\boldsymbol{\vec{\gamma}}}
\newcommand{\tangentgamma}{\boldsymbol{\dot{\vec{\gamma}}}}
\newcommand{\normalgamma}{\boldsymbol{\ddot{\vec{\gamma}}}}
\newcommand{\vecfieldB}{\boldsymbol{\vec{B}}}
\newcommand{\vecfieldJ}{\boldsymbol{\vec{J}}}
\newcommand{\vecfieldF}{\boldsymbol{\vec{F}}}
%\newcommand{\vecfieldE}{\boldsymbol{\vec{E}}}

\begin{document}

\begin{center}
   \textsc{\large MATH 272, Homework 2}\\
   \textsc{Due February 8$^\textrm{th}$}
\end{center}
\vspace{.5cm}



\begin{problem}
A rough model of a molecular crystal can be described in the following way. Take the scalar function
\[
u(x,y)=\cos^2(x)+\cos^2(y).
\]
This function $u(x,y)$ describes the \emph{potential energy} for electrons in the crystal. Electrons are attracted to the areas with the smallest potential energy and move away from areas of high potential energy. 
\begin{enumerate}[(a)]
    \item Plot this function and include a printout.  Notice what this looks like.  You can imagine that each of the low points (well) is where a nucleus is located in the crystal.
    \item Plot the level curves where $u(x,y)=0$, $u(x,y)=\frac{1}{4}$, $u(x,y)=\frac{1}{2}$, and $u(x,y)=1$ for the range of values $-\frac{3\pi}{2}\leq x \leq \frac{3\pi}{2}$ and $-\frac{3\pi}{2}\leq y \leq \frac{3\pi}{2}$. Picking the constant for the level curve tells you the \emph{kinetic energy} of the electron you are looking at.  It turns out that electrons (roughly) will orbit along these level curves.  Notice, some level curves bleed into the different troughs of neighboring molecules which means that electrons of sufficient energy happily flow through the crystal. For what energy values do the electrons move throughout the whole crystal?
    \item Find the gradient of this function $\grad u(x,y)$.
    \item At what point(s) is the gradient zero? \emph{Hint: Use your graph of the level curves to help.}
    \end{enumerate}
\end{problem}



\begin{problem}
Consider the function
\[
f(x,y)=\sin\left(\frac{2\pi x}{5}\right)\sin\left(\frac{2\pi y}{5}\right).
\]
comes up when you want to find out how a square shaped drum head will vibrate when hit. 
\begin{enumerate}[(a)]
    \item Plot this function on the region $\Omega$ given by $0\leq x \leq 5$ and $0\leq y \leq 5$.  
    \item What is the value the function $f(x,y)$ on the boundary of the given region $\Omega$ (i.e, when $x=0$, $x=5$, $y=0$, and $y=5$)?
    \item Show that $f(x,y)$ is an eigenfunction of the Laplacian $\Delta$. That is, $\Delta f = \lambda f$ for some eigenvalue $\lambda$. What is the eigenvalue?
\end{enumerate}
\end{problem}

\begin{problem}
Consider the following vector field
\[
\vecfieldB = -\frac{y}{2}\xhat + \frac{x}{2}\yhat.
\]
Here, $\vecfieldB$ denotes the magnetic field. It may be helpful to plot the fields in this problem.
\begin{enumerate}[(a)]
    \item Show that $\vecfieldB$ has no divergence. (This is one of Gauss's laws.)
    \item Show that $\grad \times \vecfieldB = \vecfieldJ$ (Amp\'ere's law) where 
    \[
    \vecfieldJ = \zhat.
    \]
    This vector field $\vecfieldJ$ represents the electric current (moving charges) in space. One could argue that the current creates the magnetic field via Amp\'ere's law.
    \item Magnetic fields induce a force $\vecfieldF$ on charged particles by the Lorentz force
    \[
    \vecfieldF = \tangentgamma \times \vecfieldB = \normalgamma
    \]
    Where $\tangentgamma$ is the velocity of the particle (where we have chosen a mass $m=1$ and charge $q=1$).  Let us do the following.
    \begin{itemize}
        \item Assume that $\tangentgamma = \xhat$, what is the force on the particle?
        \item Repeat the previous step for $\tangentgamma=\yhat$ and $\tangentgamma=\zhat$. 
        \item Compare and contrast the forces you found.
    \end{itemize}
    \item Can you argue why applying a magnetic field to a molecule may cause it to heat up? Can you compare this idea with your home microwave?
\end{enumerate}
\end{problem}

\begin{problem}
    Set up but do not compute the integrals of the given fields over the given curves.
    \begin{enumerate}[(a)]
        \item $f(x,y) = xe^{x+y}+\cos(xy)$, $\curvegamma(t) = \begin{pmatrix} \cos(t) \\ \sin(t) \end{pmatrix}$, $t_0 = 0$, $t_1=2\pi$.
        \item $g(x,y,z) = \frac{\ln(z^2)}{e^{xy}}$, $\curvegamma(t) = \begin{pmatrix} t \\ t^2 \\ t^3 \end{pmatrix}$ , $t_0 = 1$, $t_1=-1$.
        \item $\vecfieldU(x,y) = \begin{pmatrix} -y \\ x \end{pmatrix}$, $\curvegamma(t) = \begin{pmatrix} e^t \\ e^t \end{pmatrix}$, $t_0 = 5$, $t_1=10$.
        \item $\vecfieldV(x,y,z) = \begin{pmatrix} 2x \\ y \\ x \end{pmatrix}$, $\curvegamma(t) = \begin{pmatrix} \cos(t) \\ t \\ \sqrt{t} \end{pmatrix}$, $t_0 = 0$, $t_1=1$.
    \end{enumerate}
\end{problem}

\begin{problem}
Let $f(x,y,z)=x \cos(y) + yz$ be a scalar field and let $\curvegamma = \begin{pmatrix} 1 \\ t \\ \sin(t) \end{pmatrix}$ be a curve from time $t_0 = 0$ to $t_1 = 2 \pi$. Compute
\[
    \int_{\curvegamma} f(\curvegamma)d\curvegamma.
\]
\end{problem}


\begin{problem}
    Consider the following vector field
    \[
    \vecfieldE = \frac{x}{\left(x^2+y^2+z^2\right)^{3/2}} \xhat + \frac{y}{\left(x^2+y^2+z^2\right)^{3/2}} \yhat + \frac{z}{\left(x^2+y^2+z^2\right)^{3/2}} \zhat,
    \]
    which you can think of as the electric field of a positive point charge.  We argued that this field $\vecfieldE$ is conservative in a previous homework problem. Specifically, $\vecfieldE = \grad \phi$, for the scalar field 
\[
\phi(x,y,z) = \frac{1}{\sqrt{x^2+y^2+z^2}}
\] 
This follows from Faraday's law for static charges.
    \begin{enumerate}[(a)]
        \item Compute the integral
        \[
        T=\int_{\curvegamma} \vecfieldE \cdot d\curvegamma \qquad \textrm{where} \qquad \curvegamma(t) = \begin{pmatrix} t \\ t \\ t \end{pmatrix},
        \]
        and $t\in [t_0,t_1]$ with $t_0$ and $t_1$ both greater than 0.  Note that this integral $T$ describes the gain in kinetic energy of a charged particle that moved along the path $\curvegamma$. 
        \item Equivalently, since $\vecfieldE$ is conservative, we have
        \[
        T=\int_{\curvegamma} \vecfieldE \cdot d\curvegamma = \phi(\curvegamma(t_1))-\phi(\curvegamma(t_0)).
        \]
        Show that this is true for the given vector field and potential. This shows that the choice of path does not matter; only the endpoints $\curvegamma(t_0)$ and $\curvegamma(t_1)$ matter.
        \item Argue why the integral around any closed curve must be zero.
        \item Argue why it must be that $\vecfieldE$ has no curl.
    \end{enumerate}
\end{problem}


\end{document}
%%%%%%%%%%%%%%%%%%%%%%%%%%%%%%%%%%%%%%%%%%%%%%%%%%%%%%%%%%%%%%%%%%%%%%%%%%%%%%%%%%%%
% Document data
%%%%%%%%%%%%%%%%%%%%%%%%%%%%%%%%%%%%%%%%%%%%%%%%%%%%%%%%%%%%%%%%%%%%%%%%%%%%%%%%%%%%
\documentclass[12pt]{article} %report allows for chapters
%%%%%%%%%%%%%%%%%%%%%%%%%%%%%%%%%%%%%%%%%%%%%%%%%%%%%%%%%%%%%%%%%%%%%%%%%%%%%%%%%%%%
\usepackage{preamble}
\usepackage{caption}
\usepackage{subcaption}

\begin{document}

\begin{center}
   \textsc{\large MATH 271, Homework 3, \emph{Solutions}}\\
\end{center}
\vspace{.5cm}

\begin{problem}
Write down the equations for each of the reactants and products for the following reactions.
\begin{enumerate}[(a)]
    \item $A + 3B + C \xrightarrow{k} 2D+2E$.
    \item $A \xrightarrow{k_1} B + C \xrightarrow{k_2} D$.
\end{enumerate}
\end{problem}
\begin{solution}
\begin{enumerate}[(a)]
    \item We have the equations
    \begin{align*}
        [A]' &= -k[A][B]^3[C]\\
        [B]' &= -3k[A][B]^3[C]\\
        [C]' &= -k[A][B]^3[C]\\
        [D]' &= 2k[A][B]^3[C]\\
        [E]' &= 2k[A][B]^3[C].
    \end{align*}
    \item The equations are
    \begin{align*}
        [A]' &= -k_1[A]\\
        [B]' &= k_1[A] - k_2 [B][C]\\
        [C]' &= k_1[A] - k_2 [B][C]\\
        [D]' &= k_2[B][C].
    \end{align*}
\end{enumerate}
\end{solution}

\newpage

\begin{problem}
Consider the following reaction
\[
A \xrightarrow{k_1} B \xrightarrow{k_2} C.
\]
For the following parts, use the link: \url{https://www.desmos.com/calculator/srrpeadlou}.
\begin{enumerate}[(a)]
    \item Compare and contrast the reactions that take place given the three different scenarios for initial conditions. Explain why what the graph displays makes sense and include your graphs.
    \begin{itemize}
        \item $[A]_0 = 1$, $[B]_0=0$, and $[C]_0 =0$.
        \item $[A]_0 = 0$, $[B]_0=1$, and $[C]_0 =0$.
        \item $[A]_0 = 0$, $[B]_0=0$, and $[C]_0 =1$.
    \end{itemize}
    \item For the initial conditions $[A]_0 = 1$, $[B]_0=0$, and $[C]_0 =0$, explain what happens when you let
    \begin{itemize}
        \item $k_1=0$ and $k_2=1$,
        \item $k_1=1$ and $k_2=0$.
    \end{itemize}
    Include plots for these cases as well.
    \item Consider the initial conditions $[A]_0 = 1$, $[B]_0=0$, and $[C]_0 =0$ and rate constants $k_1=1$ and $k_2=2$. Then, choose initial conditions of your own and compare your plots with the other initial conditions. Why do yours behave the way they do? Include your plots.
\end{enumerate}
\end{problem}
\begin{solution}
\begin{enumerate}[(a)]
    \item Here are the graphs for the given initial conditions.
    \begin{figure}[H]
        \centering
        \begin{subfigure}[b]{0.3\textwidth}
            \centering
            \includegraphics[width=\textwidth]{a01b00c00.png}
            \caption{Curves for $[A]_0=1$, $[B]_0=0$, and $[C]_0=0$.}
        \end{subfigure}
        \hfill
        \begin{subfigure}[b]{0.3\textwidth}
            \centering
            \includegraphics[width=\textwidth]{a00b01c00.png}
            \caption{Curves for $[A]_0=0$, $[B]_0=1$, and $[C]_0=0$.}
        \end{subfigure}
        \hfill
        \begin{subfigure}[b]{0.3\textwidth}
            \centering
            \includegraphics[width=\textwidth]{a00b00c01.png}
            \caption{Curves for $[A]_0=0$, $[B]_0=0$, and $[C]_0=1$.}
        \end{subfigure}
    \end{figure}
    
    Let's now think about these graphs.
    \begin{itemize}
            \item The first sees exponential decay of species $A$ and initially a quick growth of species $B$.  But, as species $B$ reacts and creates species $C$, we start to see a plateau and then decline in the concentration of species $B$.  Ultimately, the concentration of species $C$ seems to grow consistently.  In the beginning, the concentration of species $C$ increases more slowly since there is less of $B$ to react, and it grows more slowly towards the end of the reaction for a similar reason.
            \item Now, species $A$ has no role in the reaction.  We simply see exponential decay of species $B$ as it produces $C$.
            \item There is no reaction taking place since we have removed $A$ and $B$ from the system entirely.  All we have left is the stable species $C$.
    \end{itemize}

    \item Here are the graphs for the different $k$ values.
    \begin{figure}[H]
        \centering
        \begin{subfigure}[b]{0.3\textwidth}
            \centering
            \includegraphics[width=\textwidth]{k10k21.png}
            \caption{Curves for $k_1=0$ and $k_2=1$ with initial concentrations $[A]_0=1$, $[B]_0=0$, and $[C]_0=0$. }
        \end{subfigure}
        \hspace*{2cm}
        \begin{subfigure}[b]{0.3\textwidth}
            \centering
            \includegraphics[width=\textwidth]{k11k20.png}
            \caption{Curves for $k_1=1$ and $k_2=0$ with initial concentrations $[A]_0=1$, $[B]_0=0$, and $[C]_0=0$. }
        \end{subfigure}
    \end{figure}
    
    Let's now think about these graphs.
    \begin{itemize}
            \item Here, if $k_1=0$, then the first reaction never takes place.  So, none of species $A$ can react to form species $B$.  Since there is no initial concentration for $B$ or $C$, there are never any of these products produced.  We simply see $[A]$ remain constant. This is similar to what we see in the third case in the previous part of this problem.
            \item By taking $k_2=0$ we have essentially removed the second reaction.  Now, $A$ converts to $B$ and $B$ never converts to $C$.  Thus, we see exponential decay of $A$ which produces $B$.  This is similar to what we see in the second case in the previous part of this problem.
    \end{itemize}

    \item Plots for (c) below.
    \begin{figure}[H]
        \centering
        \begin{subfigure}[b]{0.3\textwidth}
            \centering
            \includegraphics[width=\textwidth]{k11k22.png}
            \caption{Curves for $k_1=1$ and $k_2=2$ with initial concentrations $[A]_0=1$, $[B]_0=0$, and $[C]_0=0$. }
        \end{subfigure}
        \hspace*{2cm}
        \begin{subfigure}[b]{0.3\textwidth}
            \centering
            \includegraphics[width=\textwidth]{k11k22_other.png}
            \caption{Curves for $k_1=1$ and $k_2=0$ with initial concentrations $[A]_0=0.5$, $[B]_0=0.3$, and $[C]_0=0.1$. }
        \end{subfigure}
    \end{figure}

    I chose these other initial conditions because we never see an increase in the amount of species $B$.  To me, this is a bit interesting.  The rate of reaction $k_2$ being larger means that we do not always see a build up of species $B$.  
\end{enumerate}
\end{solution}

\newpage

\begin{problem}
Consider the second order chemical reaction given by
\[
A+B \xrightarrow{k} \textrm{Products}.
\]
\begin{enumerate}[(a)]
    \item Write a \emph{system} of differential equations to describe the concentration of the reactants $A$ and $B$ (this means write one for each).
    \item The concentrations of $A$ and $B$ can be related to each other in the following way: Let $A=A_0-x$ and $B=B_0-x$. Here, we think of $x$ as the amount of each chemical that has reacted, and note that it depends on time $t$. Use this change of variables to rewrite the differential equation for chemical $A$ in terms of $x$ and $t$.
    \item Solve the differential equation in (b) with the initial condition $x(0)=0$.You will need to use \emph{partial fraction decomposition} to evaluate the integral.
\end{enumerate}
\end{problem}
\begin{solution}~
\begin{enumerate}[(a)]
    \item The system of equations we will get is
    \begin{align*}
        \frac{d[A]}{dt}&=-k[A][B]\\
        \frac{d[B]}{dt}&=-k[A][B].
    \end{align*}
    \item Now, let $[A]=[A]_0-x$ and $[B]=[A]_0-x$ and, since $[A]_0$ and $[B]_0$ are constant, we get the equation for $[A]$,
    \begin{align*}
        -\frac{dx}{dt}&=k([A]_0-x)([B]_0-x).
    \end{align*}
    It turns out $[B]$ has the same equation (which you should double check yourself).
    \item This is a separable equation, so we can find the solution by
    \begin{align*}
        -\frac{dx}{dt}&=k([A]_0-x)([B]_0-x)\\
        \int \frac{dx}{([A]_0-x)([B]_0-x)}&=-k\int dt.
    \end{align*}
    Here, we can use the partial fraction decomposition to get
    \[
    \frac{1}{[A]_0-[B]_0}\ln\left( \frac{x-[A]_0}{x-[B]_0}\right)=-kt+C.
    \]
    Then we can find
    \[
    \frac{x-[A]_0}{x-[B]_0}=e^{-kt+C}
    \]
    With $x(0)=0$ we have
    \[
    \frac{-[A]_0}{-[B]_0}=e^{-kt}e^C
    \]
    and so $e^C=\frac{[A]_0}{[B]_0}$. We can rewrite this in terms of $[A]$ and $[B]$ as
    \[
    \frac{[A]}{[B]}=\frac{[A]_0}{[B]_0}e^{-kt}.
    \]
    This is as simplified as I would take the expression. What we can see here is that the ratio of $[A]$ to $[B]$ will change exponentially over time.
\end{enumerate}
\end{solution}

\newpage
\begin{problem}
If $x_1(t)$ and $x_2(t)$ are solutions to the differential equation
\[
x'' + bx' +cx = 0
\]
is $x=x_1+x_2+k$ for a constant $k$ always a solution? Is the function $y=tx_1$ a solution? 
\end{problem}
\begin{solution}
$x$ and $y$ are \emph{not} solutions.  Let's see why.  We note that $x_1$ and $x_2$ are solutions and thus
\[
x_i''+bx_i'+cx_i=0\qquad \textrm{for $i=1,2$}.
\]
Now, we check if $x$ is a solution by plugging into the left hand side
\begin{align*}
    x''+bx'+cx&= (x_1+x_2+k)''+b(x_1+x_2+k)'+c(x_1+x_2+k)\\
    &= \underbrace{x_1''+bx_1'+cx_1}_{=0}+ \underbrace{x_2''+bx_2'+cx_2}_{=0}+ck\\
    &= ck \neq 0.
\end{align*}
So this $x$ is not a solution. 

Similarly, we take $y=tx_1$ and plug it into the left hand side and find
\begin{align*}
    y''+by'+cy&=(tx_1)''+b(tx_1)'+c(tx_1)\\
    &=tx_1''+2x_1'+ b(tx_1'+x_1)+c(tx_1)\\
    &= t\underbrace{(x_1''+bx_1'+cx_1)}_{=0}+2x_1'+bx_1\\
    &=2x_1'+bx_1,
\end{align*}
which is not in general a solution unless $x_1=0$.  
\end{solution}

\newpage
\begin{problem}
Consider the following initial value problem:
\begin{align*}
    x''+4x'+3x&=0
\end{align*}
with initial data $x(0)=1$, $x'(0)=0$.  
\begin{enumerate}[(a)]
    \item Find the solution.
    \item Sketch a plot of the solution.
    \item Explain in words what is happening to the solution as time goes on. What happens as $t\to \infty$?
\end{enumerate}
\end{problem}
\begin{solution}~
\begin{enumerate}[(a)]
    \item We can solve this homogeneous second order linear equation with constant coefficients by finding roots to its characteristic polynomial. In this case, that amounts to
    \begin{align*}
        \lambda^2+4\lambda+3&=0\\
        \iff (\lambda +3)(\lambda+1)&=0,
    \end{align*}
    so the roots are $\lambda_1=-1$ and $\lambda_2=-3$.  Thus our general solution is
    \[
    x(t)=C_1 e^{\lambda_1 t}+C_2e^{\lambda_2 t}=C_1 e^{-t}+C_2e^{-3t}.
    \]
    Then we use the initial conditions to find a particular solution. Namely,
    \begin{align*}
        1=x(0)&=C_1e^{-0}+C_2e^{-3\cdot 0}=C_1+C_2\\
        0=x'(0)&=-C_1e^{-0}-3C_2e^{-3\cdot 0}=-C_1-3C_2.
    \end{align*}
    Using the second equation we get $C_1=-3C_2$. We can plug this into the first equation to get
    \[
    1=-3C_2+C_2=-2C_2
    \]
    meaning that $C_2=-\frac{1}{2}$. Thus $C_2=\frac{3}{2}$. Hence, our particular solution for this IVP is
    \[
    x(t)=\frac{3}{2} e^{-t}-\frac{1}{2}e^{-3t}.
    \]
    \item Here is a plot of the particular solution from time $t=0$ to time $t=10$.
    \begin{figure}[H]
        \centering
        \includegraphics[width=.7\textwidth]{desmos-graph(19).png}
    \end{figure}
    \item The solution decays exponentially over time.  As $t\to \infty$ our solution approaches zero.
\end{enumerate}
\end{solution}

\newpage
\begin{problem}
Write down a homogeneous second-order linear differential equation where the system displays a decaying oscillation.
\end{problem}
\begin{solution}
Since our solution should oscillate and decay, we need some form of a ``spring" and some form of damping.  These terms show up respectively as $b$ and $c$ in the equation
\[
x''+bx'+cx=0.
\]
Now, also note that (aside from one special case of two of the same real roots), our general solution has the form
\[
x(t)=C_1 e^{\lambda_1 t}+C_2e^{\lambda_2 t}
\]
where $\lambda_1$ and $\lambda_2$ are roots to the characteristic polynomial
\[
\lambda^2+b\lambda + c =0.
\]
Now, the roots for the characteristic polynomial are
\[
\lambda = \frac{-b \pm \sqrt{b^2-4c}}{2}.
\]
\begin{itemize}
    \item To have oscillation, our roots must have an imaginary part and thus 
    \[
    b^2-4c<0.
    \]
    In other words, $b^2<4c.$
    \item To have a decaying solution, the real part of the roots must be negative. The real part of the roots will be $\frac{-b}{2}$ and thus we need
    \[
    \frac{-b}{2}<0.
    \]
\end{itemize}
Now, I'll choose $b=1$ and $c=1$ which satisfy both of these requirements. We then have
\[
x''+x'+x=0
\]
as our equation.

Note, we can also find the solution as the roots are then
\[
\lambda = \frac{-1\pm \sqrt{1-4}}{2}=\frac{-1}{2}\pm \frac{\sqrt{3}}{2}.
\]
Plugging this into the form for the general solution and we get
\[
x(t)=e^{-\frac{1}{2}t}\left(C_1 \sin\left(\frac{\sqrt{3}}{2}\right) + \cos\left(\frac{\sqrt{3}}{2}\right)\right)
\]
\end{solution}

\newpage
\begin{problem}
Consider the following differential equation:
\[
x''+2x'+x=\sin(t)
\]
\begin{enumerate}[(a)]
    \item Find the homogeneous solution $x_H(t)$.
    \item Find the particular integral $x_P(t)$.
    \item Find the specific solution corresponding to the initial data $x(0)=0$, $x'(0)=0$.
    \item Plot the curve $(x(t),x'(t))$ in the plane (we often call this \emph{phase space}). Use this link here: \url{https://www.desmos.com/calculator/ouqwcxj2xz}. Use the time range $t\in [0,10\pi]$.
    \item Describe what happens with this system over time. Does it seem to approach some kind of stable solution? Note that this stable solution could be periodic.
\end{enumerate}
\end{problem}
\begin{solution}~
\begin{enumerate}[(a)]
    \item The roots to characteristic polynomial satisfy
    \[
    \lambda^2 + 2\lambda + 1 = 0
    \]
    which can be found by factoring
    \[
    (\lambda+1)^2=0,
    \]
    which gives us that $\lambda=-1$ is the only root.  Thus, this is the special case where our general solution looks slightly different.  We'll have
    \[
    \boxed{x_H(t)=C_1e^{-t}+C_2te^{-t}.}
    \]
    \item The right hand side is $\sin(t)$ which means we have our $x_p$ in the form 
    \[
    x_P = A\cos(t)+B\sin(t)
    \]
    Now we have to find the undetermined coefficients by plugging in and solving
    \begin{align*}
        x_p''+2x_p'+x_p&=\sin(t)\\
    	-(A\cos(t)+B\sin(t))+2(-A\sin(t)+B\cos(t))+A\cos(t)+B\sin(t) &= \sin(t).
    \end{align*}
    which gives us the linear system of equations for $A$ and $B$
    \begin{align*}
    	-2A&=1\\
    	2B&=0,
    \end{align*}
    and we find $A=\frac{-1}{2}$ and $B=0$. Thus
    \[
    \boxed{x_P=-\frac{1}{2}\cos(t).}
    \]
    \item Now our general solution is given by 
    \[
    x=x_H+x_P=C_1 e^{-t}+C_2 te^{-t}-\frac{1}{2}\cos(t)
    \]
    and we need only use the initial conditions to find the particular solution (i.e., to determine both $C_1$ and $C_2$). First note that $x(0)=0$ so
	\begin{align*}
		0&=x(0)=C_1-\frac{1}{2},
	\end{align*}
	and thus $C_1=\frac{1}{2}$. Next, we have $x'(0)=0$ so
    \begin{align*}
    	0&=x'(0)=-C_1+C_2,
    \end{align*}
    and since we know $C_1$, we find that $C_2=\frac{1}{2}$. This leads us to our particular solution
    \[
    \boxed{x=\frac{1}{2}e^{-t}+\frac{1}{2}te^{-t}-\frac{1}{2}\cos(t).}
    \]
    \item Let us first take the derivative of our solution
    \[
    x'= -\frac{1}{2}te^{-t}+\frac{1}{2}\sin(t).
    \]
    Now we can plot a curve $(x,x')$ using the Desmos link:
    \begin{figure}[H]
    	\centering
    	\includegraphics[width=.4\textwidth]{stable_periodic.png}
    	\caption{The trajectory of our system in phase space.}
    \end{figure}
    
    \item We should really analyze this system by thinking physically to get a feeling for what's going on. Recall that
    \[
    \underbrace{mx'}_{\textrm{Newton's}} + \underbrace{\mu x'}_{\textrm{damping}} + \underbrace{kx}_{\textrm{Hooke's}} = \underbrace{F(t)}_{\textrm{external force}}.
    \]
    What we find in this case is that $x_H$ represents what happens with our physical system if we do not provide any external force (we often call this \emph{free}). The external force for these types of problems comes from the inhomogeneity (or the nonzero right hand side). The response to this external force is $x_P$. The true system dynamics $x$ are a combination of the free system dynamics plus its response to the force so that $x=x_H+x_P$. 
    
    In this case, we can picture that we start off with a particle with position at the origin $x(0)=0$ and no initial velocity $x'(0)=0$. In absence of external forces, we would find that no motion would occur due to Newton's laws. To see this, solve the homogeneous problem with these initial conditions. However, in this inhomogeneous case, we add in a periodic forcing term $\sin(t)$ which changes our dynamics noticeably. 
    
    Taking a look at our plot from before, we can note that the system responds to this forcing term by increasing in velocity $x'$ and position $x$. But, since there is damping in our system (the $x'$ term in the ODE is nonzero), the particle will lose energy due to some kind of friction. Likewise, there is a restoring force via Hooke's law that brings the particle back to its initial starting point $x=0$. Ultimately, the phase space trajectory for the particle seems to approach a circle showing us that this system appears to be periodic over time. This is definitely sensible since our forcing term itself is periodic!
\end{enumerate}
\end{solution}

\end{document}
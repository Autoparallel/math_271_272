%%%%%%%%%%%%%%%%%%%%%%%%%%%%%%%%%%%%%%%%%%%%%%%%%%%%%%%%%%%%%%%%%%%%%%%%%%%%%%%%%%%%
% Document data
%%%%%%%%%%%%%%%%%%%%%%%%%%%%%%%%%%%%%%%%%%%%%%%%%%%%%%%%%%%%%%%%%%%%%%%%%%%%%%%%%%%%
\documentclass[12pt]{article} %report allows for chapters
%%%%%%%%%%%%%%%%%%%%%%%%%%%%%%%%%%%%%%%%%%%%%%%%%%%%%%%%%%%%%%%%%%%%%%%%%%%%%%%%%%%%
\usepackage{preamble}

\begin{document}

\begin{center}
   \textsc{\large MATH 271, Homework 9}\\
   \textsc{Due November 15$^\textrm{th}$}
\end{center}
\vspace{.5cm}

\begin{problem}
Compute the following:
\begin{enumerate}[(a)]
    \item 
    \[
    [A]=\begin{pmatrix} 1& 1& 1 \end{pmatrix}
    \begin{pmatrix} 2\\ 1\\ 3 \end{pmatrix}.
    \]
    \item
    \[
    [B]=\begin{pmatrix} 1& 2& 3& 4\\ 5& 6& 7& 8\\ 9& 10& 11& 12\end{pmatrix}
    \begin{pmatrix} 3& 2\\ 2& 3\\ 3& 2\\ 2& 3\end{pmatrix}
    \]
    \item Take
    \[
    [M]=\begin{pmatrix} 10& 15\\ 20& 10 \end{pmatrix}
    \]
    and
    \[
    [N]=\begin{pmatrix} 1 & 2\\ 2& 1\end{pmatrix}.
    \]
    Compute $[M][N]$ and $[N][M]$ to see that matrices do not commute in general.
\end{enumerate}
\end{problem}

\begin{problem}
Compute the following determinants:
\begin{enumerate}[(a)]
    \item
    \[
    \det([A])=\left| \begin{array}{ccc}
    -3& 1 & 5\\
    -3& 4 & 2\\
    -3& 2 & 1
    \end{array}\right|
    \]
    \item 
    \[
    \det([B])=\left| \begin{array}{ccc}
    1& 2& 3\\
    4& 5& 6\\
    7& 8& 9
    \end{array}\right|
    \]
    \item Compute $\det([A][B])$ using properties of the determinant. \emph{Hint: this should be very quick to do. Do not compute the product of the matrices $[A]$ and $[B]$!}
\end{enumerate}
\end{problem}

\begin{problem}~
\begin{enumerate}[(a)]
    \item Show that for any $2\times 2$-matrix that the sign of the determinant changes if either a row or column is swapped. \emph{Note: this is true for square matrices of any size}.
    \item Show that for any $2\times 2$-matrix that multiplying a column by a constant scales the determinant by that constant as well. \emph{Note: this is true for square matrices of any size.}
    \item Show that for any $2\times 2$-matrix that adding a scalar multiple one column to the other will not change the determinant. \emph{Note: this is true in broader generality. In fact, adding linear combinations of columns to another column will not change the determinant.}
\end{enumerate}

\end{problem}

\begin{problem}$\boldsymbol{~^*}$
Using the facts above, argue that a square matrix with columns that are linearly dependent must have a determinant of zero.
\end{problem}

\begin{problem}
What does a zero determinant indicate about the solutions of a non-homogeneous system of linear equations? (Think geometrically!)
\end{problem}

\begin{problem}
What does a zero determinant indicate about the solutions of a homogeneous system of linear equations? (Think geometrically!)
\end{problem}

\begin{problem}
Given the matrices
\[
[A]=\begin{pmatrix} 1 & 0 & 2 \\ 2  & 1 & 3 \\ -2 & -2 & 0 \end{pmatrix} \qquad \textrm{and} \qquad [B]=\begin{pmatrix} -3 & 1 & 1 \\ 2 & -2 & 4 \\ -1 & -1 & -1 \end{pmatrix}.
\]
\begin{enumerate}[(a)]
    \item Compute $\tr([A])$ and $\tr([B])$.  
    \item Compute $\tr([A][B])$ and compare it to $\tr([B][A])$.
\end{enumerate}
\end{problem}

\begin{problem}
Consider the equation
\[
[A]\vecv = \zerovec,
\]
where
\[
[A] = \begin{pmatrix} 0 & 1 & 0 \\ 1 & 0 & 1 \\ 0 & 1 & 0 \end{pmatrix}.
\]
\begin{enumerate}[(a)]
    \item What vector(s) $\vecv$ satisfy this equation? In other words, what is $\Null([A])$?
    \item Using what you found above, what must $\det([A])$ be equal to? \emph{Hint: you do not need to compute the determinant!}
\end{enumerate}
\end{problem}

\end{document}
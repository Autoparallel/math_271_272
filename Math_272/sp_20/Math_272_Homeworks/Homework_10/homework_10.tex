%%%%%%%%%%%%%%%%%%%%%%%%%%%%%%%%%%%%%%%%%%%%%%%%%%%%%%%%%%%%%%%%%%%%%%%%%%%%%%%%%%%%
% Document data
%%%%%%%%%%%%%%%%%%%%%%%%%%%%%%%%%%%%%%%%%%%%%%%%%%%%%%%%%%%%%%%%%%%%%%%%%%%%%%%%%%%%
\documentclass[12pt]{article} %report allows for chapters
%%%%%%%%%%%%%%%%%%%%%%%%%%%%%%%%%%%%%%%%%%%%%%%%%%%%%%%%%%%%%%%%%%%%%%%%%%%%%%%%%%%%
\usepackage{preamble}

\newcommand{\vecfieldB}{\boldsymbol{\vec{B}}}
\newcommand{\vecfieldE}{\boldsymbol{\vec{E}}}
\newcommand{\vecfieldJ}{\boldsymbol{\vec{J}}}
\newcommand{\grad}{\boldsymbol{\vec{\nabla}}}
\newcommand{\veclaplace}{\boldsymbol{\vec{\Delta}}}
\newcommand{\vecfieldV}{\boldsymbol{\vec{V}}}

\begin{document}

\begin{center}
   \textsc{\large MATH 272, Homework 10}\\
   \textsc{Due April 27$^\textrm{th}$}
\end{center}
\vspace{.5cm}

\begin{problem}
Consider the 1-dimensional wave equation given by
\[
\left(-\frac{\partial^2}{\partial x^2} +\frac{1}{c^2} \frac{\partial^2}{\partial  t^2} \right) u(x,t) = 0.
\]
We'll consider two distinct scenarios. First, we'll take an infinitely long elastic rod and second we'll take a rod of finite length with Dirichlet boundary conditions.
\begin{enumerate}[(a)]
    \item For a rod of infinite length, consider the initial conditions
    \[
    u(x,0) = \begin{cases} x+1 & -1\leq x \leq 0 \\ 1-x & 0\leq x \leq 1 \\ 0 & \textrm{otherwise} \end{cases} \qquad \textrm{and} \qquad \frac{\partial}{\partial t} u(x,0) = 0.
    \]
    Find and plot the portion of the wave that moves to the right with $c=1$.
    \item Let $u_R(x,t)$ be your solution from (a), show that this satisfies the right-moving wave equation
    \[
    \left(\frac{\partial}{\partial x} + \frac{1}{c} \frac{\partial}{\partial t} \right)u_R(x,t) = 0.
    \]
    \item Why is it that we can ignore the points where your function $u_R(x,t)$ is not differentiable even though we are considering this as a solution to a PDE?
    \item For an elastic rod $\Omega$ of finite length, $\Omega = [0,1]$, assume that we take the Dirichlet conditions $u(0,t)=0=u(1,t)$.  With the initial conditions
    \[
    u(x,0) = \sin(\pi x) \qquad \textrm{and} \qquad \frac{\partial}{\partial t} u(x,0)=0,
    \]
    find the solution using d'Alembert's formula.
    \item Let $w(x,t)$ be your solution for (d), show that it is indeed equal to
    \[
    w(x,t) = \sin(\pi x)\cos(\pi c t).
    \]
    \item With your result from (e), explain how we can decompose a standing wave into a linear combination of two waves; one moving towards the left and one moving towards the right and both reflecting off the boundary.
\end{enumerate}
\end{problem}

\begin{problem}
Previously we studied the time-independent Schr\"odinger equation. Now, we can take a look at the time-dependent version given by
\[
H \Psi(x,t) = i\hbar \frac{\partial}{\partial t} \Psi(x,t),
\]
where $H$ is the Hamiltonian operator.  Consider the situation for the free particle in the 1-dimensional box of length $L$ so that $V(x)=0$ and $\Psi(0,t)=0=\Psi(L,t)$.  
\begin{enumerate}[(a)]
    \item Take a separation of variables ansatz and find a set of solutions (one for every positive integer $n$) to the time-dependent equation.
    \item Show that a super position of solutions is also a solution.
    \item For a single state $\psi_n(x,t)$, show that 
    \[
    \int_0^L \left|\psi_n(x,t)\right|^2 dx,
    \]
    is independent of $t$. This shows that the states $\psi_n$ are \emph{stationary} since their total probability does not depend on time.
\end{enumerate}
\end{problem}

\begin{problem}
Maxwell's equations are given as
\begin{align*}
\grad \cdot \vecfieldB &= 0  & \grad \cdot \vecfieldE &= \frac{\rho}{\epsilon_0}\\
\grad \times \vecfieldB -\mu_0 \epsilon_0\frac{\partial \vecfieldE}{\partial t}&=\mu_0 \vecfieldJ & \grad \times \vecfieldE + \frac{\partial \vecfieldB}{\partial t} &= \zerovec
\end{align*}
\begin{enumerate}[(a)]
    \item Look up each of the terms in the equations above and describe them.
    \item Describe what each equation is saying and why these are PDEs.
    \item In the absence of all charges we will have $\vecfieldJ=\zerovec$ and $\rho=0$.  Using that and the following two facts
    \[
    \veclaplace \vecfieldV = \grad (\grad \cdot \vecfieldV) - \grad \times (\grad \times \vecfieldV) \qquad \textrm{and} \qquad \grad \times \frac{\partial \vecfieldV}{\partial t} = \frac{\partial}{\partial t} (\grad \times \vecfieldV),
    \]
    derive the vector wave equations for light
    \[
    \left( - \veclaplace + \mu_0 \epsilon_0 \frac{\partial^2 }{\partial t^2}\right) \vecfieldE = \zerovec
    \]
    and
    \[
    \left( - \veclaplace + \mu_0 \epsilon_0 \frac{\partial^2 }{\partial t^2}\right) \vecfieldB = \zerovec
    \]
    \item From the equations you derived, determine the wave speed of light in the vacuum, $c_0$.
\end{enumerate}
\end{problem}


\end{document}
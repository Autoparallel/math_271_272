%%%%%%%%%%%%%%%%%%%%%%%%%%%%%%%%%%%%%%%%%%%%%%%%%%%%%%%%%%%%%%%%%%%%%%%%%%%%%%%%%%%%
% Document data
%%%%%%%%%%%%%%%%%%%%%%%%%%%%%%%%%%%%%%%%%%%%%%%%%%%%%%%%%%%%%%%%%%%%%%%%%%%%%%%%%%%%
\documentclass[12pt]{article} %report allows for chapters
%%%%%%%%%%%%%%%%%%%%%%%%%%%%%%%%%%%%%%%%%%%%%%%%%%%%%%%%%%%%%%%%%%%%%%%%%%%%%%%%%%%%
\usepackage{preamble}

\begin{document}

\begin{center}
   \textsc{\large MATH 271, Homework 5}\\
   \textsc{Due October 11$^\textrm{th}$}
\end{center}
\vspace{.5cm}


\begin{problem}[Euler's Formula]
	Given that
	\[
	\cos(x) = \sum_{n=0}^\infty \frac{(-1)^n x^{2n}}{(2n)!} \qquad \textrm{and} \qquad \sin(x) = \sum_{n=0}^\infty \frac{(-1)^n x^{2n+1}}{(2n+1)!},
	\]
	\begin{enumerate}[(a)]
		\item Plot the approximations of both $\cos(x)$ and $\sin(x)$ versus the original function for order 1,5,20 over the domain $[-4\pi,4\pi]$.
\item Show that
\[
e^{ix} = \cos(x) + i \sin(x),
\]
using the power series representation for the exponential function $e^x$.
\item Show that cosine is even, $\cos(-x)=\cos(x)$, and that sine is odd $\sin(-x)=-\sin(x)$.
\item Compute $\frac{d}{dx} e^{ix}$ using the series representation and show that $\frac{d}{dx}\cos(x)=-\sin(x)$ and $\frac{d}{dx}\sin(x)=\cos(x)$. 
	\end{enumerate}

\end{problem}

%\begin{problem} $p$-series are actually related to a very important function called the \emph{Riemann zeta function}.  This function is involved in a million dollar math problem! If you're interested in other million dollar problems, look up the Clay Institute Millennium Problems. The Riemann zeta function is given by
%\[
%\zeta (s) = \sum_{n=1}^\infty \frac{1}{n^s}.
%\]
%\begin{enumerate}[(a)]
%    \item Use the integral test to show that the $p$-series
%    \[
%    \sum_{n=1}^\infty \frac{1}{n^2}
%    \]
%    converges.  Look up what this series converges to and write it down. This is $\zeta(2)$.
%    \item Use the comparison test to show that the $p$-series
%    \[
%    \sum_{n=1}^\infty \frac{1}{n^3}
%    \]
%    converges. This converges as well to $\zeta(3)$. Look up what this approximate value is.
%\end{enumerate}
%\end{problem}

%\begin{problem}
%Find the radius of convergence for the following power series
%\begin{enumerate}[(a)]
%    \item $\displaystyle{\sum_{n=1}^\infty \frac{x^n}{n(n+1)}}$;
%    \item $\displaystyle{\sum_{n=0}^\infty \frac{x^{2n+1}}{(2n+1)!}}$.
%\end{enumerate}
%\end{problem}


\begin{problem} Consider the function
\[
f(x)=\frac{1}{1-x}.
\]
\begin{enumerate}[(a)]
    \item Compute the Taylor series centered at $a=0$ for the function.
    \item Find the antiderivative $\int \frac{dx}{1-x}$ using the Taylor series for $f(x)$ found in (a).  
    \item Write down the Taylor series for $\ln(1-x)$ centered at $a=0$ and compare to your answer in (b).
\end{enumerate}
\end{problem}

\begin{problem}~
\begin{enumerate}[(a)]
    \item Compute the Taylor series centered at $a=0$ for $f(x)=e^{-\frac{x^2}{2}}$.
    \item Use the Taylor series for $e^x$ and modify it to find a power series for $f(x)$. Is this the same as the series in (a)?
    \item Plot the original function $f(x)$ compared to the first, second, third, and fourth term approximation for the series on the same graph.
\end{enumerate}
\end{problem}

\begin{problem}
How can we approximate a (possibly complicated) function by using a power series? Why is this useful (specifically for computation on a computer)?
\end{problem}

\begin{problem}[Explicit Euler Method]
	Consider the differential equation $x'=kx$ where $k\in \C$ is a complex parameter for the system. Note that the solution to this equation given the initial condition $x(0)=1$ is $x=e^{kt}$. 
	\begin{enumerate}[(a)]
		\item Suppose we want to find an approximation to a solution using a computer. Let $t_0$ be some arbitrary time, define $\delta t$ to some fixed change in the input variable $t$ and let $\delta x =x(t_0+\delta t)-x(t_0)$ be the corresponding change in the output $x$. Compute the first order Taylor approximation of $x$ at the point $t_0$ to see that
		\begin{equation}
			\label{eq:1}
		x(t_0+\delta t) \approx x(t_0)+x'(t_0)(\delta t)
		\end{equation}
		from which you can then note that
		\begin{equation}
			\label{eq:2}
		x'(t_0) \approx \frac{\delta x}{\delta t}
		\end{equation}
	\item Define the \emph{explicit Euler approximation sequence} $\{x_\tau\}_{\tau=0}^T$ so that $x(t_0)=x_0$ and at later times $x(t_0+\tau\delta t)=x_\tau$. Show using the previous equations, the ODE itself, and the fact that $x_{\tau-1}+\delta x = x_{\tau}$ means can make a sequence
	\[
	x_{\tau} = x_{\tau-1} + kx_{\tau-1} \delta t.
	\]
	\item Let $k=1$, $\delta t = 0.01$, $t_0=0$, and let $x(0)=1$. Plot the explicit Euler approximation sequence using the following URL \url{http://www.calcul.com/show/calculator/recursive}. Compare this graph to the solution $x(t)=e^{t}$.
	\item Let $k=-1$, $\delta t = 2$, $t_0=0$, and let $x(0)=1$. Plot the approximation again. Is there something wrong? How does this compare to what the solution should be?
	\end{enumerate}
\end{problem}



\end{document}
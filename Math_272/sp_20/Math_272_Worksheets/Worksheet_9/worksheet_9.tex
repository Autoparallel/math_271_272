%%%%%%%%%%%%%%%%%%%%%%%%%%%%%%%%%%%%%%%%%%%%%%%%%%%%%%%%%%%%%%%%%%%%%%%%%%%%%%%%%%%%
% Document data
%%%%%%%%%%%%%%%%%%%%%%%%%%%%%%%%%%%%%%%%%%%%%%%%%%%%%%%%%%%%%%%%%%%%%%%%%%%%%%%%%%%%
\documentclass[12pt]{article} %report allows for chapters
%%%%%%%%%%%%%%%%%%%%%%%%%%%%%%%%%%%%%%%%%%%%%%%%%%%%%%%%%%%%%%%%%%%%%%%%%%%%%%%%%%%%
\usepackage{preamble}

\begin{document}

\begin{center}
   \textsc{\large MATH 272, Worksheet 9}\\
   \textsc{Cylindrical and Spherical Coordinates}
\end{center}
\vspace{.5cm}

\begin{problem}
    Consider a description of the plane $\R^2$ in both Cartesian coordinates $(x,y)$ and polar coordinates $(r,\theta)$.  Recall the coordinate transformations
    \begin{align*}
        x(r,\theta)&= r\cos \theta\\
        y(r,\theta)&= r\sin \theta.
    \end{align*}
    Let $f(x,y) = \frac{1}{\sqrt{1-x^2-y^2}}$ and let us attempt to integrate
    \[
    \iint_{\Sigma} f(x,y)d\Sigma,
    \]
    where $\Sigma$ is the unit disk defined by $x^2+y^2\leq 1$.
    \begin{enumerate}[(a)]
        \item Convert $f(x,y)$ in Cartesian coordinates to a function $f(r,\theta)$ in polar coordinates.
        \item Note that we can convert the area form $d\Sigma=dxdy$ in Cartesian coordinates to an area form in polar coordinates.  Think of the function
        \[
        \vec{\textrm{Pol}}(r,\theta) = \begin{pmatrix} x(r,\theta) \\ y(r,\theta) \end{pmatrix},
        \]
        as the function that coverts Cartesian coordinates into polar coordinates.  Then, 
        \[
        [J]_{\vec{\textrm{Pol}}} = \begin{pmatrix} \frac{\partial x}{\partial r} & \frac{\partial x}{\partial \theta} \\ \frac{\partial y}{\partial r} & \frac{\partial y}{\partial \theta} \end{pmatrix},
        \]
        is the Jacobian of this transformation. The magnitude of determinant of a matrix describes the stretching that the matrix does to the space at a point, and thus the magnitude of the determinant will be a function that depends on each point that describes the local stretching behavior.
        
        Compute $[J]_{\vec{\textrm{Pol}}}$ and compute the determinant of this matrix as well. Simplify this expression as much as possible.
        \item Now, to find the area form in polar coordinates, we simply take
        \[
        \left|\det\left([J]_{\vec{\textrm{Pol}}}\right)\right|drd\theta.
        \]
        Confirm that you have the area form $rdrd\theta$.
        \item Draw a picture of a small segment (sides $dr$ and $d\theta$) in the plane.  Can you see why the area of this segment depends on the radius? Can you also see why it does \underline{not} depend on the angle?
        \item We know we have this correct if 
        \[
        \int_{0}^{2\pi} \int_0^R rdrd\theta = \pi R^2,
        \]
        since this is the area contained inside a circle of radius $R$.  Show that the above integral is true.
        \item Now, set up the integral posed initially in polar coordinates and evaluate this integral using a change of variables (\underline{do not use WolframAlpha}).
    \end{enumerate}
\end{problem}

\vspace*{1cm}
\begin{problem}
    Repeat a very similar argument to show that the volume element $d\Omega$ in cylindrical coordinates is $\rho d\rho d\theta dz$.  \emph{Hint: The coordinate transformations for $x$ and $y$ are analogous.  But, we have the addition of the $z$-coordinate, but this is identical to the Cartesian $z$-coordinate.}
\end{problem}

\vspace*{1cm}
\begin{problem}
Provide an explicit or implicit description (both if possible) of the following regions in $\R^3$ in Cartesian coordinates, cylindrical coordinates, and spherical coordinates.
    \begin{enumerate}[(a)]
        \item A solid box with side lengths $a$, $b$, and $c$.
        \item A solid cylinder of height $h$.
        \item A solid ball of radius $R$.
    \end{enumerate}
\end{problem}

\vspace*{1cm}
\begin{problem}
Likewise, provide an explicit or implicit description (both if possible) of the following surfaces in $\R^3$ in Cartesian coordinates, cylindrical coordinates, and spherical coordinates.
    \begin{enumerate}[(a)]
        \item A the surface of a box with side lengths $a$, $b$, and $c$.
        \item A the surface of a cylinder of height $h$ including endcaps.
        \item A the surface of a ball of radius $R$ (i.e., the sphere).
    \end{enumerate}
\end{problem}

\vspace*{1cm}
\begin{problem}
One can also take a look at curves in each coordinate system.  Plot the following curves by hand. Play around with different choices of parameterizations. Hold certain variables constant. See how this affects your curve!
\begin{enumerate}[(a)]
    \item $x(t)=t$, $y(t)=t$, $z(t)=t$.
    \begin{itemize}
        \item What happens if we hold $x(t)=C$ constant? 
        \item What if we held both $x(t)=C_1$, and $y(t)=C_2$ constant?
        \item Choose some other functions (including constants) for $x(t)$, $y(t)$, and $z(t)$, and plot these as well.
    \end{itemize}
    \item $\rho(t)=t$, $\theta(t)=t$, $z(t)=t$.
    \begin{itemize}
        \item What happens if we hold $\rho(t)=C$ constant? 
        \item What if we held both $\rho(t)=C_1$, and $\theta(t)=C_2$ constant? Or held $\rho(t)$ and $z(t)$ constant?
        \item Choose some other functions (including constants) for $\rho(t)$, $\theta(t)$, and $z(t)$, and plot these as well.
    \end{itemize}
    \item $r(t)=t$, $\theta(t)=t$, $\phi(t)=t$.
    \begin{itemize}
        \item What happens if we hold $r(t)=C$ constant? 
        \item What if we held both $r(t)=C_1$, and $\theta(t)=C_2$ constant? Or held $r(t)$ and $\phi(t)$ constant?
        \item Choose some other functions (including constants) for $\rho(t)$, $\theta(t)$, and $z(t)$, and plot these as well.
    \end{itemize}
    \item ** Find a parameterization of a straight line passing through a point $(x_0,y_0,z_0)$ in cylindrical and spherical coordinates. \emph{Hint: This isn't too hard in those coordinate systems if the curve passes through the origin. But, it can be a bit difficult otherwise!}
\end{enumerate}
\end{problem}

\vspace*{1cm}
\begin{problem}
Integrate the following functions in their relevant coordinate systems over the given region $\Omega$.
\begin{enumerate}[(a)]
    \item $f(x,y,z) = xy+yz$ over $\Omega$ which is the solid unit cube.
    \item $f(\rho,\theta,z) = \frac{z}{\rho}\sin(\theta)$ over $\Omega$ which is the cylinder of radius $1$ and of height $2$. Align this cylinder so the $z$-axis runs through the core of the cylinder and so the height is split at the $xy$-plane.
    \item $f(r,\theta,\phi) = \frac{1}{r^2}\sin(\theta)\sin(\phi)$ over the unit ball centered at the origin.
\end{enumerate}
\end{problem}

\end{document}

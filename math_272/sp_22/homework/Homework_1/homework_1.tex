%%%%%%%%%%%%%%%%%%%%%%%%%%%%%%%%%%%%%%%%%%%%%%%%%%%%%%%%%%%%%%%%%%%%%%%%%%%%%%%%%%%%
% Document data
%%%%%%%%%%%%%%%%%%%%%%%%%%%%%%%%%%%%%%%%%%%%%%%%%%%%%%%%%%%%%%%%%%%%%%%%%%%%%%%%%%%%
\documentclass[12pt]{article} %report allows for chapters
%%%%%%%%%%%%%%%%%%%%%%%%%%%%%%%%%%%%%%%%%%%%%%%%%%%%%%%%%%%%%%%%%%%%%%%%%%%%%%%%%%%%
\usepackage{preamble}

\begin{document}

\begin{center}
   \textsc{\large MATH 272, Homework 1}\\
   \textsc{Due January 29$^\textrm{th}$}
\end{center}
\vspace{.5cm}

\begin{problem}
\textbf{(9 pts.)} Give a real world example of a
\begin{enumerate}[(a)]
    \item curve;
    \item scalar field;
    \item vector field;
\end{enumerate}
that take place in 3-dimensional space. (Don't use examples from this homework -- come up with your own new examples.) Explain the differences between each.
\end{problem}

\vspace*{0.5cm}

\begin{problem}
\textbf{(8 pts.)} Plot the following curves and print pictures of each using GeoGebra.
\begin{enumerate}[(a)]
	\item \textbf{(2 pts.)} (Helix) $\curvegamma_1(t) = \begin{pmatrix} 3\cos(t) \\ 3\sin(t) \\ t\end{pmatrix}$, from $t=0$ to $t=2\pi$. Where might this show up? If you think about the Earth moving through space and Moon orbiting Earth, then the Moon follows a (locally) helical path.

	\item \textbf{(2 pts.)} (Falling Ball) $\curvegamma_2(t) = \begin{pmatrix} t\\  0.5t \\ 9-t^2 \end{pmatrix}$ from $t=0$ to $t=3$.

	\item \textbf{(2 pts.)} (Trefoil knot) $\curvegamma_3(t) = \begin{pmatrix} \sin(t)+2\sin(2t) \\ \cos(t)-2\cos(2t) \\ -\sin(3t) \end{pmatrix}$ from $t=0$ to $t=2\pi$. (Note that this is the simplest nontrivial \emph{knot}. See: \url{https://en.wikipedia.org/wiki/Trefoil_knot} to learn more.)

	\item \textbf{(2 pts.)} Create your own curve.
\end{enumerate}
\end{problem}

\vspace*{0.5cm}

\begin{problem}
\textbf{(6 pts.)} For the helical curve $\curvegamma(t) = \begin{pmatrix} \cos(t) \\ \sin(t) \\ t \end{pmatrix}$, compute the tangent (velocity) vector and the normal (acceleration) vector as a function of $t$. Explain why there is no acceleration in the $z$-direction.
\end{problem}

\vspace*{0.5cm}

\begin{problem}
\textbf{(6 pts.)} The length of a curve is an important notion. In fact, the length of a curve is often related to the energy of some configuration. We can compute the length of a curve over the time $t=t_0$ to $t=t_1$ by integrating the \emph{speed} of the curve over that time.  That is,
\[
\ell(\curvegamma) = \int_{t_0}^{t_1} \left|\tangentgamma(t)\right| \dif t.
\]
We can compute the \emph{energy} of a curve by taking
\[
E(\curvegamma) = \int_{t_0}^{t_1} \frac{1}{2} \left| \tangentgamma(t)\right|^2 \dif t.
\]
Find the length and energy of the Helix from Problem 1 (a).
\end{problem}

\vspace*{0.5cm}

\begin{problem}
\textbf{(12 pts.)} Given a scalar field of two variables $f(x,y)$, we can create an object called the \emph{graph} of $f(x,y)$ by plotting the set of points $(x,y,f(x,y))$. In fact, you have done this many times in your life. For example, you have consistently plotted the graph of a function $f(x)$ by plotting $(x,f(x))$ in the plane!

Using GeoGebra, plot the graph of the following functions and include them with your submission.  Also, describe the what the graph of the function does as we move along the $x$-direction, the $y$-direction, and along the direction where $y=x$. For each, use the range $-3\leq x \leq 3$ and $-3\leq y \leq 3$. Explicitly write down the function along these slices, that is, along the $x$-direction, the $y$-direction, and $y=x$ direction!
\begin{enumerate}[(a)]
	\item \textbf{(4 pts.)} $f(x,y) = \frac{4xy}{1+x^2+y^2}$.
	\item \textbf{(4 pts.)} $g(x,y) = \sin(xy)$.
	\item \textbf{(4 pts.)} $h(x,y) = \frac{-x^2-y^2}{5}$.
\end{enumerate}
\end{problem}

\vspace*{0.5cm}

\begin{problem}
\textbf{(8 pts.)} Let us visualize vector fields using GeoGebra (specifically \url{https://www.geogebra.org/m/u3xregNW}). Plot the following vector fields and print them out.
\begin{enumerate}[(a)]
    \item \textbf{(2 pts.)} (Constant wind from the northwest) $\vecfieldV(x,y)=\begin{pmatrix} 1 \\ -1 \\ 0\end{pmatrix}$.
    \item \textbf{(2 pts.)} (Two wind fronts meeting) $\vecfieldU(x,y,z)=\begin{pmatrix} y \\ x \\ 0 \end{pmatrix}$.
    \item \textbf{(2 pts.)} (Source) $\vecfieldE(x,y,z) = \begin{pmatrix} \frac{x}{(x^2+y^2+z^2)^{3/2}} \\ \frac{y}{(x^2+y^2+z^2)^{3/2}} \\ \frac{z}{(x^2+y^2+z^2)^{3/2}} \end{pmatrix}$.
    \item \textbf{(2 pts.)} (Vortex) $\boldsymbol{\vec{S}}(x,y,z)=\begin{pmatrix} \frac{-y}{x^2+y^2+z^2} \\ \frac{x}{x^2+y^2+z^2} \\ 0\end{pmatrix}.$
\end{enumerate}
\end{problem}

\vspace*{0.5cm}

\begin{problem}
\textbf{(4 pts.)} Consider the vector field $\vecfieldV$ defined by
    \[
        \vecfieldV(x,y) = \begin{pmatrix} x^2 \\ xy \end{pmatrix}.
    \]
    Plot and label the vector field at the following points.
\begin{enumerate}[(a)]
    \item $\vecx_1 = (1,2)$;
    \item $\vecx_2 = (0,-3)$;
    \item $\vecx_3 = (-1,-1)$.
\end{enumerate}
\end{problem}

\vspace*{0.5cm}

\begin{problem}
\textbf{(2 pts.)} Consider the following scalar field and vector field
\[
f(x,y,z) = x^2+y^2-z^2.
\]
\begin{enumerate}[(a)]
    \item Compute all first order partial derivatives of $f$.
    \item Show that $\frac{\partial^2 f}{\partial x \partial y} = \frac{\partial^2 f}{\partial y \partial x}$.
\end{enumerate}
\end{problem}

\vspace*{0.5cm}

\begin{problem}
\textbf{(12 pts.)} For this problem, let us consider a family of scalar fields of varying dimensionality. In the previous problem, we plotted the graph of a scalar field with two inputs, but when there are more than two inputs we must resort to other methods of visualization.

In particular, we will seek out an understanding of the \emph{level sets} and how to relate these to the gradient of a scalar field. For each part, compute the set of points such that $f(\vecx)=1$, $f(\vecx)=2$, and $f(\vecx)=3$ and plot these sets (including all the different level sets in one plot per function).

Then for each field, compute the gradient (row) vector
\[
\nablavec f(\vecx) = \begin{pmatrix} \frac{\partial f}{\partial x_1} & \frac{\partial f}{\partial x_2} & \cdots & \frac{\partial f}{\partial x_n}\end{pmatrix}.
\]
Finally, draw an approximation of the the gradient vector field on your plots at a three different points for each part. (\emph{Hint: think of how the gradient relates to level sets of functions!})
\begin{enumerate}[(a)]
	\item \textbf{(3 pts.)} Consider the 1-dimensional scalar field
	\[
	f(x) = |x| = \sqrt{x^2}.
	\]
	Here each level set will be made up of distinct points.
	\item \textbf{(3 pts.)} Consider the 2-dimensional scalar field
	\[
	f(x,y) = |\vecx| = \sqrt{x^2+y^2}.
	\]
	Here each level set will be a curve.
	\item \textbf{(3 pts.)} Consider the 3-dimensional scalar field
		\[
		f(x,y,z) = |\vecx| = \sqrt{x^2+y^2+z^2}.
		\]
		Here each level set will be a surface.
\end{enumerate}
\end{problem}

\vspace*{0.5cm}

\begin{problem}
\textbf{(10 pts.)} A rough model of a molecular crystal can be described in the following way. Take the scalar function
\[
u(x,y)=\cos^2(x)+\cos^2(y).
\]
This function $u(x,y)$ describes the \emph{potential energy} for electrons in the crystal. Electrons are attracted to the areas with the smallest potential energy and move away from areas of high potential energy.
\begin{enumerate}[(a)]
    \item \textbf{(2 pts.)} Plot this function and include a printout.  Notice what this looks like.  You can imagine that each of the low points (well) is where a nucleus is located in the crystal.
    \item \textbf{(3 pts.)} Plot the level curves where $u(x,y)=0$, $u(x,y)=\frac{1}{4}$, $u(x,y)=\frac{1}{2}$, and $u(x,y)=1$ for the range of values $-\frac{3\pi}{2}\leq x \leq \frac{3\pi}{2}$ and $-\frac{3\pi}{2}\leq y \leq \frac{3\pi}{2}$. Picking the constant for the level curve tells you the \emph{kinetic energy} of the electron you are looking at.  It turns out that electrons (roughly) will orbit along these level curves.  Notice, some level curves bleed into the different troughs of neighboring molecules which means that electrons of sufficient energy happily flow through the crystal. For what energy values do the electrons move throughout the whole crystal?
    \item \textbf{(2 pts.)} Find the gradient of this function $\grad u(x,y)$.
    \item \textbf{(3 pts.)} At what point(s) is the gradient zero? \emph{Hint: Use your graph of the level curves to help.}
    \end{enumerate}
\end{problem}

\end{document}
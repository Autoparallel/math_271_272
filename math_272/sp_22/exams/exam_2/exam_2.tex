\documentclass[12pt]{amsbook}
\usepackage{preamble}


\begin{document}
\pagenumbering{gobble}       % This kills the page numbering

\begin{center}
   \textsc{\large MATH 272, Exam 2}\\
   \textsc{Oral Examination Problems}\\
   \textsc{Due one hour before your exam time slot.}
\end{center}

\vspace{1cm}

\noindent\textbf{Instructions} \; You are allowed a textbook, homework, notes, worksheets, material on our Canvas page.  You can use online tools such as Desmos and Wolfram Alpha to check your work, but you will need to explain how you arrived at your answers.  You can work with other students and this is, in fact, encouraged! However, I will not be giving out direct help for these problems but can answer questions about previous problems and notes, for example. Ambiguous or illegible answers will not be counted as correct. Scan your solutions and submit them as a pdf on Canvas under Oral Exam 2.


\vspace{1cm}


\hrule

\vspace*{1cm}
\noindent\emph{Note, there are three total problems.}

\newpage

\begin{problem}
Consider a two dimensional Helmholtz equation problem
\[
\begin{cases}
    \Delta f(x,y) = -k^2 f(x,y) & \textrm{on the unit square $0<x<1$ and $0<y<1$}\\
    f(x,y) = 0 & \textrm{on the boundary of the unit square}.
\end{cases}
\]
Note that $k$ is a real number that we are \underline{not} fixing to be any specific value. $k$ 
\begin{enumerate}[(a)]
\vspace*{0.25cm}
    \item Take a separation of variables ansatz $f(x,y)=X(x)Y(y)$ and separate the PDE into two ODEs.
\vspace*{0.25cm}
    \item Apply the boundary conditions for $X(x)$ and determine a set of general solutions $X_m(x)$ for each integer $m$.
\vspace*{0.25cm}
    \item Repeat the same process but for $Y(y)$ to determine a set of general solutions $Y_n(y)$ for each integer $n$.
\vspace*{0.25cm}
    \item Note that $f_{mn}=C_{mn}\sin(m\pi x)\sin(n \pi y)$. Does this agree with your solutions found by $f_{mn}(x,y)=X_m(x)Y_n(x)$? To this end, consider the following:
\vspace*{0.25cm}
\begin{itemize}
    \item What is $f_{mn}(x,y)$ if $m=0$? What about $n=0$? 
\vspace*{0.25cm}
    \item Plot your solution for $m=3$ and $n=1$ \emph{only on the unit square} and just let $C_{mn}=1$.
\end{itemize}
\vspace*{0.25cm}
    \item Deduce the constant $-k^2$ in terms of $m$ and $n$. 
\end{enumerate}
\end{problem}

\newpage
\begin{problem}
The wave equation in 2-dimensions can be used to model the vibrational states of membranes. Indeed, it is actually partly determined by the Helmholtz equation. To this end, let us consider the wave problem
\[
\begin{cases}
    \frac{\partial^2}{\partial t^2} u(x,y,t) = \Delta u(x,y,t) & \textrm{on the unit square}\\
    u(x,y,t) = 0 & \textrm{on the boundary of the unit square}\\
    u(x,y,0) = \sin(3\pi x)\sin(\pi y) \quad \frac{\partial}{\partial t} u(x,y,0)=0 &\textrm{as the initial conditions}.
\end{cases}
\]
where we have set $c=1$ and we note that $\Delta$ is the spatial Laplacian from Problem 1.
\begin{enumerate}[(a)]
\vspace*{0.25cm}
    \item Let $u(x,y,t)=T(t)f(x,y)$ where $f(x,y)$ is from Problem 1. Show that this reduces to the 2-dimensional Helmholtz equation for $f(x,y)$ and an ODE for $T(t)$.
\vspace*{0.25cm}
    \item Given that you have the solutions $f_{mn}(x,y)$ for the Helmholtz equation, solve the ODE for $T(t)$ to find a general solution. \emph{Hint: you know what $\Delta f_{mn}(x,y)$ is equal to and you know the value for $-k^2$. You can save yourself quite a bit of work by finding your solution $T(t)$ using this data. Note that $T(t)$ should have some dependence on your choice of $m$ and $n$!}
\vspace*{0.25cm}
    \item Find the particular solution for $u(x,y,t)$ using the given initial condition. 
\vspace*{0.25cm}
    \item Let us plot $u(x,y,t)$ \emph{only on the unit square}.
\vspace*{0.25cm}
\begin{itemize}
    \item Plot $u(x,y,0)$ and show this yields the same surface as in Problem 1.
\vspace*{0.25cm}
    \item Plot $u(x,y,0.1)$.
\vspace*{0.25cm}
    \item Plot $u(x,y,0.25)$.
\vspace*{0.25cm}
\end{itemize}
    \item What is the first time $\tau$ so that $u(x,y,0)=u(x,y,\tau)$?
\end{enumerate}
\end{problem}


\newpage
\begin{problem}
Suppose that we are considering a 2-dimensional time dependent Schr\"odinger equation
\[
\left(-\frac{\hbar^2}{2M} \Delta + V - i \hbar \frac{\partial}{\partial t}\right) \Psi(x,y,t)=0
\] 
on the unit square $\Omega = [0,1]\times [0,1]$. Suppose further that the particle we care about is not coupled to any potential (i.e., it is a free particle and hence the potential is zero). Then we have the solutions
\[
\psi_{mn}(x,y,t) = 2 e^{-i\frac{E_{mn}}{\hbar}t} \sin(m \pi x) \sin(n \pi y)
\]
for integers $m,n>0$ and where the energy eigenvalues are
\[
E_{mn} = \frac{(m^2+n^2) \hbar^2 \pi^2}{2M}.
\]
\begin{enumerate}[(a)]
\vspace*{0.25cm}
    \item Imagine now that we place an infinitely tall obstacle at the point $(x,y)=\left(\frac{1}{2},\frac{1}{2} \right)$. Explain why the state $\psi_{11}$ is no longer a solution.
\vspace*{0.25cm}
    \item To this end, what are the set of possible solutions to this problem with the included obstacle? \emph{Hint: Can you treat this as another boundary condition? This process only removes states from the problem where there was no obstacle, so which states are removed?}
\vspace*{0.25cm}
    \item Plot the probability amplitude of the ground state (i.e., lowest energy) of the problem with an obstacle only on the domain $\Omega$. To do this, just choose $M=1$ and $\hbar=1$.
\vspace*{0.25cm}
	\item \emph{Bonus:} Prove that the real and imaginary parts of stationary states to the Schr\"odinger equation (with or without the obstacle) satisfy the usual wave equation like that in Problem 2. What is the wave speed?
\end{enumerate}
\end{problem}






\end{document}  
\documentclass[12pt]{amsbook}
\usepackage{preamble}
\newcommand{\R}{\mathbb{R}}
\newcommand{\quati}{\mathbf{i}}
\newcommand{\quatj}{\mathbf{j}}
\newcommand{\quatk}{\mathbf{k}}



\begin{document}
\pagenumbering{gobble}       % This kills the page numbering

\begin{center}
   \textsc{\large MATH 271, Quiz 4}\\
   \textsc{Due November 13$^\textrm{th}$ at the end of class}
\end{center}

\vspace{1cm}

\noindent\textbf{Instructions} \; You are allowed a textbook, homework, notes, worksheets, material on our Canvas page, but no other online resources (including calculators or WolframAlpha) for this quiz.  \textbf{Do not discuss any problem any other person.} All of your solutions should be easily identifiable and supporting work must be shown.  Ambiguous or illegible answers will not be counted as correct.


\vspace*{.5cm}
\hrule
\vspace*{.5cm}

\begin{center}\textbf{\large THERE ARE 4 PROBLEMS AND 2 BONUS PROBLEMS.}\normalsize \end{center}

\begin{problem} Consider the matrix
\[
[A] = \begin{pmatrix} 1 & 3 \\ 3 & 0 \end{pmatrix}.
\]
\begin{enumerate}[(a)]
    \item \textbf{(3 pts.)} Show that for any arbitrary vectors $\vecu,\vecv \in \R^2$ that
    \[
        ([A] \vecu)\cdot \vecv = \vecu \cdot ([A] \vecv).
    \]
    \item \textbf{(2 pts.)} Is $[A]$ hermitian? Explain.
\end{enumerate}
\end{problem}

\begin{problem}
For the following decide whether the statement is true or false. For full credit, provide an adequate explanation for your answer.
\begin{enumerate}[(a)]
    \item \textbf{(2 pts.)} Every matrix is diagonalizable.
    \item \textbf{(2 pts.)} Every $n\times n$ matrix has $n$ complex eigenvalues.
    \item \textbf{(2 pts.)} The trace satisfies $\tr([A][B][C]) = \tr([A])\tr([B])\tr([C])$.
    \item \textbf{(2 pts.)} Similar matrices have the same eigenvalues.
\end{enumerate}
\end{problem}

\begin{problem}
Consider the matrix
\[
[A] = \begin{pmatrix} 2 & 1 \\ 1 & 2 \end{pmatrix}.
\]
\begin{enumerate}[(a)]
    \item \textbf{(2 pts.)} Argue why the eigenvalues of $[A]$ must be real.
    \item \textbf{(2 pts.)} Find the eigenvalues of $[A]$.
    \item \textbf{(2 pts.)} Find the eigenvectors of $[A]$.
    \item \textbf{(2 pts.)} Show that the eigenvectors of $[A]$ are orthogonal.
    \item \textbf{(2 pts.)} Explain how you can use the eigenvectors of $[A]$ to transform $[A]$ into a similar diagonal matrix.
\end{enumerate}
\end{problem}


\begin{problem}
\textbf{(3 pts.)} Let $A$ be a linear transformation and let $\evec_1,\dots,\evec_k$ be eigenvectors all with corresponding eigenvalue $\lambda$.  Show that any vector in the span of $\{\evec_1,\dots,\evec_k\}$ is also an eigenvector with eigenvalue $\lambda$.
\end{problem}


\begin{problem}
    \textbf{(Bonus 3 pts.)} Consider the following set $\{1,-1,\quati,-\quati,\quatj,-\quatj,\quatk,-\quatk\}$ with the relationships
    \[
    \quati^2=\quatj^2=\quatk^2=\quati \quatj \quatk=-1,
    \]
    where $1$ and $-1$ behave as expected. Show that this set with multiplication is a group (called the \emph{quaternion group}).
\end{problem}

\begin{problem}
    \textbf{(Bonus 3 pts.)} Consider the group of $n\times n$ invertible matrices denoted by $\operatorname{GL}(n)$. Explain why $\operatorname{GL}(n)$ is a group when the product operation is given by matrix multiplication but it is \textbf{\underline{NOT}} a group if we instead chose the operation to be addition.
\end{problem}

\end{document} 
\documentclass[12pt]{amsbook}
\usepackage{preamble}


\begin{document}
\pagenumbering{gobble}       % This kills the page numbering


\begin{center}
   \textsc{\large MATH 271, Quiz 2}\\
   \textsc{Due September 17$^\textrm{th}$ at the end of class}
\end{center}

\vspace{1cm}

\makebox[.8\textwidth]{Name:\enspace\hrulefill}

\vspace{.5cm}

\renewcommand{\arraystretch}{1.5}
\begin{center}
\begin{tabular}{|c|c|c|c||c|}
\hline
    \textrm{Problem 1} & \textrm{Problem 2} & \textrm{Problem 3} & \textrm{Problem 4}  & \qquad\textrm{Total}\qquad\qquad\\
\hline
~ & ~ & ~ & ~ &  ~\\
~ & ~ & ~ & ~ &  ~\\
~ & ~ & ~ & ~ &  ~\\
\hline
    \textbf{8 pts.} & \textbf{4 pts.} & \textbf{6 pts.} & \textbf{4 pts.} &  \textbf{20 pts.}\\
\hline
\end{tabular}
\end{center}


\vspace*{.5cm}
\hrule
\vspace*{.5cm}

\noindent \textbf{\large There are 4 problems worth a total of 22 points.}\\
\noindent \textbf{\large The quiz will be graded out of 20.}

\normalsize

\vspace*{.5cm}
\hrule
\vspace*{.5cm}


\begin{center}\fbox{\fbox{\parbox{6in}{\textbf{Instructions:} \; You are allowed a textbook, homework, notes, worksheets, and material on our Canvas page, but no other online resources (including calculators or WolframAlpha) for this quiz.  \textbf{Do not discuss any problem any other person.} All of your solutions should be easily identifiable and supporting work must be shown.  Ambiguous or illegible answers will not be counted as correct. Staple your work to this sheet.}}}\end{center}


\vspace*{0.5cm}



\begin{problem} ~
\begin{enumerate}[(a)]
\item \textbf{(3 pts.)} Give a definition of a sequence and explain the difference between a sequence and series. (\emph{This does not need to be overly precise!})

\item \textbf{(3 pts.)} Give a definition for a sequence to converge to a limit $L$. How can you use this to give a definition for a series to converge? (\emph{Again, this does not need to be overly precise!})

    \item \textbf{(2 pts.)} Explain why the following statement is true or false. To show the statement is true, provide a sound argument. To show a statement is false, you need only provide a counter example.\\

\emph{If the sequence $a_n \to 0$, then $\displaystyle{\sum_{n=1}^\infty a_n}$ converges.}

\end{enumerate}
\end{problem}

\begin{problem}
Note that the power series for $\sin(x)$ is given by
\[
\sin(x) = \sum_{n=0}^\infty \frac{(-1)^{n}}{(2n+1)!}x^{2n+1}.
\]
\begin{enumerate}[(a)]
\item \textbf{(2 pts.)} To third order, we have
\[
\sin(x) \approx x -\frac{x^3}{6}.
\]
Approximate $\sin(\pi)$ and compare to to the true answer ($\sin(\pi)=0$). \emph{Hint: you can assume $\pi \approx 3$.}
\item \textbf{(2 pts.)} Write down the power series for $\sin(3x^3)$.
\end{enumerate}
\end{problem}

\vspace*{0.5cm}
%\begin{problem}~
%\begin{enumerate}[(a)]
%\item \textbf{(2 pts.)} Compute the derivative of the power series
%\[
%f(x) = \sum_{n=1}^\infty \frac{x^n}{n^n}.
%\]
%\item \textbf{(2 pts.)} Compute the antiderivative of the power series
%\[
%g(x) = \sum_{n=0}^\infty (n+1)x^n.
%\]
%\end{enumerate}
%\end{problem}


\begin{problem}~
\begin{enumerate}[(a)]
\item \textbf{(2 pts.)} Consider the recursive sequence
\[
a_{n} = \frac{1}{2} a_{n-1}
\]
with the base case $a_0=1$. Show that the formula for $a_n$ is
\[
a_n = \frac{1}{2^n}.
\]
\item \textbf{(2 pts.)} Consider the power series with these coefficients
\[
f(x)=\sum_{n=0}^\infty a_n x^n = \sum_{n=0}^\infty \frac{x^n}{2^n}.
\]
Argue that
\[
f(x)=\frac{2}{2-x}.
\]
\emph{Hint: can you think of this as a geometric series?}
\item \textbf{(2 pts.)} Compute the derivative of $f(x)$ using the power series representation.
\end{enumerate}
\end{problem}


\begin{problem} Consider the function
\[
f(x)=\sqrt{x}.
\]
\begin{enumerate}[(a)]
\item \textbf{(2 pts.)} Find the second order Taylor approximation to $f(x)$ at the point $x=1$.
\item \textbf{(2 pts.)} Consider the nonlinear 2$\textrm{nd}$ order autonomous equation
\[
x'' + x' + \sqrt{x} = 0.
\]
How could we make an approximate ODE using Taylor approximations?
\end{enumerate}
\end{problem}

%\begin{problem}~
%\begin{enumerate}[(a)]
%    \item \textbf{(3 pts.)} Write down the first order approximation to $\tan(x)$ about the point $x=0$.
%    \item \textbf{(2 pts.)} Explain how you would determine a higher order approximation.
%\end{enumerate}
%\end{problem}


%\begin{problem}
%\textbf{(2 pts.)} Consider the differential equation
%\[
%x' = t^2 + \tan(x).
%\]
%Explain how you can use a first order approximation to $\tan(x)$ to approximate the above (nonlinear) equation as a first order linear equation. \textcolor{red}{Should say something about initial conditions}
%\end{problem}



\end{document}
%%%%%%%%%%%%%%%%%%%%%%%%%%%%%%%%%%%%%%%%%%%%%%%%%%%%%%%%%%%%%%%%%%%%%%%%%%%%%%%%%%%%
% Document data
%%%%%%%%%%%%%%%%%%%%%%%%%%%%%%%%%%%%%%%%%%%%%%%%%%%%%%%%%%%%%%%%%%%%%%%%%%%%%%%%%%%%
\documentclass[12pt]{article} %report allows for chapters
%%%%%%%%%%%%%%%%%%%%%%%%%%%%%%%%%%%%%%%%%%%%%%%%%%%%%%%%%%%%%%%%%%%%%%%%%%%%%%%%%%%%
\usepackage{preamble}

\begin{document}

\begin{center}
   \textsc{\large MATH 271, Worksheet 4}\\
   \textsc{Second Order Linear Equations and Boundary Value Problems}
\end{center}
\vspace{.5cm}

\begin{problem}
Write down the characteristic polynomial for the following equations.  Then, find the roots to the characteristic polynomial and write down the general solution.
\begin{enumerate}[(a)]
    \item $x''+x'+x=0$.
    \item $x''-x'-x=0$.
    \item $x''-x'+x=0$.
    \item $x''+x'-x=0$.
\end{enumerate}
\end{problem}

\begin{problem}
For the above solutions, analyze their behavior qualitatively. That is, do the solutions oscillate, grow, decay, or some combination of these, or something else entirely?
\end{problem}

\begin{problem}
Consider the equation
\[
x''+bx'+cx=0.
\]
The roots to the characteristic polynomial are then
\[
\lambda = \frac{-b\pm \sqrt{b^2-4c}}{2}.
\]
\begin{enumerate}[(a)]
    \item Explain why if $c>0$ and $b=0$ the solution $x(t)$ will be purely oscillatory.
    \item Explain why if $b>0$ and $b^2<4c$, the solution will oscillate and decay.
    \item Explain why if $b<0$ and $b^2<4c$, the solution will oscillate and grow.
\end{enumerate}
\end{problem}

\begin{problem}
Write down a second order linear differential equation that oscillates and also decays over time.
\end{problem}

\begin{problem}
Consider the following differential equation
\[
x''+x=0.
\]
\begin{enumerate}[(a)]
    \item Find the general solution to this equation.
    \item Given the initial conditions $x(0)=1$ and $x'(0)=1$, find the particular solution.
    \item Plot your particular solution.
    \item Does the solution grow or decay over time? 
    \item What is $\lim_{t\to \infty}x(t)$?
\end{enumerate}

\end{problem}

\begin{problem}
Next, consider a related equation
\[
x''+x=t.
\]
that has an additional linear external force.
\begin{enumerate}[(a)]
    \item What is the solution to the homogenous equation?
    \item Find the particular integral with the given forcing term.
    \item What is the specific solution to this equation?
    \item Does the solution grow or decay over time?
    \item What is $\lim_{t\to \infty}x(t)$?
\end{enumerate}
\end{problem}

\begin{problem}
Consider now the equation
\[
x''+x=F(t)
\]
where the external force is $F(t)=\cos(t)$.
\begin{enumerate}[(a)]
    \item Find the particular integral with the given forcing term.
    \item What is the specific solution to this equation?
    \item What is $\lim_{t\to \infty}$? What does this mean about the growth or decay of the solution over time?
\end{enumerate}
\end{problem}

\begin{problem}
Consider the boundary value problem
\[
x''=g
\]
with boundary values $x(0)=0$, $x\left(-\frac{2}{g}\right)=0$ and $g=-9.8[m/s^2]$.  We can think of this as solving the \emph{inverse problem} of one that we have seen in a homework. Specifically, think of this as knowing where a ball is launched and knowing where it lands and trying to find the speed it must have been thrown at.  

Another interpretation is the shape of a rod bending due to gravity.  $x''$ would measure the curvature of this rod, and this equation would say that the rod under the force of gravity would have a constant curvature. In this case, the dependent variable $t$ should be thought of as spatial rather than temporal.

Finally, this equation above is referred to as \emph{Poisson's equation}.
\begin{enumerate}[(a)]
    \item Find the general solution. If you already know it from the homework, just write it down.
    \item Use the boundary values above to find the particular solution.
    \item Is the solution unique? 
\end{enumerate}
\end{problem}

\begin{problem}
Consider the \emph{time independent Sch\"odinger equation} for a \emph{free particle} constrained inside of a 1-dimensional box of length $L$. That is, we have the equation
\[
-\frac{\hbar^2}{2m}\frac{d^2}{dx^2}\psi(x)=E\psi(x)
\]
on the unit interval $[0,L]$.
\begin{enumerate}[(a)]
    \item Find the general solution to this equation with no constraint.
    \item Given the constraint, we have the boundary values $\psi(0)=\psi(L)=0$. What are the general solutions given this constraint?
    \item Show that the sum of two solution $\psi_1(x)$ and $\psi_2(x)$ is also a solution. When we have a particle whose state (or \emph{wavefunction}) $\psi$ is a sum of general solutions, we say that $\psi$ is in a \emph{superposition state.}
    \item The wavefunction is not really a physically meaningful quantity.  However, if we consider a region $[a,b]$ in the box $[0,L]$ the quantity
    \[
    P([a,b])=\int_a^b |\psi(x)|^2dx
    \]
    \emph{is} meaningful. This expression tells us the \emph{probability} that a particle will be observed in the region $[a,b]$.  Take your general solutions you found in (b) (with the constraint) and solve for the constants that give you
    \[
    \int_0^L |\psi(x)|^2dx=1.
    \]
    We call this \emph{normalization} and we must do so for each state so that we can interpret the integral $P([a,b])$ as a probability.
\end{enumerate}
\end{problem}











\end{document}

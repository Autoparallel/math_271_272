%%%%%%%%%%%%%%%%%%%%%%%%%%%%%%%%%%%%%%%%%%%%%%%%%%%%%%%%%%%%%%%%%%%%%%%%%%%%%%%%%%%%
% Document data
%%%%%%%%%%%%%%%%%%%%%%%%%%%%%%%%%%%%%%%%%%%%%%%%%%%%%%%%%%%%%%%%%%%%%%%%%%%%%%%%%%%%
\documentclass[12pt]{article} %report allows for chapters
%%%%%%%%%%%%%%%%%%%%%%%%%%%%%%%%%%%%%%%%%%%%%%%%%%%%%%%%%%%%%%%%%%%%%%%%%%%%%%%%%%%%
\usepackage{preamble}

\begin{document}

\begin{center}
   \textsc{\large MATH 271, Homework 4}\\
   \textsc{Due October 4$^\textrm{th}$}
\end{center}
\vspace{.5cm}

\begin{problem}
Consider the following sequences,
\[
a_n = \frac{1}{2}, \frac{1}{4}, \frac{1}{8}, \dots, \frac{1}{2^n}, \dots,
\]
and
\[
b_n =  1, \frac{1}{2}, \frac{1}{6},\frac{1}{24}, \dots,\frac{1}{n!}, \dots.
\]
\begin{enumerate}[(a)]
    \item For what values of $N$ do we need for $a_N<0.01$ and $b_N<0.01$? Note, these will be different values for $N$.
    \item Compute $\displaystyle{\lim_{n\to \infty} a_n}$.
    \item Compute $\displaystyle{\lim_{n\to \infty} b_n}$.
    \item Which sequence converges more quickly to its limit? (\emph{Hint: consider a ratio of the terms of the sequences and take a limit. Part (a) should help you think about this.})
\end{enumerate}
\end{problem}

\begin{problem}
With the same $a_n$ from $1$, consider the series
\[
A = \sum_{n=1}^\infty a_n.
\]
\begin{enumerate}[(a)]
    \item Write down the $N^\textrm{th}$ partial sum $A_N$ for this series.  
    \item Does this sequence of partial sums converge? If so, to what?
    \item Note that this is an \emph{geometric series} with $a=1$ and $r=\frac{1}{2}$. However, we start from $n=1$ instead of $n=0$. Show the value that this series converges to using the formula for a geometric series.
\end{enumerate}
\end{problem}

\begin{problem}
With the same $b_n$ from $1$, consider the series
\[
B = \sum_{n=0}^\infty b_n.
\]
\begin{enumerate}[(a)]
    \item Use the ratio test to show that this series converges.
    \item Approximate the value the series converges to by considering larger and larger partial sums.
    \item What number does this series converge to?
\end{enumerate}
\end{problem}

\begin{problem}
Consider the two series
\[
\cos(x) = \sum_{n=0}^\infty \frac{(-1)^n x^{2n}}{(2n)!} \qquad \textrm{and} \qquad \sin(x) = \sum_{n=0}^\infty \frac{(-1)^n x^{2n+1}}{(2n+1)!}.
\]
\begin{enumerate}[(a)]
    \item Show that $\cos(-x)=\cos(x)$.
    \item Show that $\sin(-x)=-\sin(x)$.
    \item To take a derivative of a power series $f(x) = \displaystyle{\sum_{n=0}^\infty a_n x^n}$ we can do the following: 
    \[
    \frac{d}{dx}f(x) = \frac{d}{dx} \sum_{n=0}^\infty a_n x^n = \sum_{n=0}^\infty a_n \frac{d}{dx} x^n.
    \]
    Compute $\frac{d}{dx} \sin(x)$ and $\frac{d}{dx} \cos(x)$ and show that they are equal to what you already know. \emph{Warning: be careful with the powers of $x$ in the case with $\sin$ and $\cos$!}
\end{enumerate}
\end{problem}

\begin{problem}
Consider the \emph{$p$-series}:
\[
\sum_{n=1}^\infty \frac{1}{n^p}.
\]
\begin{enumerate}[(a)]
    \item For $p=1$, show that the ratio test is inconclusive.
    \item For $p=2$, show that the ratio test is again inconclusive.
    \item Look up the sum of the series for $p=1$ and $p=2$.  Notice how the ratio test is not perfect!
\end{enumerate}
\end{problem}

\end{document}
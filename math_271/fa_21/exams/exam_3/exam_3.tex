\documentclass[12pt]{amsbook}
\usepackage{preamble}
%\newcommand{\vecy}{\boldsymbol{\vec{y}}}
%\newcommand{\vecu}{\boldsymbol{\vec{u}}}
%\newcommand{\vecv}{\boldsymbol{\vec{v}}}
%\newcommand{\vecw}{\boldsymbol{\vec{w}}}
\newcommand{\R}{\mathbb{R}}
\newcommand{\C}{\mathbb{C}}
\newcommand{\blade}[1]{\boldsymbol{\vec{#1}}}
\newcommand{\ehat}{\boldsymbol{\hat{e}}}


\begin{document}
\pagenumbering{gobble}       % This kills the page numbering

\begin{center}
   \textsc{\large MATH 271, Exam 3}\\
   \textsc{Oral Examination Problems}\\
   \textsc{Due one hour before your exam time slot.}
\end{center}

\vspace{1cm}

\noindent\textbf{Instructions} \; You are allowed a textbook, homework, notes, worksheets, material on our Canvas page.  You can use online tools such as Desmos and Wolfram Alpha to check your work, but you will need to explain how you arrived at your answers.  You can work with other students and this is, in fact, encouraged! However, I will not be giving out direct help for these problems but can answer questions about previous problems and notes, for example. Ambiguous or illegible answers will not be counted as correct. Scan your solutions and submit them as a pdf on Canvas under Oral Exam 3.


\vspace{1cm}


\hrule

\vspace*{1cm}
\noindent\emph{Note, there are four total problems and a bonus.}

\newpage
\begin{problem}
Consider the following vectors in $\R^2$:
\[
\vecu = \xhat -3\yhat \quad \vecv = -2\xhat + 2\yhat \quad \vecw = -\xhat -\yhat.
\]
\begin{enumerate}[(a)]
    \item Draw all vectors $\vecu$, $\vecv$, and $\vecw$ in the plane. Draw $\vecu + \vecv$ in the plane as well.
    \item Are any of these vectors orthogonal? Explain.
    \item Compute the area of the parallelogram generated by $\vecu$ and $\vecv$. Draw this parallelogram in the plane.
    \item Explain why $\vecu$ and $\vecv$ form a basis for $\R^2$.
    \item Given the vector $\vecy = 13\xhat + \yhat$, write $\vecy$ as a linear combination of $\vecu$ and $\vecv$.
\end{enumerate}
\end{problem}

%\newpage
%\begin{problem}
%Consider the vectors $\vecu, \vecv \in \R^3$ given by
%\[
%\vecu = -\xhat - 2\yhat \qquad \textrm{and} \qquad \vecv = 3\zhat.
%\]
%\begin{enumerate}[(a)]
%    \item Compute $\vecu \times \vecv$.
%    \item Given any $\vecw \in \R^3$ with $\vecw = w_1 \xhat + w_2 \yhat + w_3\zhat$ we can create the matrix
%    \[
%    [\vecw] = \begin{pmatrix} 0 & -w_3 & w_2 \\ w_3 & 0 & -w_1 \\ -w_2 & w_1 & 0 \end{pmatrix}.
%    \]
%    Show that
%    \[
%    [\vecu \times \vecv] = [\vecu][\vecv]-[\vecv][\vecu].
%    \]
%\end{enumerate}
%
%\end{problem}

\newpage
\begin{problem}
Consider a linear transformation $T\colon \R^3 \to \R^3$ defined by
\begin{align*}
T(\ehat_1) &= \ehat_1 + \ehat_2 + \ehat_3\\
T(\ehat_2) &= \ehat_1 + \ehat_2\\
T(\ehat_3) &= 2\ehat_1 + 2\ehat_2 + \ehat_3\\
\end{align*}
\begin{enumerate}[(a)]
    \item What is the kernel of $T$?
    \item What is the image of $T$?
    \item Write down a matrix representation $[T]$ for the transformation $T$.
    \item Compute the characteristic polynomial for $T$, $p(\lambda)$. Then show that $p([T])=[0]$ is the zero matrix. \emph{This just means to plug in the matrix $[T]$ for the variable $\lambda$ and $[0]$ is the all zeros matrix.}
\end{enumerate}
\end{problem}


\newpage
\begin{problem}
Take the matrix
\[
[A] = \begin{pmatrix} 1 & 0 & 0 \\ 0 & 0 & 1 \\ 0 & 1 & 0 \end{pmatrix}.
\]
\begin{enumerate}[(a)]
    \item Show that $\ehat_1$ is an eigenvector with eigenvalue $\lambda_1=1$.
    \item Compute $\det([A])$ and $\tr([A])$ and using these quantities plus your knowledge from (a), show that the other two eigenvalues are $\lambda_2 = 1$ and $\lambda_3 = -1$. (DO NOT USE THE CHARACTERISTIC POLYNOMIAL!)
\end{enumerate}
\end{problem}

\newpage
\begin{problem}
Consider the matrix
\[
[A] = \begin{pmatrix} 1 & 1 \\ 0 & 2 \end{pmatrix}.
\]
\begin{enumerate}[(a)]
    \item Find the eigenvalues and eigenvectors for this matrix.
    \item Construct the matrix $[P]$ such that
    \[
    [\Lambda] = [P]^{-1}[A][P]
    \]
    from the eigenvectors you found.
    \item Find $[P]^{-1}$ and compute
    \[
    [\Lambda] = [P]^{-1}[A][P].
    \]
    Is this $[\Lambda]$ diagonal?
\end{enumerate}
\end{problem}

\newpage
\begin{problem}[BONUS]
Let $V$ be a vector space and let $\langle \blade{u},\blade{v} \rangle$ be an inner product. We say that an operator $T\colon V \to V$ is self-adjoint if $\langle T\blade{u},\blade{v}\rangle = \langle \blade{u},T\blade{v}\rangle$. Let $T \colon \C^n \to \C^n$ be self adjoint and take the inner product on $\C^n$ to be given by
\[
\langle \blade{u},\blade{v}\rangle = \sum_{j=1}^n u_j v_j^*
\]
We want to prove the two theorems in the text.
\begin{enumerate}[(a)]
    \item Show that all eigenvalues of $A$ are real.
    \item Show that eigenvectors corresponding to different eigenvalues are orthogonal with the hermitian inner product.
\end{enumerate}
\end{problem}






\end{document}
\documentclass[12pt]{amsbook}
\usepackage{preamble}


\begin{document}
\pagenumbering{gobble}       % This kills the page numbering

\begin{center}
   \textsc{\large MATH 271, Exam 1}\\
   \textsc{Oral Examination Problems}\\
   \textsc{Due one hour before your exam time slot.}
\end{center}

\vspace{1cm}

\noindent\textbf{Instructions} \; You are allowed a textbook, homework, notes, worksheets, material on our Canvas page.  You can use online tools such as Desmos and Wolfram Alpha to check your work, but you will need to explain how you arrived at your answers.  You can work with other students and this is, in fact, encouraged! However, I will not be giving out direct help for these problems but can answer questions about previous problems and notes, for example. Ambiguous or illegible answers will not be counted as correct. Scan your solutions and submit them as a pdf on Canvas under Oral Exam 1.


\vspace{1cm}


\hrule

\vspace*{1cm}
\noindent\emph{Note, there are three total problems.}

\newpage

\begin{problem}
Consider the endothermic breakdown of a molecule $x$ given by
\[
x \xrightarrow{kt^2} \textrm{Products}
\]
where we let $x(t)$ denote the concentration of reactants. Since the reaction is endothermic, if we also heat up the solution over time, we get a factor of $t^2$ as well since the reaction occurs more readily in higher temperatures. The concentration decreases over time based on differential equation
\[
x'=-kt^2(x-x_e).
\]
where $x_e$ is a constant that denotes the equilibrium concentration.
\begin{enumerate}[(a)]
    \item Write an equivalent equation with the change of variables $\delta=x-x_e$.
    \item Find the general solution to this new equation.
    \item What is the general solution in terms of the original variables $x$?
    \item Given the initial amount of $x$ is $x(0)=1$, the equilibrium concentration is $x_e=1/2$, and $k=1$, find the particular solution for $x(t)$.
    \item Does this reaction ever reach the equilibrium state?
\end{enumerate}
\end{problem}

\newpage
\begin{problem}
Let the height above ground at time $t$ be given by the function $y(t)$. A ball falling through air experiences gravitational acceleration and damping due to air friction. It follows the differential equation
\[
y'' = -ky' - g.
\]
\begin{enumerate}[(a)]
    \item Describe the type of this equation (e.g., separable, autonomous, linear, etc.). What is the order?
    \item Find the solution to the homogeneous equation. Call this solution $x_H$.
    \item Find the particular integral to the inhomogeneous equation. Call this $x_P$.
    \item What is the general solution to the ODE given in this problem?
    \item Suppose that $k=1$ and $g=10$. Let $y(0)=1,000$ and $y'(0)=0$. Find the particular solution to this problem.
    \item Plot your particular solution. Plot the derivative of your solution $y'(t)$ as well. If your solution is correct, the derivative (velocity) $y'$ should approach a constant value called the \emph{terminal velocity}.
\end{enumerate}
\end{problem}


\newpage
\begin{problem}
Consider the Schr\"odinger equation for a particle in a box of length 1,
\[
-\frac{\hbar^2}{2m}\frac{d^2 \Psi}{dx^2} = E\Psi,
\]
with boundary conditions $\Psi(0)=0$ and $\Psi(1)=0$.
\begin{enumerate}[(a)]
    \item Find the general solution to the differential equation.
    \item Apply the boundary conditions and write down a solution for each positive integer $n$. Recall that we call these solutions \emph{states} and denote the states by $\psi_n$.
    \item Determine the normalization constant for each $\psi_n$.  Does this constant depend on $n$?
    \item Explain why a superposition of states is also a solution to the Schr\"odinger equation.
\end{enumerate}
\end{problem}






\end{document}
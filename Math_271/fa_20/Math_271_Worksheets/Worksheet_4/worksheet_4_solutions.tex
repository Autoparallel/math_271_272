%%%%%%%%%%%%%%%%%%%%%%%%%%%%%%%%%%%%%%%%%%%%%%%%%%%%%%%%%%%%%%%%%%%%%%%%%%%%%%%%%%%%
% Document data
%%%%%%%%%%%%%%%%%%%%%%%%%%%%%%%%%%%%%%%%%%%%%%%%%%%%%%%%%%%%%%%%%%%%%%%%%%%%%%%%%%%%
\documentclass[12pt]{article} %report allows for chapters
%%%%%%%%%%%%%%%%%%%%%%%%%%%%%%%%%%%%%%%%%%%%%%%%%%%%%%%%%%%%%%%%%%%%%%%%%%%%%%%%%%%%
\usepackage{preamble}

\begin{document}

\begin{center}
   \textsc{\large MATH 271, Worksheet 4, \emph{Solutions}}\\
   \textsc{Second order linear equations and boundary value problems.}
\end{center}
\vspace{.5cm}

\begin{problem}
Write down the characteristic polynomial for the following equations.  Then, find the roots to the characteristic polynomial and write down the general solution.
\begin{enumerate}[(a)]
    \item $x''+x'+x=0$.
    \item $x''-x'-x=0$.
    \item $x''-x'+x=0$.
    \item $x''+x'-x=0$.
\end{enumerate}
\end{problem}
\begin{solution}
    \begin{enumerate}[(a)]~
        \item The characteristic polynomial is 
         \[
\lambda^2+\lambda+1=0.
\]
 This has roots
        \[
            \lambda_1 = -\frac{1}{2} -i\frac{\sqrt{3}}{2} \qquad \textrm{and} \qquad \lambda_2 = -\frac{1}{2} + i\frac{\sqrt{3}}{2}.
        \]
    This leads to the general solution
        \[
            \boxed{x(t) = e^{-\frac{1}{2}t} \left(c_1 \sin\left(\frac{\sqrt{3}}{2}t\right) +c_2 \cos\left(\frac{\sqrt{3}}{2}t\right) \right).}
        \]

        \item The characteristic polynomial is 
        \[
            \lambda^2-\lambda-1=0.
        \]
         This has roots
        \[
            \lambda_1 = \frac{1}{2} -\frac{\sqrt{5}}{2} \approx -0.618 \qquad \textrm{and} \qquad \lambda_2 = \frac{1}{2} + \frac{\sqrt{5}}{2} \approx 1.618.
        \]
    This leads to the general solution
        \[
            \boxed{x(t) = c_1 e^{\left(\frac{1}{2}-\frac{\sqrt{5}}{2}\right)t}  +c_2 e^{\left(\frac{1}{2}+\frac{\sqrt{5}}{2}\right)t}.}
        \]

        \item The characteristic polynomial is 
         \[
\lambda^2-\lambda+1=0.
\]
 This has roots
        \[
            \lambda_1 = \frac{1}{2} -i\frac{\sqrt{3}}{2} \qquad \textrm{and} \qquad \lambda_2 = \frac{1}{2} + i\frac{\sqrt{3}}{2}.
        \]
    This leads to the general solution
        \[
            \boxed{x(t) = e^{\frac{1}{2}t} \left(c_1 \sin\left(\frac{\sqrt{3}}{2}t\right) +c_2 \cos\left(\frac{\sqrt{3}}{2}t\right) \right).}
        \]

        \item The characteristic polynomial is 
        \[
            \lambda^2+\lambda-1=0.
        \]
         This has roots
        \[
            \lambda_1 = -\frac{1}{2} -\frac{\sqrt{5}}{2} \approx -1.618 \qquad \textrm{and} \qquad \lambda_2 = -\frac{1}{2} + \frac{\sqrt{5}}{2} \approx 0.618.
        \]
    This leads to the general solution
        \[
            \boxed{x(t) = c_1 e^{\left(-\frac{1}{2}-\frac{\sqrt{5}}{2}\right)t}  +c_2 e^{\left(-\frac{1}{2}+\frac{\sqrt{5}}{2}\right)t}.}
        \]

    \end{enumerate}
\end{solution}

\newpage

\begin{problem}
For the above solutions, analyze their behavior qualitatively. That is, do the solutions oscillate, grow, decay, or some combination of these, or something else entirely?
\end{problem}
\begin{solution}
\begin{enumerate}[(a)]
    \item This solution shows decaying oscillatory behavior.
    \item This solution shows exponential growth with a transient decay behavior.  Notice that the term with a negative in the exponential will seem to disappear over time.  We refer to this as transient.
    \item This solution shows exponential growth and oscillatory behavior.
    \item This solution is analogous to the one in (b) just with slightly different exponents.
\end{enumerate}
\end{solution}

\newpage

\begin{problem}
Consider the equation
\[
x''+bx'+cx=0.
\]
The roots to the characteristic polynomial are then
\[
\lambda = \frac{-b\pm \sqrt{b^2-4c}}{2}.
\]
\begin{enumerate}[(a)]
    \item Explain why if $c>0$ and $b=0$ the solution $x(t)$ will be purely oscillatory.
    \item Explain why if $b>0$ and $b^2<4c$, the solution will oscillate and decay.
    \item Explain why if $b<0$ and $b^2<4c$, the solution will oscillate and grow.
\end{enumerate}
\end{problem}
\begin{solution}~
    \begin{enumerate}[(a)]
        \item If $b=0$, then the real part of the roots $\lambda_1$ and $\lambda_2$ will be zero. Thus, there will be no growth or decay.  If $c>0$ as well, then we will have a square root of a negative appear, and we will get a nonzero imaginary part for the roots. Thus, we will have oscillation.
        \item If $b>0$, then the real part of the roots will be negative and we will see decay.  If $b^2<4c$, then we have a square root of a negative appearing which gives a nonzero imaginary part to the roots and we will have oscillation.
        \item This answer is the same as (b) except for if $b<0$ we will see growth instead of decay since the real part of the roots will be positive.
    \end{enumerate}
\end{solution}

\newpage

\begin{problem}
Write down a second order linear differential equation that oscillates and also decays over time.
\end{problem}
\begin{solution}
Since our solution should oscillate and decay, we need some form of a ``spring" and some form of damping.  These terms show up respectively as $b$ and $c$ in the equation
\[
x''+bx'+cx=0.
\]
Now, also note that (aside from one special case of two of the same real roots), our general solution has the form
\[
x(t)=C_1 e^{\lambda_1 t}+C_2e^{\lambda_2 t}
\]
where $\lambda_1$ and $\lambda_2$ are roots to the characteristic polynomial
\[
\lambda^2+b\lambda + c =0.
\]
Now, the roots for the characteristic polynomial are
\[
\lambda = \frac{-b \pm \sqrt{b^2-4c}}{2}.
\]
\begin{itemize}
    \item To have oscillation, our roots must have an imaginary part and thus 
    \[
    b^2-4c<0.
    \]
    In other words, $b^2<4c.$
    \item To have a decaying solution, the real part of the roots must be negative. The real part of the roots will be $\frac{-b}{2}$ and thus we need
    \[
    \frac{-b}{2}<0.
    \]
\end{itemize}
Now, I'll choose $b=1$ and $c=1$ which satisfy both of these requirements. We then have
\[
x''+x'+x=0
\]
as our equation.

Note, we can also find the solution as the roots are then
\[
\lambda = \frac{-1\pm \sqrt{1-4}}{2}=\frac{-1}{2}\pm \frac{\sqrt{3}}{2}.
\]
Plugging this into the form for the general solution and we get
\[
x(t)=e^{-\frac{1}{2}t}\left(C_1 \sin\left(\frac{\sqrt{3}}{2}\right) + \cos\left(\frac{\sqrt{3}}{2}\right)\right)
\]
\end{solution}

\newpage

\begin{problem}
Consider the following differential equation
\[
x''+x=0.
\]
\begin{enumerate}[(a)]
    \item Find the general solution to this equation.
    \item Given the initial conditions $x(0)=1$ and $x'(0)=1$, find the particular solution.
    \item Plot your particular solution.
    \item Does the solution grow or decay over time? 
    \item What is $\lim_{t\to \infty}x(t)$?
\end{enumerate}
\end{problem}
\begin{solution}~
\begin{enumerate}[(a)]
    \item To find a general solution to this equation we can write down the characteristic polynomial
    \[
    \lambda^2+1
    \]
    and find the roots
    \begin{align*}
        \lambda^2+1&=0\\
        \lambda^2&=-1\\
        \lambda&=\pm i.
    \end{align*}
    Thus the general solution is
    \[
    x(t)=C_1e^{\lambda_1 t}+C_2e^{\lambda_2 t}=C_1e^{it}+C_2e^{-it},
    \]
    where $C_1$ and $C_2$ are complex numbers. We could also equivalently write
    \[
    x(t)=C_1\sin(t)+C_2\cos(t).
    \]
    \item This solution oscillates with the same amplitude for all time. So it does not grow or decay!
    \item ``If the limit never approaches anything... The limit does not exist. The limit does not exist!"  Who doesn't love a Mean Girls quote.
\end{enumerate}
\end{solution}

\newpage

\begin{problem}
Next, consider a related equation
\[
x''+x=t.
\]
that has an additional linear external force.
\begin{enumerate}[(a)]
    \item What is the solution to the homogenous equation?
    \item Find the particular integral with the given forcing term.
    \item What is the specific solution to this equation?
    \item Does the solution grow or decay over time?
    \item What is $\lim_{t\to \infty}x(t)$?
\end{enumerate}
\end{problem}
\begin{solution}
\begin{enumerate}[(a)]
    \item We found the homogeneous solution in the previous problem. We have
    \[
    x_h = C_1\sin(t)+C_2\cos(t).
    \]
    \item With this forcing term we would take 
    \[
    x_p = a_0 + a_1t.
    \]
    \item We need to find the undetermined coefficients $a_0$ and $a_1$.  So we plug in $x_p$ into our differential equation
    \begin{align*}
        x_p''+x_p&=t\\
        a_0+a_1t&=t
    \end{align*}
    so $a_0=0$ and $a_1=1$. So the specific solution to this problem is
    \[
    \boxed{x=x_h+x_p=C_1\sin(t)+C_2\cos(t)+t.}
    \]
    \item This solution grows over time since the $t$ term in our solution $x$ dominates the oscillating terms.
    \item Part (d) can be equivalently stated in this way.  We consider $\lim_{t\to \infty} t = \infty$, which shows us that the solutions grows over time.
\end{enumerate}
\end{solution}

\newpage

\begin{problem}
Consider now the equation
\[
x''+x=F(t)
\]
where the external force is $F(t)=\cos(t)$.
\begin{enumerate}[(a)]
    \item Find the particular integral with the given forcing term.
    \item What is the specific solution to this equation?
    \item What is $\lim_{t\to \infty}$? What does this mean about the growth or decay of the solution over time?
\end{enumerate}
\end{problem}
\begin{solution}
\begin{enumerate}[(a)]
    \item With this forcing term we try 
    \[
    x_p = K \cos(t)+M\sin(t).
    \]
    \item If we try to find the specific solution with this ansatz above, there will be an issue.  Let's see what happens.  
    \begin{align*}
        x_p''+x_p &= \cos(t)\\
        (-K\cos(t)-M\sin(t))+(K\cos(t)+M\sin(t)&=\cos(t)\\
        0&=\cos(t).
    \end{align*}
    So this ansatz is not correct.  It turns out we must consider an ansatz of
    \[
    x_p=Kt\cos(t)+Mt\sin(t).
    \]
    In general, whenever the ansatz doesn't work, be can add another term that has an extra power of $t$ on it.
    \item This solution will in fact grow over time.  What we get is an ever increasing amplitude of oscillation. This specific case is called \emph{resonance} since we are forcing the system at its fundamental frequency.
\end{enumerate}
\end{solution}

\newpage

\begin{problem}
Consider the boundary value problem
\[
x''=g
\]
with boundary values $x(0)=0$, $x\left(-\frac{2}{g}\right)=0$ and $g=-9.8[m/s^2]$.  We can think of this as solving the \emph{inverse problem} of one that we have seen in a homework. Specifically, think of this as knowing where a ball is launched and knowing where it lands and trying to find the speed it must have been thrown at.  

Another interpretation is the shape of a rod bending due to gravity.  $x''$ would measure the curvature of this rod, and this equation would say that the rod under the force of gravity would have a constant curvature. In this case, the dependent variable $t$ should be thought of as spatial rather than temporal.

Finally, this equation above is referred to as\emph{Poisson's equation}.
\begin{enumerate}[(a)]
    \item Find the general solution. If you already know it from the homework, just write it down.
    \item Use the boundary values above to find the particular solution.
    \item Is the solution unique? 
\end{enumerate}
\end{problem}
\begin{solution}
\begin{enumerate}[(a)]
    \item The general solution is found by integrating twice.  It has been done in the homework, so I'll just say that we have
    \[
    x=\frac{1}{2}gt^2+C_1t+C_2.
    \]
    \item We plug in the boundary values to get
    \begin{align*}
        0=x(0)&=\frac{1}{2}g\cdot 0 + C_1 \cdot 0 + C_2 = C_2
    \end{align*}
    and so $C_2=0$. Then we also have
    \begin{align*}
        0=x\left(-\frac{2}{g}\right)&= \frac{1}{2}g\cdot \left(\frac{-2}{g}\right)^2+C_1 \cdot \frac{-2}{g}= \frac{2}{g}+\frac{-2}{g}C_1\\
        \iff -\frac{2}{g}&=-\frac{2}{g}C_1
    \end{align*}
    so $C_1=1$. Thus our particular solution is
    \[
    x=\frac{1}{2}gt^2 + t.
    \]
    \item Yes, the solution is unique.  We did not find any other option it could be.  However, we could ask related questions that sometimes don't have unique answers!
\end{enumerate}
\end{solution}

\newpage

\begin{problem}
Consider the \emph{time independent Sch\"odinger equation} for a \emph{free particle} constrained inside of a 1-dimensional box of length $L$. That is, we have the equation
\[
-\frac{\hbar^2}{2m}\frac{d^2}{dx^2}\psi(x)=E\psi(x)
\]
on the unit interval $[0,L]$.
\begin{enumerate}[(a)]
    \item Find the general solution to this equation with no constraint.
    \item Given the constraint, we have the boundary values $\psi(0)=\psi(L)=0$. What are the general solutions given this constraint?
    \item Show that the sum of two solution $\psi_1(x)$ and $\psi_2(x)$ is also a solution. When we have a particle whose state (or \emph{wavefunction}) $\psi$ is a sum of general solutions, we say that $\psi$ is in a \emph{superposition state.}
    \item The wavefunction is not really a physically meaningful quantity.  However, if we consider a region $[a,b]$ in the box $[0,L]$ the quantity
    \[
    P([a,b])=\int_a^b |\psi(x)|^2dx
    \]
    \emph{is} meaningful. This expression tells us the \emph{probability} that a particle will be observed in the region $[a,b]$.  Take your general solutions you found in (b) (with the constraint) and solve for the constants that give you
    \[
    \int_0^L |\psi(x)|^2dx=1.
    \]
    We call this \emph{normalization} and we must do so for each state so that we can interpret the integral $P([a,b])$ as a probability.
\end{enumerate}
\end{problem}
\begin{solution}
\begin{enumerate}[(a)]
    \item We have a second order linear differential equation with constant coefficients. In fact, it is also homogeneous as we can write 
\[
\frac{d^2\psi}{dx^2}+\frac{2mE}{\hbar^2}\psi =0.
\]
Now, to solve this, find roots $\lambda_1$ and $\lambda_2$ to the characteristic polynomial
\[
\lambda^2+\frac{2mE}{\hbar^2}=0.
\]
We solve this by letting $\omega^2 = \frac{2mE}{\hbar^2}$ and putting
\begin{align*}
    \lambda^2&=-\omega\\
    \lambda&=\pm i \omega,
\end{align*}
so $\lambda_1=i\omega$ and $\lambda_2=\lambda_1^*$. This then gives us the general solution
\[
\psi(x)=C_1 e^{i\omega x}+C_2 e^{-i\omega x}.
\]
Of course, it is also possible to write 
\[
\psi(x)=C_1\cos(\omega x)+C_2\sin(\omega x),
\]
as this is just an equivalent way to write out the general solution. 
\item Now, we have our boundary conditions $\psi(0)=0$ and $\psi(L)=0$ as well.  Plugging these into our general solution gives us
\begin{align*}
    0=\psi(0)&=C_1 \cos(\omega \cdot 0)+C_2 \sin(\omega \cdot 0)\\
    &= C_1,
\end{align*}
so $C_1=0$.  Next, we have
\begin{align*}
    0=\psi(L)&=C_2\sin(\omega \cdot L).
\end{align*}
Now, how are we to solve this equation? We must have that input to the sin function must be an integer $n=\dots,-2,-1,0,1,2,\dots$ copy of $\pi$ as $\sin(n\pi)=0$. Else, we force $C_2=0$ which gives us nothing!  So, we require
\[
\omega L = n\pi.
\]
Recall that $\omega = \frac{2mE}{\hbar^2}$ and that $E$ is not determined (yet)! So now we have that $\omega = \frac{n\pi}{L}$ which gives us a general solution we will denote with a subscript $n$
\[
C\psi_n(x)=\sin\left(\frac{n\pi x}{L}\right).
\]
\item Let's consider two solutions $\psi_n=C_n\sin\left(\frac{n\pi x}{L}\right)$ and $\psi_m=C_m\sin\left(\frac{m\pi x}{L}\right)$ and the sum of solutions
\[
\Psi=\psi_n + \psi_m.
\]
Then we can plug these into the equation
\begin{align*}
    \frac{d^2}{dx^2}\Psi + \omega^2 \Psi&=\frac{d^2}{dx^2}(\psi_n + \psi_m)+\omega^2(\psi_n+\psi_m)\\ 
    &= \psi_n'' + \psi_m''  + \omega^2 \psi_n + \omega^2 \psi_m\\
    &= -\omega^2 \psi_n -\omega^2\psi_m+\omega^2\psi_n + \omega^2\psi_m\\
    &=0.
\end{align*}
Indeed, the sum of two solutions is a solution.
\item We can integrate
\begin{align*}
1=\int_0^L \left|C_n\sin(\left(\frac{n\pi x}{L}\right)\right|^2dx&= |C_n|^2 \int_0^L \sin^2\left(\frac{n \pi x}{L}\right)\\
&= C_n \frac{L}{2}
\end{align*}
which means that $C_n=\sqrt{\frac{2}{L}}$. In fact, this is true for all $n$.
\end{enumerate}
\end{solution}






\end{document}

%%%%%%%%%%%%%%%%%%%%%%%%%%%%%%%%%%%%%%%%%%%%%%%%%%%%%%%%%%%%%%%%%%%%%%%%%%%%%%%%%%%%
% Document data
%%%%%%%%%%%%%%%%%%%%%%%%%%%%%%%%%%%%%%%%%%%%%%%%%%%%%%%%%%%%%%%%%%%%%%%%%%%%%%%%%%%%
\documentclass[12pt]{article} %report allows for chapters
%%%%%%%%%%%%%%%%%%%%%%%%%%%%%%%%%%%%%%%%%%%%%%%%%%%%%%%%%%%%%%%%%%%%%%%%%%%%%%%%%%%%
\usepackage{preamble}

\begin{document}

\begin{center}
   \textsc{\large MATH 271, Homework 7}\\
   \textsc{Due November 1$^\textrm{st}$}
\end{center}
\vspace{.5cm}

\begin{problem}
Let $S$ be the set of general solutions $x(t)$ to the following homogeneous linear differential equation 
\[
x''+f(t)x'+g(t)x=0.
\]
Show that this set $S$ is a vector space over the complex numbers by doing the following. Let $x(t),y(t) \in S$ be solutions to the above equation and let $\alpha, \beta \in \C$ be complex scalars.
\begin{enumerate}[(a)]
    \item Write down the eight requirements for $S$ to be a vector space.  
    \item Identify the $\zerovec \in S$ and $1\in \C$. 
    \item Show that $\alpha x(t) + \beta y(t) \in S$. That is, show that a superposition of solutions is also a solution. \emph{Hint: We have shown this before.}
\end{enumerate}
\end{problem}

\begin{problem}
Consider the following vectors in the real plane $\R^2$. We let
\[
\vecu = 1\xhat + 2\yhat \qquad \textrm{and} \qquad \vecv = -3\xhat+ 3\yhat.
\]
\begin{enumerate}[(a)]
    \item Draw both $\vecu$ and $\vecv$ in the plane and label the origin.
    \item Draw the vector $\vecw = \vecu+\vecv$ in the plane.
    \item Find the area of the parallelogram generated by $\vecu$ and $\vecv$.
\end{enumerate}
\end{problem}

\begin{problem}~
\begin{enumerate}[(a)]
    \item We can reflect a vector in the plane by first reflecting basis vectors. Let $R\colon \R^2 \to \R^2$ be a function be defined by 
    \[
    R(\xhat) = -\xhat \qquad \textrm{and} \qquad R(\yhat)=\yhat.
    \]
    Let $\vecv = \alpha_1 \xhat + \alpha_2 \yhat$ and define
    \[
    R(\vecv) = \alpha_1 R(\xhat) + \alpha_2 R(\yhat).
    \]
    When this is the case, we call the function $T$ \underline{linear}.\\
    \noindent Show that $R$ reflects the vector $\vecu = 1\xhat + 2\yhat$ about the $y$-axis and draw a picture.
    \item We can rotate a vector in the plane by first rotating the basis vectors $\xhat$ and $\yhat$. Define a \underline{linear} function $T\colon \R^2 \to \R^2$ defined by
    \[
    T(\xhat)=\yhat \qquad \textrm{and} \qquad T(\yhat)=-\xhat.
    \]
    \noindent Show that $T$ rotates $\vecu$ by $\pi/2$ in the counterclockwise direction and draw a picture.
\end{enumerate}
\end{problem}

\begin{problem}
Consider the following vectors in space $\R^3$
\[
\vecu = 1\xhat + 2\yhat + 3\zhat \qquad \textrm{and} \qquad \vecv = -2\xhat +1\yhat -2\zhat.
\]
\begin{enumerate}[(a)]
    \item Compute the dot product $\vecu\cdot \vecv$. 
    \item Compute the cross product $\vecu \times \vecv$.
    \item \textbf{(Experimental)} Let us try this: Take
    \begin{align*}
        \vecu \vecv &= (1\xhat + 2\yhat + 3\zhat)(-2\xhat + 1\yhat -2\zhat).
    \end{align*}
    Distribute the above multiplication.
    \item \textbf{(Experimental)} Now, in the above multiplication, adopt the following rules:
    \begin{align*}
        \xhat \xhat = \yhat \yhat = \zhat \zhat &= 1 &&&
        \xhat \yhat &= - \yhat \xhat &&& \xhat \zhat &= - \zhat \xhat &&& \yhat \zhat &= -\zhat \yhat.
    \end{align*}
    Then, simplify the multiplication in part (c) to
    \[
    \vecu\vecv = \alpha + \beta_1 \yhat\zhat  \ + \beta_2 \zhat \xhat + \beta_3 \xhat \yhat. 
    \]
    That is, what are $\alpha$, $\beta_1$, $\beta_2$, and $\beta_3$?
    \item \textbf{(Experimental)} If we perform one more step, we will notice something quite nice.  Note that the pairs of vectors above define a plane, and there is a unique vector perpendicular to that plane.  Using this fact, we can let 
    \[
    \yhat \zhat = \xhat \qquad \zhat \xhat = \yhat \qquad \xhat \yhat = \zhat.
    \]
    In other words, we can replace the two vectors above with their cross product, (i.e., $\yhat \zhat = \yhat \times \zhat = \xhat$.) Show that with these rules
    \[
    \vecu \vecv = \vecu\cdot \vecv + \vecu \times \vecv.
    \]
\end{enumerate}
\end{problem}

\begin{problem}
Consider the same vectors $\vecu,\vecv\in \R^3$ from Problem 4.  
\begin{enumerate}[(a)]
    \item Compute the lengths $\|\vecu\|$ and $\|\vecv\|$ using the dot product.
    \item Compute the angle between vectors $\vecu$ and $\vecv$. \emph{Hint: Save some work and use results from Problem 4.}
    \item Compute the projection of $\vecu$ in the direction of $\vecv$. \emph{Hint: Again, save yourself some time and use results from Problem 4.}
\end{enumerate}
\end{problem}
\end{document}
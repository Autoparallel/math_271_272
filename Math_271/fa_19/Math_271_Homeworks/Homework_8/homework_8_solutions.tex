%%%%%%%%%%%%%%%%%%%%%%%%%%%%%%%%%%%%%%%%%%%%%%%%%%%%%%%%%%%%%%%%%%%%%%%%%%%%%%%%%%%%
% Document data
%%%%%%%%%%%%%%%%%%%%%%%%%%%%%%%%%%%%%%%%%%%%%%%%%%%%%%%%%%%%%%%%%%%%%%%%%%%%%%%%%%%%
\documentclass[12pt]{article} %report allows for chapters
%%%%%%%%%%%%%%%%%%%%%%%%%%%%%%%%%%%%%%%%%%%%%%%%%%%%%%%%%%%%%%%%%%%%%%%%%%%%%%%%%%%%
\usepackage{preamble}

\begin{document}

\begin{center}
   \textsc{\large MATH 271, Homework 8, \emph{Solutions}}\\
   \textsc{Due November 8$^\textrm{th}$}
\end{center}
\vspace{.5cm}

\begin{problem}
Let a mass $m_1$ weighing $1kg.$ be placed at $\vecr_1=2\xhat -3 \yhat -\zhat$ and a mass $m_2$ of $2kg.$ be placed at $\vecr_2 = 4\yhat -2\zhat$.  Where must a mass $m_3$ of $3kg.$ be placed so that the center of mass is at the origin $\zerovec$?
\end{problem}
\begin{solution}
One can compute the center of mass $\boldsymbol{\vec{R}}_{CM}$ by
\[
\boldsymbol{\vec{R}}_{CM} = \frac{m_1\vecr_1 + m_2\vecr_2 + m_3 \vecr_3}{m_1 + m_2 + m_3}.
\]
Here, we know everything but $\vecr_3$.  Since we want the center of mass at the origin $\zerovec$, then
\begin{align*}
\zerovec &= \frac{1}{m_1+m_2 + m_3} \left( m_1 \vecr_1 + m_2 \vecr_2 + m_3 \vecr_3 \right)\\
\begin{pmatrix} 0 \\ 0 \\ 0\end{pmatrix}&= \frac{1}{6} \left( \begin{pmatrix} 2 \\ -3 \\ -1 \end{pmatrix} + 2 \begin{pmatrix} 0 \\ 4 \\ -2 \end{pmatrix} + 3 \begin{pmatrix} x \\ y \\ z \end{pmatrix}\right).
\end{align*}
What we have above is three equations and three unknowns. That is, one equation for the $\xhat$-component, one for the $\yhat$-component, and one for the $\zhat$-component. We have
\begin{align*}
    0 & = \frac{1}{6} (2 + 2\cdot 0 + 3x)\\
    0 & = \frac{1}{6} (-3 + 2\cdot 4 + 3y)\\
    0 & = \frac{1}{6} (-1 +2\cdot (-2) + 3z).
\end{align*}
Taking the first, we find
\begin{align*}
    0&= \frac{1}{3} +\frac{1}{2}x\\
    -\frac{1}{3}&= \frac{1}{2}x\\
    \implies~ x&= -\frac{2}{3}.
\end{align*}
Next,
\begin{align*}
    0&= -\frac{1}{2} +\frac{4}{3}+\frac{1}{2}y\\
    -\frac{5}{6}&= \frac{1}{2}y\\
    \implies~ y&= -\frac{5}{3}.
\end{align*}
Lastly, we have
\begin{align*}
    0&= -\frac{1}{6}-\frac{2}{3} +\frac{1}{2}z\\
    \frac{5}{6} &= \frac{1}{2}z\\
    \implies~ z&= \frac{5}{3}.
\end{align*}
Thus we have that $\vecr_3 = -\frac{2}{3} \xhat - \frac{5}{3}\yhat + \frac{5}{3}\zhat.$
\end{solution}

\newpage
\begin{problem}
Which of the following are linear transformations? For those that are not, which properties of \emph{linearity} (the properties (i) and (ii) in our notes) fail? Show your work.
\begin{enumerate}[(a)]
    \item $T_a \colon \R \to \R$ given by $T_a(x)=\frac{1}{x}$.
    \item $T_b \colon \R^3 \to \R^2$ given by
    \[
    T_b \begin{pmatrix} x\\ y\\ z \end{pmatrix}
    = \begin{pmatrix} x\\ y \end{pmatrix}.
    \]
    \item $T_c \colon \R \to \R^3$ given by
    \[
    T_c(t)=\begin{pmatrix} t\\ t^2\\ t^3 \end{pmatrix}.
    \]
    \item $T_d \colon \R^2 \to \R^3$ given by
    \[
    T_d \begin{pmatrix} x\\ y \end{pmatrix}
    = \begin{pmatrix} x+y\\ x+y\\ x+y \end{pmatrix}.
    \]
\end{enumerate}
\end{problem}
\begin{solution}~
\begin{enumerate}[(a)]
    \item This transformation fails both properties.  For (i), take
    \[
    T_a(x+y) = \frac{1}{x+y} \neq \frac{1}{x}+\frac{1}{y} = T_a(x)+T_a(y).
    \]
    For (ii), take
    \[
    T_a(\alpha x) = \frac{1}{\alpha x} \neq \alpha \frac{1}{x} = \alpha T_a(x).
    \]
    \item This is a linear transformation.  To see (i) holds, take
    \begin{align*}
        T_b(\vecu +\vecv)&= T_b \left( \begin{pmatrix} u_x \\ u_y \\ u_z \end{pmatrix} + \begin{pmatrix} v_x \\ v_y \\ v_z \end{pmatrix} \right)\\
        &=T_b \begin{pmatrix} u_x + v_x \\ u_y + v_y \\ u_z + v_z \end{pmatrix}\\
        &= \begin{pmatrix} u_x + v_x \\ u_y + v_y \end{pmatrix}\\
        &= \begin{pmatrix} u_x \\ u_y \end{pmatrix} + \begin{pmatrix} v_x \\ v_y \end{pmatrix}\\
        &= T_b(\vecu) + T_b(\vecv).
    \end{align*}
    And for (ii), we take
    \begin{align*}
        T_b(\alpha \vecv) &= T_b \left( \alpha \begin{pmatrix} v_x \\ v_y \\ v_z \end{pmatrix}\right)\\
        &= T_b \begin{pmatrix} \alpha v_x \\ \alpha v_y \\ \alpha v_z \end{pmatrix}\\
        &= \begin{pmatrix} \alpha v_x \\ \alpha v_y \end{pmatrix}\\
        &= \alpha \begin{pmatrix} v_x \\ v_y \end{pmatrix}\\
        &= \alpha T_b(\vecv).
    \end{align*}
    \item This is not a linear transformation as both properties fail. Indeed, for (i) we take
    \begin{align*}
        T_c(\vecu +\vecv) &= T_c \left( \begin{pmatrix} u_x \\ u_y \\ u_z \end{pmatrix} + \begin{pmatrix} v_x \\ v_y \\ v_z \end{pmatrix} \right)\\
        &= T_c \begin{pmatrix} u_x + v_x \\ u_y + v_y \\ u_z + v_z \end{pmatrix}\\
        &= \begin{pmatrix} u_x + v_x \\ (u_y + v_y)^2 \\ (u_z + v_z)^3 \end{pmatrix},
    \end{align*}
    whereas 
    \begin{align*}
    T_c (\vecu)+T_c(\vecv) &= T_c \begin{pmatrix} u_x \\ u_y \\ u_z \end{pmatrix} +T_c\begin{pmatrix} v_x \\ v_y \\ v_z \end{pmatrix}\\
    &= \begin{pmatrix} u_x \\ u_y^2 \\ u_z^3 \end{pmatrix} + \begin{pmatrix} v_x \\ v_y^2 \\ v_y^3 \end{pmatrix}\\
    &= \begin{pmatrix} u_x + v_x \\ u_y^2 + v_y^2 \\ u_z^3 + v_z^3 \end{pmatrix}.
    \end{align*}
    Note that $u_y^2+v_y^2\neq (u_y+v_y)^2$ and $u_z^3+v_z^3\neq (u_z+v_z)^3$. 
    
    To see that (ii) does not hold, take 
    \begin{align*}
        T_c(\alpha \vecv ) &= T_c\begin{pmatrix} \alpha v_x \\ \alpha v_y \\ \alpha v_z \end{pmatrix}\\
        &= \begin{pmatrix} \alpha v_x \\ \alpha^2 v_y^2 \\ \alpha^3 v_z^3\end{pmatrix},
    \end{align*}
    whereas
    \begin{align*}
        \alpha T_c(\vecv)&= \begin{pmatrix} \alpha v_x \\ \alpha v_y^2 \\ \alpha v_z^3\end{pmatrix}.
    \end{align*}
    These are clearly not equal for every scalar $\alpha$.
    \item This function is linear. For (i), we have
    \begin{align*}
        T_d(\vecu + \vecv) &= T_d \begin{pmatrix} u_x + v_x \\ u_y + v_y \end{pmatrix}\\
        &=\begin{pmatrix} (u_x+v_x)+(u_y+v_y) \\ (u_x+v_x)+(u_y+v_y) \\ (u_x+v_x) + (u_y +v_y) \end{pmatrix}\\
        &= \begin{pmatrix} u_x + u_y \\ u_x + u_y \\ u_x + u_y \end{pmatrix} + \begin{pmatrix} v_x + v_y \\ v_x + v_y \\ v_x + v_y \end{pmatrix}\\
        &= T(\vecu) + T(\vecv).
    \end{align*}
    And for (ii) we have
    \begin{align*}
        T_d(\alpha \vecv) &= T_d \begin{pmatrix} \alpha v_x \\ \alpha v_y \end{pmatrix}\\
        &= \begin{pmatrix} \alpha v_x + \alpha v_y \\ \alpha v_x + \alpha v_y \\ \alpha v_x + \alpha v_y \end{pmatrix}\\
        &= \alpha T_d(\vecv).
    \end{align*}
\end{enumerate}
\end{solution}

\newpage
\begin{problem}
Write down the matrix for the following linear transformation $T\colon \R^3 \to \R^3$:
\[
T\begin{pmatrix} x\\ y\\ z \end{pmatrix}
= \begin{pmatrix} x+y+z\\ 2x\\ 3y + z \end{pmatrix}.
\]
\end{problem}
\begin{solution}
We need that
\begin{align*}
[T] \begin{pmatrix} x \\ y \\ z \end{pmatrix} &= \begin{pmatrix} x+y+z\\ 2x\\ 3y + z \end{pmatrix}
\end{align*}
via matrix multiplication. Since the input vector is a 3-dimensional vector, and the output vector is $3$-dimensional, we must have that $[T]$ is a $3\times 3$-matrix. Hence,
\[
[T] = \begin{pmatrix} t_{11} & t_{12} & t_{13} \\ t_{21} & t_{22} & t_{23} \\ t_{31} & t_{32} & t_{33} \end{pmatrix}.
\]
Then we have
\begin{align*}
\begin{pmatrix} t_{11} & t_{12} & t_{13} \\ t_{21} & t_{22} & t_{23} \\ t_{31} & t_{32} & t_{33} \end{pmatrix} \begin{pmatrix} x \\ y \\ z \end{pmatrix} &=  \begin{pmatrix} t_{11}x + t_{12}y + t_{13}z \\ t_{21}x +  t_{22}y + t_{23}z \\ t_{31}x + t_{32}y + t_{33}z \end{pmatrix} = \begin{pmatrix} x + y + z \\ 2x \\ 3y+z \end{pmatrix}.
\end{align*}
If we match the coefficients on the $x$, $y$, and $z$, we find that
\[
[T] = \begin{pmatrix} 1 & 1 & 1 \\ 2 & 0 & 0 \\ 0 & 3 & 1 \end{pmatrix}.
\]
\end{solution}

\newpage
\begin{problem}
A linear transformation $T\colon \R^3 \to \R^3$ is given by the matrix
\[
[T]= \begin{pmatrix}
1& 2& 0\\
2& 1& 2\\
0& 2& 1
\end{pmatrix}.
\]
\begin{enumerate}[(a)]
    \item Compute how $T$ transforms the standard basis elements for $\R^3$. That is, find
    \[
    T(\xhat), \qquad
    T(\yhat), \qquad 
    T(\zhat).
    \]
    This gives a nice interpretation of matrix vector multiplication as linear combinations of the column vectors that make up a matrix.
    \item If we apply this linear transformation to the unit cube (that is, all points who have $(x,y,z)$ coordinates with $0\leq x \leq 1$, $0\leq y \leq 1$, and $0\leq z \leq 1$), what will the volume of the transformed cube be? (\emph{Hint: the determinant of this matrix $[T]$ provides us this information.})
\end{enumerate}
\end{problem}
\begin{solution}~
\begin{enumerate}[(a)]
    \item The point here is that we can understand the matrix $[T]$ and matrix multiplication better by seeing how the basis vectors are transformed. So we have
    \begin{align*}
        T(\xhat) &= \begin{pmatrix}
1& 2& 0\\
2& 1& 2\\
0& 2& 1
\end{pmatrix} \begin{pmatrix} 1 \\ 0 \\ 0 \end{pmatrix}\\
&= \begin{pmatrix} 1 \\ 2 \\ 0 \end{pmatrix},
    \end{align*}
    which is just the first column of the matrix. Then we have
    \begin{align*}
                T(\yhat) &= \begin{pmatrix}
1& 2& 0\\
2& 1& 2\\
0& 2& 1
\end{pmatrix} \begin{pmatrix} 0 \\ 1 \\ 0 \end{pmatrix}\\
&= \begin{pmatrix} 2 \\ 1 \\ 2 \end{pmatrix},
    \end{align*}
    which is just the second column of the matrix. Lastly we have
    \begin{align*}
                T(\zhat) &= \begin{pmatrix}
1& 2& 0\\
2& 1& 2\\
0& 2& 1
\end{pmatrix} \begin{pmatrix} 0 \\ 0 \\ 1 \end{pmatrix}\\
&= \begin{pmatrix} 0 \\ 2 \\ 1 \end{pmatrix},
    \end{align*}
    which is the last column of the matrix.
    \item The three basis vectors 
    \[
    \xhat = \begin{pmatrix} 1 \\ 0 \\ 0 \end{pmatrix}, \qquad \yhat = \begin{pmatrix} 0 \\ 1 \\ 0 \end{pmatrix}, \qquad \zhat =\begin{pmatrix} 0 \\ 0 \\ 1 \end{pmatrix}
    \]
    define the volume of the unit cube. That is, the parallelepiped generated by $\xhat$, $\yhat$, and $\zhat$ is the unit cube. Hence, if we know how these vectors are transformed, we just need to find the volume of the paralellepiped given by the transformed vectors $T(\xhat)$, $T(\yhat)$, and $T(\zhat)$.  Now, we can collect these vectors into a matrix,
    \begin{align*}
    \begin{pmatrix} \vert & \vert & \vert \\ T(\xhat) & T(\yhat) & T(\zhat) \\ \vert & \vert & \vert \end{pmatrix} &= \begin{pmatrix}  1& 2& 0\\
2& 1& 2\\
0& 2& 1\end{pmatrix},
    \end{align*}
    which is exactly $[T]$! This is what we realized in part (a)! Now, the determinant of the matrix gives us the signed volume of the parallelepiped generated by the three column vectors, and hence
    \[
    \mathrm{Area}=|\det([T])|=|-7|=7.
    \]
\end{enumerate}
\end{solution}

\newpage
\begin{problem}
What does a zero determinant indjicate about the solutions of a non-homogeneous system of linear equations? (Think geometrically!)
\end{problem}

\newpage
\begin{problem}
What does a zero determinant indicate about the solutions of a homogeneous system of linear equations? (Think geometrically!)
\end{problem}

\newpage
\begin{problem}
Solve the following equation.
\[
\begin{pmatrix} 1 & 1 & 1 \\ 1 & 2 & 1 \\ 1 & 2 & 2 \end{pmatrix} \begin{pmatrix} x \\ y \\ z \end{pmatrix} = \begin{pmatrix} 6 \\ 8 \\ 11 \end{pmatrix}.
\]
\end{problem}



\end{document}
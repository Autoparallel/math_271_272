%%%%%%%%%%%%%%%%%%%%%%%%%%%%%%%%%%%%%%%%%%%%%%%%%%%%%%%%%%%%%%%%%%%%%%%%%%%%%%%%%%%%
% Document data
%%%%%%%%%%%%%%%%%%%%%%%%%%%%%%%%%%%%%%%%%%%%%%%%%%%%%%%%%%%%%%%%%%%%%%%%%%%%%%%%%%%%
\documentclass[12pt]{article} %report allows for chapters
%%%%%%%%%%%%%%%%%%%%%%%%%%%%%%%%%%%%%%%%%%%%%%%%%%%%%%%%%%%%%%%%%%%%%%%%%%%%%%%%%%%%
\usepackage{preamble}

\begin{document}

\begin{center}
   \textsc{\large MATH 271, Homework 5, \emph{Solutions}}\\
   \textsc{Due October 11$^\textrm{th}$}
\end{center}
\vspace{.5cm}

\begin{problem} $p$-series are actually related to a very important function called the \emph{Riemann zeta function}.  This function is involved in a million dollar math problem! If you're interested in other million dollar problems, look up the Clay Institute Millennium Problems. The Riemann zeta function is given by
\[
\zeta (s) = \sum_{n=1}^\infty \frac{1}{n^s}.
\]
\begin{enumerate}[(a)]
    \item Use the integral test to show that the $p$-series
    \[
    \sum_{n=1}^\infty \frac{1}{n^2}
    \]
    converges.  Look up what this series converges to and write it down. This is $\zeta(2)$.
    \item Use the comparison test to show that the $p$-series
    \[
    \sum_{n=1}^\infty \frac{1}{n^3}
    \]
    converges. This converges as well to $\zeta(3)$. Look up what this approximate value is.
\end{enumerate}
\end{problem}
\begin{solution}~
\begin{enumerate}[(a)]
    \item Note that $a_n = f(n)=\frac{1}{n^2}$.  Hence we can make a comparison to the integral
    \[
    \int_1^\infty f(x)dx,
    \]
    where we start at $x=1$ since our sum begins there as well.  We evaluate the integral
    \begin{align*}
        \int_1^\infty f(x)dx &= \int_1^\infty \frac{1}{x^2}dx\\
        &= \left[\frac{-1}{x}\right]_1^\infty\\
        &= \lim_{b\to \infty} \left[ \frac{-1}{x}\right]_1^b\\
        &= \lim_{b\to \infty} \frac{-1}{b} - \frac{-1}{1}\\
        &= 1.
    \end{align*}
    So since the integral is finite the series converges.  \emph{Warning: the integral does \underline{not} tell us what the series converges to! The series in fact converges to $\zeta(2)=\frac{\pi^2}{6}$.}
    \item Note that for $N\geq 2$ we have that $b_n=\frac{1}{n^3}\leq \frac{1}{n^2}=a_n$. Thus, since we have this inequality and we found $\sum_{n=1}^\infty a_n$ converges, we also have that $\sum_{n=1}^\infty b_n$ converges as well. \emph{Warning: again, this comparison test does \underline{not} tell us what the series converges to. If we look it up we find that $\zeta(3)\approx 1.20206$.}
\end{enumerate}
\end{solution}

\newpage
\begin{problem}
How can we approximate a (possibly complicated) function by using a power series? Why is this useful (specifically for computation on a computer)?
\end{problem}
\begin{solution}
We can often times approximate a function about a point $x=a$ using a truncated Taylor series centered about $a$. What I mean is that we will have a function $f(x)$ which (inside its interval of convergence) will be equal to
\[
f(x)=\sum_{n=0}^\infty \frac{f^{(n)}(a)}{n!}(x-a)^n
\]
This proves to be very useful as we can take finite truncations of the complete power series above. Specifically, we have that
\[
f(x)\approx \sum_{n=0}^N \frac{f^{(n)}(a)}{n!}(x-a)^n.
\]
It turns out that this is a reasonable approximation for $f$ as long as we don't look too far away from $x=a$.  Also, computers really only have the ability to add.  Of course, multiplication is sequential adding, division can be done through subtraction and multiplication, and powers come from sequential multiplication.  The point is, computers work with polynomials (or rational functions) and this gives us a way to realize a complicated non-polynomial function as (approximately) a polynomial function.
\end{solution}

\newpage
\begin{problem} Consider the function
\[
f(x)=\frac{1}{1-x}.
\]
\begin{enumerate}[(a)]
    \item Compute the Maclaurin series for the function.
    \item Find the integral $\int \frac{dx}{1-x}$ using the Maclaurin series for $f(x)$ found in (a).  
    \item Write down the Maclaurin series for $\ln(1-x)$ and compare to your answer in (b).
\end{enumerate}
\end{problem}
\begin{solution}~
\begin{enumerate}[(a)]
    \item To find the Maclaurin series (i.e., the Taylor series centered at $a=0$), we must compute $f^{(n)}(0)$ as we desire to find
    \[
    f(x) = \sum_{n=0}^\infty \frac{f^{(n)}(0)}{n!}x^n.
    \]
    The derivatives are
    \begin{align*}
        f^{(0)}(x) &= \frac{1}{1-x} &\implies&& f^{(0)}(0)&=1=0!\\
        f^{(1)}(x) &= \frac{1}{(1-x)^2} &\implies&& f^{(1)}(0)&=1=1!\\
        f^{(2)}(x) &= \frac{2}{(1-x)^3} &\implies&& f^{(2)}(0)&=2=2!\\
        f^{(3)}(x) &= \frac{6}{(1-x)^4} &\implies&& f^{(3)}(0)&=6=3!\\
        f^{(4)}(x) &= \frac{24}{(1-x)^5} &\implies&& f^{(4)}(0)&=24=4!\\
        && \vdots && &\\
        f^{(n)}(x) &= \frac{n!}{(1-x)^{n+1}} &\implies&& f^{(n)}(0)&=n!.\\
    \end{align*}
    Hence, if we plug this into the formula for the Maclaurin series we have
    \[
    f(x) = \sum_{n=0}^\infty x^n.
    \]
    \item We can integrate this series term by term to find the desired antiderivative. So we have
    \[
    \int\frac{dx}{1-x} = \int \sum_{n=0}^\infty x^n dx = C + \sum_{n=0}^\infty \frac{x^{n+1}}{n+1}.
    \]
    \item We can find the Maclaurin series for $\ln(1-x)$ in the same way as above. However, notice that $\frac{d}{dx} \ln(1-x) =\frac{-1}{1-x}$, and thus up to determining the constant $C$, we have that
    \[
    -\left(C + \sum_{n=0}^\infty \frac{x^{n+1}}{n+1}\right) = \ln(1-x).
    \]
\end{enumerate}
\end{solution}

\newpage
\begin{problem} 
Compute the Taylor series centered at $a=0$ for $f(x)=e^{-\frac{x^2}{2}}$. Then, instead use the Taylor series for $e^x$ and modify it to work for $f(x)$.  For each of these power series, plot the original function $f(x)$ compared to the four term approximation on the same graph.
\end{problem}
\begin{solution}
Again, we find the Taylor series centered at $a=0$ by computing $f^{(n)}(0)$. The derivatives are
    \begin{align*}
        f^{(0)}(x) &= e^{-\frac{x^2}{2}} &\implies&& f^{(0)}(0)&=1\\
        f^{(1)}(x) &= xe^{-\frac{x^2}{2}} &\implies&& f^{(1)}(0)&=0\\
        f^{(2)}(x) &= (x^2-1)e^{-\frac{x^2}{2}} &\implies&& f^{(2)}(0)&=-1\\
        f^{(3)}(x) &= x(x^2-3)e^{-\frac{x^2}{2}} &\implies&& f^{(3)}(0)&=0\\
        f^{(4)}(x) &= (x^4-6x^2+3)e^{-\frac{x^2}{2}} &\implies&& f^{(4)}(0)&=3.
    \end{align*}
    This gives us the first five terms of the Taylor series for $f(x)$ so that we have
    \[
    f(x)\approx 1 + 0 + \frac{-1}{2}x^2 + 0 \frac{3}{5!}x^5.
    \]
    
    The easier way is to modify an already known power series like $e^x$. We have
    \[
    e^x = \sum_{n=0}^\infty \frac{x^n}{n!}.
    \]
    If we replace $x\mapsto -\frac{x^2}{2}$ then we have
    \[
    e^{-\frac{x^2}{2}} = \sum_{n=0}^\infty \frac{ \left( -\frac{x^2}{2}\right)}{n!} = \sum_{n=0}^\infty \frac{(-1)^n x^{2n}}{2^n n!}.
    \]
    Below is a graph showing the three different functions.
    \begin{figure}[H]
        \centering
        \includegraphics[width=.8\textwidth]{desmos-graph(4).png}
        \caption{Red: $f(x)$; Green: Four terms from Taylor series; Purple: Four terms from modified Taylor series.}
        \label{fig:my_label}
    \end{figure}
    \begin{remark}
    The reason why the graphs are different is because the modified series skips the zero terms that show up in the Taylor series computation. So when we plot four terms in the modified series, it is equal to plotting the first eight terms of the actual Taylor series.
    \end{remark}
\end{solution}

\newpage
\begin{problem}
Find the radius of convergence for the following power series
\begin{enumerate}[(a)]
    \item $\displaystyle{\sum_{n=1}^\infty \frac{x^n}{n(n+1)}}$;
    \item $\displaystyle{\sum_{n=0}^\infty \frac{x^{2n+1}}{(2n+1)!}}$.
\end{enumerate}
\end{problem}
\begin{solution}~
\begin{enumerate}[(a)]
    \item To find the radius of convergence we must look at the ratio of terms in the sequence as $n\to \infty$. Specifically, we take
    \begin{align*}
        \lim_{n\to \infty} \left| \frac{\frac{x^{n+1}}{(n+1)(n+2)}}{\frac{x^n}{n(n+1)}}\right| &= \lim_{n\to \infty}\left| \frac{nx}{n+2}\right|\\
        &= |x|.
    \end{align*}
    Now, in order for this series to converge, the ratio test above must give us a limit $L<1$, and hence we must have that $|x|<1$. So the radius of convergence is 1.
    \item Similarly, we have
    \begin{align*}
        \lim_{n\to \infty} \left| \frac{\frac{x^{2n+3}}{(2n+3)!}}{\frac{x^{2n+1}}{(2n+1)!}} \right| &= \lim_{n\to \infty} \left|\frac{x^2}{(2n+3)(2n+2)}\right|\\
        &= 0.
    \end{align*}
    Hence the limit is $0<1$ and so for any value of $x$ this series converges and it follows that the radius of convergence is infinite.
\end{enumerate}
\end{solution}


\end{document}
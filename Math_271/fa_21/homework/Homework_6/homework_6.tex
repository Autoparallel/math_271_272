%%%%%%%%%%%%%%%%%%%%%%%%%%%%%%%%%%%%%%%%%%%%%%%%%%%%%%%%%%%%%%%%%%%%%%%%%%%%%%%%%%%%
% Document data
%%%%%%%%%%%%%%%%%%%%%%%%%%%%%%%%%%%%%%%%%%%%%%%%%%%%%%%%%%%%%%%%%%%%%%%%%%%%%%%%%%%%
\documentclass[12pt]{article} %report allows for chapters
%%%%%%%%%%%%%%%%%%%%%%%%%%%%%%%%%%%%%%%%%%%%%%%%%%%%%%%%%%%%%%%%%%%%%%%%%%%%%%%%%%%%
\usepackage{preamble}

\begin{document}

\begin{center}
   \textsc{\large MATH 271, Homework 6}\\
   \textsc{Due October 19$^\textrm{th}$}
\end{center}
\vspace{.5cm}

\begin{problem}
Consider the differential equation
\[
f'(x) = \frac{1}{\sqrt{1-x^2}} f(x).
\]
\begin{enumerate}[(a)]
    \item Write down the 2$^\textrm{nd}$ order Taylor approximation to $\frac{1}{\sqrt{1-x^2}}$ centered at zero.
    \item Using this second order approximation, find the general solution to the differential equation using separation.
    \item The solution you find using the approximation doesn't have an issue at $x=1$, but I claim the original equation does.  What is wrong at $x=1$? Our approximation is then only reasonable in the window $[0,1)$ (and really isn't that accurate near 1 either).
\end{enumerate}
\end{problem}

\begin{problem}
Consider the differential equation
\[
f'(x)=xf(x)
\]
with initial condition $f(0)=1$.  
\begin{enumerate}[(a)]
    \item Find the particular solution to this separable differential equation.
    \item What is the Taylor series centered at zero for this solution?
    \item Now, assume that the solution $f(x)$ can be written as a power series
    \[
    f(x) = \sum_{n=0}^\infty a_n x^n.
    \]
    Determine all of the coefficients $a_n$ which will give us the power series representation for $f(x)$. \emph{Hint: use your solution from (a) to help you.}
\end{enumerate}
\end{problem}

\begin{problem}
Consider the differential equation
\[
(x-1)f'(x) + f(x)=0
\]
with initial condition $f(0)=1$.
\begin{enumerate}[(a)]
    \item Find the solution to this equation using separation.
    \item Find the Taylor series centered at zero for your solution in (a).
    \item Again, suppose that the solution can be written as a power series and determine all the coefficients $a_n$ so that we find the power series representation for $f(x)$. \emph{Hint: use your solution from (a) to help you.}
\end{enumerate}
\end{problem}

\begin{problem}
We derived two linearly independent (even and odd) solutions to \emph{Legendre's equation}
\[
(1-x^2)f''(x)-2xf'(x)+\alpha(\alpha+1)f(x)=0
\]
which were
\[
f(x)=\sum_{n=0}^\infty a_{2n}x^{2n} \qquad \textrm{and} \qquad f(x)=\sum_{n=0}^\infty a_{2n+1}x^{2n+1}.
\]
\begin{enumerate}[(a)]
    \item Look up where this equation shows up in quantum mechanics and write it down.
    \item If we add boundary conditions then we get a finite polynomial for each choice of $\alpha = 0,1,2,3,\dots$. Using this, the first four polynomials are
    \begin{align*}
        f_0(x)&=1 &&& f_1(x)&=x\\
        f_2(x)&=1-3x^2 &&& f_3(x)&=x-\frac{5x^3}{3}.
    \end{align*}
    Show that these above polynomials are \emph{orthogonal} by showing
    \[
    \int_{-1}^1 f_i(x)f_j(x)dx = 0 
    \]
    when $i\neq j$.
\end{enumerate}
\end{problem}

\end{document}
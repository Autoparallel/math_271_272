\documentclass[12pt]{amsbook}
\usepackage{preamble}


\begin{document}
\pagenumbering{gobble}       % This kills the page numbering

\begin{center}
   \textsc{\large MATH 271, Quiz 1}\\
   \textsc{Due September 3$^\textrm{rd}$ at the end of class}
\end{center}

\vspace{1cm}

\noindent\textbf{Instructions} \; You are allowed a textbook, homework, notes, worksheets, material on our Canvas page, but no other online resources (including calculators or WolframAlpha) for this quiz.  \textbf{Do not discuss any problem any other person.} All of your solutions should be easily identifiable and supporting work must be shown.  Ambiguous or illegible answers will not be counted as correct.


\vspace*{.5cm}
\hrule
\vspace*{.5cm}

\begin{center}\textbf{\large THERE ARE 5 TOTAL PROBLEMS.}\normalsize \end{center}

\begin{problem}
\textbf{(3 pts.)} Given two complex numbers $z_1=1-i$ and $z_2 = -2+i$, draw a picture showing how to find the sum $z_1+z_2$. Also, draw a picture of $-z_1$ and $2z_1$ and explain what these scaling operations do to a complex number.
\end{problem}


\begin{problem}
\textbf{(3 pts.)} Let $z_1 = e^{i\frac{\pi}{4}}$ and $z_2=e^{-i \frac{\pi}{2}}$. Graph $z_1$, $z_2$, and the product $z_1 z_2$ in the $\mathbb{C}$-plane.  \emph{Hint: recall that there are $2\pi$ radians in a full circle.}
\end{problem}

\begin{problem}
Consider the function $x(t)=e^{it}$. 
\begin{enumerate}[(a)]
	\item \textbf{(2 pts.)} Explain why this function is periodic with period $2\pi$.
	\item \textbf{(2 pts.)} What is the real part of $x$?
	\item \textbf{(2 pts.)} Show that $x$ is a solution to the ODE: $x''=-x$. 
\end{enumerate}
\end{problem}

\begin{problem}
    \textbf{(4 pts.)} Explain what it means to be a general solution to an ODE. Explain what it means to be a particular solution to an initial value problem.  What are the key differences between general and particular solutions?
\end{problem}

\begin{problem}
	\textbf{(3 pts.)} (Hooke's law) Write down an initial value problem based on the following statement.\\
	\emph{``The rate of change of the rate of change of position of a mass is proportional to the position but in the opposite direction."}
\end{problem}

\end{document} 
%\begin{problem}
%\textbf{(4 pts.)} Given a distance $r$ from the origin and an angle $\theta$ measured counter clockwise from the positive real axis, draw a picture and explain how you can find the real and imaginary part this number using trigonometry. Then, describe how you can write this number in polar form using Euler's formula.
%\end{problem}

%\begin{problem} For the following, say whether the statement is true or false. For full credit, justify your answer with an explanation.
%\begin{enumerate}[(a)]
%    \item \textbf{(2 pts.)} The equation $0=a_0 + a_1 z^1 + a_2 z^2 + \cdots + a_n^z$ \emph{always} has $n$ solutions when $z$ is allowed to be a complex number.
%    \item \textbf{(2 pts.)} A second order ODE requires two initial conditions to have a unique particular solution.
%\end{enumerate}

%\begin{problem}
%\textbf{(3 pts.)} Write down some initial value problem that satisfies the following.
%\begin{itemize}
%    \item The dependent variable (i.e., the solution function) is given by $x(t)$.
%    \item The equation is first order.
%    \item The equation is separable.
%    \item The independent variable $t$ appears \emph{explicitly} in the ODE.
%\end{itemize}
%\end{problem}

%\begin{problem}
%\textbf{(2 pts.)} Write down an example of a second order differential equation.
%\end{problem}  


%\begin{problem}
%\textbf{(4 pts.)} Show that the function $x(t)=C_1 e^{-t}+C_2 t e^{-t}$ is a general solution to the equation
%\[
%x''+2x'+x=0.
%\]
%If you have trouble showing $x$ is a solution, explain how you would go about doing so. \emph{Hint: we don't yet know how to find solutions to equations like this, but you can still show that $x$ is a solution!}
%\end{problem} 
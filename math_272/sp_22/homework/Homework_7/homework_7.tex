%%%%%%%%%%%%%%%%%%%%%%%%%%%%%%%%%%%%%%%%%%%%%%%%%%%%%%%%%%%%%%%%%%%%%%%%%%%%%%%%%%%%
% Document data
%%%%%%%%%%%%%%%%%%%%%%%%%%%%%%%%%%%%%%%%%%%%%%%%%%%%%%%%%%%%%%%%%%%%%%%%%%%%%%%%%%%%
\documentclass[12pt]{article} %report allows for chapters
%%%%%%%%%%%%%%%%%%%%%%%%%%%%%%%%%%%%%%%%%%%%%%%%%%%%%%%%%%%%%%%%%%%%%%%%%%%%%%%%%%%%
\usepackage{preamble}
\newcommand{\grad}{\boldsymbol{\vec{\nabla}}}
\newcommand{\curvegamma}{\boldsymbol{\vec{\gamma}}}
\newcommand{\tangentgamma}{\boldsymbol{\dot{\vec{\gamma}}}}
\newcommand{\normalgamma}{\boldsymbol{\ddot{\vec{\gamma}}}}
\newcommand{\vecfieldE}{\boldsymbol{\vec{E}}}
\newcommand{\rhat}{\boldsymbol{\hat{r}}}
\newcommand{\thetahat}{\boldsymbol{\hat{\theta}}}
\newcommand{\phihat}{\boldsymbol{\hat{\phi}}}
\newcommand{\rhohat}{\boldsymbol{\hat{\rho}}}
\newcommand{\unitvec}{\boldsymbol{\hat{n}}}
\newcommand{\vecfieldB}{\boldsymbol{\vec{B}}}
\newcommand{\vecfieldJ}{\boldsymbol{\vec{J}}}
\newcommand{\vecfieldF}{\boldsymbol{\vec{F}}}
\newcommand{\vecfieldV}{\boldsymbol{\vec{V}}}
\newcommand{\vecx}{\boldsymbol{\vec{x}}}

\newcommand{\veclaplace}{\boldsymbol{\vec{\Delta}}}


\begin{document}

\begin{center}
   \textsc{\large MATH 272, Homework 7}\\
   \textsc{Due May 4$^\textrm{rd}$}
\end{center}
\vspace{.5cm}


\begin{problem}
\textbf{(BONUS 17 pts.)} Let us use the Fourier transform to deduce the diffraction pattern of a single slit. In this case, we shine a plane wave of light from a monochromatic source towards a very thin slit. The light passes through this slit and forms a diffraction pattern on the image plane which is located a distance away from the slit such that this distance is much larger than the slit itself. One receives a pattern like this:
\begin{figure}[H]
    \centering
    \includegraphics[width=.6\textwidth]{single_slit.jpg}
\end{figure}
From this exercise, you will be able to realize that a diffraction pattern is created in essence by taking the Fourier transform of the apperture the light passes through. I've taken some liberty here in dropping some details, but the key fact remains.
\begin{enumerate}[(a)]
    \item \textbf{(1 pts.)} Let us define the aperture function
    \[
    A(x) = \begin{cases} 1, & -1 \leq x \leq 1\\
    0, &\textrm{otherwise}.
    \end{cases}
    \]
    Plot this function $A(x)$ and note that this defines a slit of width 1 centered at the origin which we call the aperture. For all other $x$ values outside the range of $-1 \leq x \leq 1$, we can imagine there is a barrier that does not allow light to pass through.
    \item \textbf{(2 pts.)} Using WolframAlpha, compute the Fourier transform of $A(x)$ and call it $\hat{A}(k)$. Note that I mean
	\[
	\hat{A}(k) = \int_{-\infty}^\infty A(x) e^{ikx}dx.
	\]
    \item \textbf{(2 pts.)} The brightness of light (or, really, the electric field strength) on the image plane is given by the function $\hat{A}^2(k)$. Plot this function. %If noPlot $\hat{A}^2(k)$ and compare this with the given image.
    \item \textbf{(5 pts.)} Repeat the above with a double slit aperture that was made inside the single slit, e.g.,
    \[
    B(x) = \begin{cases} 1, & x\in \left[-1,-\frac{1}{2}\right] \cup \left[\frac{1}{2},1\right] \\
    0, &\textrm{otherwise}.
    \end{cases}
    \]
	\item \textbf{(2 pts.)} Show that the double slit pattern $\hat{B}^2(k)$ rests underneath the pattern (envelope) of the combination of the two slits by choosing the aperture
	\[
	C(x) = \begin{cases} 2, & -\frac{1}{4} \leq x \leq \frac{1}{4}\\
    0, &\textrm{otherwise}.
    \end{cases}
	\]
	That is, let the envelope be $\hat{B}^2(k)$.
	\item \textbf{(5 pts.)} Repeat the process from before but with the aperture that is inverted from $A(x)$. That is, instead of a slit, you can picture a thin object with air around it.
\end{enumerate}
\end{problem}
 
\begin{problem}
\textbf{(BONUS 22 pts.)} If we consider the wave equation on $\R$ given by
\[
\left( c^2\frac{\partial^2}{\partial x^2} -\frac{\partial^2}{\partial t^2}\right) u(x,t) = 0.
\]
Let $\square = c^2\frac{\partial^2}{\partial x^2} -\frac{\partial^2}{\partial t^2}$ be the \emph{wave operator} (or sometimes called the \emph{d'Alembertian}) then we can factor this into two expressions:
\[
\left( c\frac{\partial}{\partial x} + \frac{\partial}{\partial t} \right) u_R(x,t) =0 \qquad \textrm{and} \qquad \left( c\frac{\partial}{\partial x} - \frac{\partial}{\partial t} \right) u_L(x,t) =0
\]
where $u_R$ and $u_L$ are left and right moving wave solutions. This was how we found the d'Alembert solution to the wave equation! Let's see how I just hid the Fourier transform from you all along.

Sometimes these solutions are called \emph{plane waves} or even \emph{solitons}. Formally, you the left and right moving wave operators (that are first order) are almost Dirac operators that are used in quantum mechanics. By appealing to quaternions, you can factor the wave operator $\square$ into two equal operators (we won't do this).
\begin{enumerate}[(a)]
\item \textbf{(3 pts.)} Let us write the Fourier transform for a function of space $f(x)$ by
\[
\hat{f}(k) = \int_{-\infty}^\infty f(t) e^{i k x} dx
\]
Show that by applying this Fourier transform to the wave equation we get
\[
\frac{\partial^2}{\partial t^2}\hat{u}(k,t) = -c^2 k^2 \hat{u}(k,t).
\]
\item \textbf{(4 pts.)} Show that for any fixed value of $k$, the solution to the ODE above is
\[
\hat{u}(k,t) = \hat{F}(k)e^{-ickt} + \hat{G}(k)e^{ickt}.
\]
\item \textbf{(4 pts.)} Recall that the inverse Fourier transform is given by
\[
u(x,t) = \frac{1}{2\pi} \int_{-\infty}^\infty \hat{u}(k,t) e^{ikx} dk.
\]
Apply the inverse Fourier transform to $\hat{u}$ from before and show that
\[
u(x,t) = F(x-ct) + G(x+ct)
\]
using the fact that
\[
\delta_{x_0}(x) = \delta(x-x_0) = \int_{-\infty}^\infty  e^{ik(x-x_0)} dk.
\]

\item \textbf{(4 pts.)} Suppose that the initial conditions are $u(x,0) = h(x)$ and $\frac{\partial}{\partial t} u(x,0) = 0$. Given your solution from before $u(x,t)=F(x-ct)+G(x+ct)$, show that $F(x)=G(x)=\frac{1}{2}h(x)$ is a solution.

\item \textbf{(2 pts.)} Go back to the d'Alembert solution and explain how we stumbled upon this using the Fourier transform.

\item \textbf{(3 pts.)} Explain a bit about plane waves in higher dimensions and their Fourier transforms (namely that they are higher dimensional Dirac deltas just like ours here).

\item \textbf{(2 pts.)} Let us define a space-time Fourier transform by taking some function of space and time $f(x,t)$ and computing its Fourier transform $\hat{f}$ by 
\[
\mathcal{F}(f)(k,\omega) = \hat{f}(k,\omega) = \frac{1}{2\pi} \int_{-\infty}^\infty \int_{\infty}^\infty f(x,t) e^{i(kx-\omega t)} dxdt. 
\]
Let's use this with the wave equation and its factored version. Show that the space-time Fourier transform of the wave operator $\square$ is given by
\[
\mathcal{F}(\square) = -c^2k^2 + \omega^2.
\]
\end{enumerate}
\end{problem}



\end{document}
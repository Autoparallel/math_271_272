%%%%%%%%%%%%%%%%%%%%%%%%%%%%%%%%%%%%%%%%%%%%%%%%%%%%%%%%%%%%%%%%%%%%%%%%%%%%%%%%%%%%
% Document data
%%%%%%%%%%%%%%%%%%%%%%%%%%%%%%%%%%%%%%%%%%%%%%%%%%%%%%%%%%%%%%%%%%%%%%%%%%%%%%%%%%%%
\documentclass[12pt]{article} %report allows for chapters
%%%%%%%%%%%%%%%%%%%%%%%%%%%%%%%%%%%%%%%%%%%%%%%%%%%%%%%%%%%%%%%%%%%%%%%%%%%%%%%%%%%%
\usepackage{preamble}

\begin{document}

\begin{center}
   \textsc{\large MATH 272, Homework 1, \emph{Solutions}}\\
   \textsc{Due January 31$^\textrm{st}$}
\end{center}
\vspace{.5cm}

\begin{problem}
	Plot the following complex functions as vector fields. Then explain the differences between them.
	\begin{enumerate}[(a)]
		\item $f(z) = z$;
		\item $g(z)=iz$.
	\end{enumerate}
You can, for example, use the plotter here: \url{https://www.desmos.com/calculator/eijhparfmd} or find your own (Matlab for example can plot vector fields quite easily). Note that you will have to convert from the complex numbers to 2-dimensional real vectors (i.e., vectors in $\R^2$).  
\end{problem}
\begin{solution}
	I have used Matlab to plot these functions over a reasonable input range $\RE(z),\IM(z) \in [-3,3]$.
		\begin{enumerate}[(a)]
			\item We have the plot:
			\begin{figure}[H]
				\centering
				\includegraphics[width=\textwidth]{div_field.png}
				\caption{The vector field for $f(z)=z$ (think of $f(x+iy)=x+iy$).}
			\end{figure}
			\item We have the plot:
						\begin{figure}[H]
							\centering
							\includegraphics[width=\textwidth]{rot_field.png}
							\caption{The vector field for $g(z)=iz$ (think of $f(x+iy)=-y+ix$).}
						\end{figure}
			\end{enumerate}
		The differences between these two functions are pretty important. One seems to point radially outward from the origin and the lengths of the vectors grow as we move further away from the origin as well. There is no rotation in the field for (a). However, for (b), there is only rotation and there is no notion of the vectors pointing radially away from the origin. Again, the vectors do get longer as we move further from the origin.
		
		These fields will arise again. In (a), the field has no curl but has divergence. In (b), the field has no divergence but has curl. We will see the definitions of these words later on.
\end{solution}

\newpage
\begin{problem}
	Let $\Psi(x)$ be a complex function with domain $[0,L]$.  Show that multiplication by a global phase $e^{i\theta}$ does not affect the norm of $\Psi(x)$ under the Hermitian (integral) inner product. In more generality, this shows that you cannot fully determine a quantum state -- there will always be an undetermined phase.
\end{problem}
\begin{solution}
	We take the following
	\begin{align*}
		\|e^{i\theta} \Psi\|^2=\innprod{e^{i\theta}\Psi}{e^{i\theta}\Psi} &= \int_0^L \left(e^{i\theta}\Psi(x)\right)\left(e^{i\theta}\Psi(x)\right)^*dx\\
		&= \int_0^L e^{i\theta}e^{-i\theta} \Psi(x)\Psi^*(x)dx\\
		&= \int_0^L \Psi(x)\Psi^*(x)dx\\
		&= \innprod{\Psi}{\Psi}\\
		&= \|\Psi\|^2.
	\end{align*}
\end{solution}

\newpage
\begin{problem}
	Consider the real function $f(x)=1$ on the domain $[0,L]$.
	\begin{enumerate}[(a)]
		\item What is the norm of $f$, $\|f\|$?
		\item Normalize $f(x)$.
		\item Find a nonzero normalized polynomial of degree $\leq 1$ that is orthogonal to $f(x)$.
	\end{enumerate}
\end{problem}
\begin{solution}~
	\begin{enumerate}[(a)]
		\item We compute the norm by
		\begin{align*}
			\|f\| = \sqrt{\innprod{f}{f}} &=\sqrt{ \int_0^L f^2(x)dx}\\
			&=\sqrt{ \int_0^L 1 dx}\\
			&= \sqrt{L}.
		\end{align*}
		\item We can normalize $f$ by letting $c$ be some constant and forcing
		\[
		1=\|cf\| = c^2L.
		\]
		Thus $c=\frac{1}{\sqrt{L}}$.  We can write the normalized function as
		\[
		h(x)=\frac{1}{\sqrt{L}}. 
		\]
		\item Consider an arbitrary polynomial of degree $\leq 1$ by putting $g(x)=ax+b$.  Now, we want this polynomial to be orthogonal to $f(x)$ which means that we want
		\[
		\innprod{f}{g}=0.
		\]
		Let us compute the above
		\begin{align*}
			\innprod{f}{g} &= \int_0^L f(x)g(x)dx\\
			&=\int_0^L ax+bdx\\
			&= \frac{aL^2}{2}+bL\\
			&= \frac{1}{2}L\left(aL+2b\right).
		\end{align*}
		Hence, we can solve for $a$ by
		\[
		0=aL+2b \quad \implies \quad a= -\frac{2b}{L}.
		\]
		Now, $g(x)=-\frac{2b}{L}x+b$.  But, we require $g(x)$ to be normalized as well hence
		\begin{align*}
			1=\innprod{g}{g} &= \int_0^L \left(-\frac{2b}{L}x+b\right)^2dx\\
			&= \frac{b^2L}{3}.
		\end{align*}
		Solving for $b$, we find $b=\sqrt{\frac{3}{L}}$ and hence we have that
		\[
		g(x) = -2\sqrt{\frac{3}{L^3}}x+\sqrt{\frac{3}{L}}..
		\]
	\end{enumerate}
\end{solution}

\newpage
\begin{problem}
	A wavefunction $\Psi(x)$ for a particle in the 1-dimensional box $[0,L]$ could be written as a superposition of normalized states
	\[
	\psi_n(x) = \sqrt{\frac{2}{L}} \sin\left(\frac{n\pi x}{L}\right).
	\]
	That is,
	\[
	\Psi(x) = \sum_{n=1}^\infty a_n \psi_n(x),
	\]
	for some choice of the coefficients $a_n$.
	\begin{enumerate}[(a)]
		\item Let $a_n = \frac{\sqrt{6}}{n\pi}$. Show that $\Psi(x)$ is normalized. \emph{Hint: first, use orthogonality of the states $\psi_n(x)$ to your advantage. Then you will need to know what an infinite series evaluates to. Use a tool like WolframAlpha to evaluate this series.}
		\item Note that we can approximate $\Psi(x)$ by taking a finite sum approximation up to some chosen $N$ by
		\[
			\Psi(x) \approx \sum_{n=1}^N a_n \psi_n(x).
		\]
		Plot the approximation of $\Psi(x)$ for $N=1,5,50,100$.  \emph{Hint: you can modify my Desmos examples.}
		\end{enumerate}
\end{problem}
\begin{solution}
	\begin{enumerate}[(a)]
		\item To see that $\Psi(x)$ is normalized we take
		\begin{align*}
			\innprod{\Psi}{\Psi} &= \innprod{\sum_{n=1}^\infty a_n \psi_n(x)}{\sum_{n=1}^\infty a_n \psi_n(x)}\\
			&= \sum_{n=1}^\infty \|a_n\|^2 \innprod{\psi_n}{\psi_n} &\textrm{by orthogonality of the states}\\
			&= \sum_{n=1}^\infty \frac{6}{n^2 \pi^2}\\
			&= \frac{6}{\pi^2} \sum_{n=1}^\infty \frac{1}{n^2}\\
			&= \frac{6}{\pi^2} \zeta(2)\\
			&= 1.
		\end{align*}
		Note the sum above is the Zeta function we saw in Math 271 and $\zeta(2)$ is a well-known value (that you can find by computing the above sum in, for example, WolframAlpha.
		\item ~
	\begin{figure}[H]
	\centering
		\begin{subfigure}[h]{0.48\textwidth}
			\centering
			\includegraphics[width=.8\textwidth]{n=1.png}
			\caption{The approximation to $\Psi(x)$ with $N=1$.}
		\end{subfigure}
		~
		\begin{subfigure}[h]{0.48\textwidth}
			\centering
			\includegraphics[width=.8\textwidth]{n=5.png}
			\caption{The approximation to $\Psi(x)$ with $N=5$.}
		\end{subfigure}
		\\
		\begin{subfigure}[h]{0.48\textwidth}
			\centering
			\includegraphics[width=.8\textwidth]{n=50.png}
			\caption{The approximation to $\Psi(x)$ with $N=50$.}
		\end{subfigure}
		~
		\begin{subfigure}[h]{0.48\textwidth}
			\centering
			\includegraphics[width=.8\textwidth]{n=100.png}
			\caption{The approximation to $\Psi(x)$ with $N=100$.}
		\end{subfigure}
	\end{figure}
	\end{enumerate}
\end{solution}

\newpage
\begin{problem}
	Suppose we have two vectors $\vecu,\vecv \in \R^3$.  We can compute the distance between the vectors
	\[
	d(\vecu,\vecv) = \|\vecu-\vecv\| = \sqrt{(\vecu-\vecv)\cdot(\vecu-\vecv)}.
	\]
	That is to say, we inherit not only a norm from an inner product, but a distance function from a norm!  Intuitively, we are finding the length (or norm) of the vector extending from the head of $\vecv$ to the head of $\vecu$.
	\begin{enumerate}[(a)]
		\item Show that
		\[
		d(\vecu,\vecv) = \sqrt{\|\vecu\|^2+\|\vecv\|^2-2\vecu\cdot \vecv}.
		\]
		\item Compute the distance between vectors $\vecu=\xhat + \zhat$ and $\vecv = \xhat - \yhat$.  
		\item Extend this notion to compute the distance between the Legendre polynomials $f_1,f_2\colon [-1,1] \to \R$ where $f_1(x)=\sqrt{\frac{3}{2}}x$ and $f_2(x)=\sqrt{\frac{5}{8}}\left(1-3x^2\right)$. \emph{Hint: make sure you use the correct integral inner product for this domain!}
	\end{enumerate}
\end{problem}
\begin{solution}~
	\begin{enumerate}[(a)]
		\item We take
		\begin{align*}
			d(\vecu,\vecv)&= \sqrt{(\vecu-\vecv)\cdot(\vecu-\vecv)}\\
				&= \sqrt{\vecu\cdot \vecu + \vecu\cdot (-\vecv) + (-\vecv)\cdot \vecu +(-\vecv)\cdot (-\vecv)}\\
				&= \sqrt{\|\vecu\|^2 + \|\vecv\|^2 -2\vecu \cdot \vecv}.
		\end{align*}
		\item Let us compute each part of the distance formula,
		\[
		\|\vecu\|= \sqrt{2}, \qquad \|\vecv\|=\sqrt{2}, \qquad \vecu \cdot \vecv = 1.
		\]
		Hence we have
		\[
		d(\vecu,\vecv) = \sqrt{2+2-2}=\sqrt{2}.
		\]
		\item In order to compute a distance between functions we can just replace the dot product with the Hermitian inner product in the form 
		\[
		\innprod{f}{g}=\int_{-1}^1 f(x)g(x)dx. 
		\]
		This then gives us an induced norm that we will use in replacement of the Euclidean norm above.  We now compute
		\[
		\|f_1\| = \int_{-1}^1 |f_1(x)|^2 dx = 1, \qquad \|f_2\|=\int_{-1}^1 |f_2(x)|^2dx =1, \qquad \innprod{f_1}{f_2} = \int_{-1}^1 f_1(x)f_2(x)dx =0.
		\]
		Thus we have the distance
		\[
		d(f_1,f_2) = \sqrt{1+1-0}=\sqrt{2}.
		\]
	\end{enumerate}
\end{solution}
\end{document}
%%%%%%%%%%%%%%%%%%%%%%%%%%%%%%%%%%%%%%%%%%%%%%%%%%%%%%%%%%%%%%%%%%%%%%%%%%%%%%%%%%%%
% Document data
%%%%%%%%%%%%%%%%%%%%%%%%%%%%%%%%%%%%%%%%%%%%%%%%%%%%%%%%%%%%%%%%%%%%%%%%%%%%%%%%%%%%
\documentclass[12pt]{article} %report allows for chapters
%%%%%%%%%%%%%%%%%%%%%%%%%%%%%%%%%%%%%%%%%%%%%%%%%%%%%%%%%%%%%%%%%%%%%%%%%%%%%%%%%%%%
\usepackage{preamble}

\begin{document}

\begin{center}
   \textsc{\large MATH 271, Homework 3}\\
   \textsc{Due September 17$^\textrm{th}$}
\end{center}
\vspace{.5cm}

\begin{problem}
Write down the equations for each of the reactants and products for the following reactions.
\begin{enumerate}[(a)]
    \item $A + 3B + C \xrightarrow{k} 2D+2E$.
    \item $A \xrightarrow{k_1} B + C \xrightarrow{k_2} D$.
\end{enumerate}
\end{problem}

\begin{problem}
Consider the following reaction
\[
A \xrightarrow{k_1} B \xrightarrow{k_2} C.
\]
For the following parts, use the link: \url{https://www.desmos.com/calculator/srrpeadlou}.
\begin{enumerate}[(a)]
    \item Compare and contrast the reactions that take place given the three different scenarios for initial conditions. Explain why what the graph displays makes sense and include your graphs.
    \begin{itemize}
        \item $[A]_0 = 1$, $[B]_0=0$, and $[C]_0 =0$.
        \item $[A]_0 = 0$, $[B]_0=1$, and $[C]_0 =0$.
        \item $[A]_0 = 0$, $[B]_0=0$, and $[C]_0 =1$.
    \end{itemize}
    \item For the initial conditions $[A]_0 = 1$, $[B]_0=0$, and $[C]_0 =0$, explain what happens when you let
    \begin{itemize}
        \item $k_1=0$ and $k_2=1$,
        \item $k_1=1$ and $k_2=0$.
    \end{itemize}
    Include plots for these cases as well.
    \item Consider the initial conditions $[A]_0 = 1$, $[B]_0=0$, and $[C]_0 =0$ and rate constants $k_1=1$ and $k_2=2$. Then, choose initial conditions of your own and compare your plots with the other initial conditions. Why do yours behave the way they do? Include your plots.
\end{enumerate}
\end{problem}

\begin{problem}
Consider the second order chemical reaction given by
\[
A+B \xrightarrow{k} \textrm{Products}.
\]
\begin{enumerate}[(a)]
    \item Write a \emph{system} of differential equations to describe the concentration of the reactants $A$ and $B$ (this means write one for each).
    \item The concentrations of $A$ and $B$ can be related to each other in the following way: Let $A=A_0-x$ and $B=B_0-x$. Here, we think of $x$ as the amount of each chemical that has reacted, and note that it depends on time $t$. Use this change of variables to rewrite the differential equation for chemical $A$ in terms of $x$ and $t$.
    \item Solve the differential equation in (b) with the initial condition $x(0)=0$.You will need to use \emph{partial fraction decomposition} to evaluate the integral.
\end{enumerate}
\end{problem}

\begin{problem}
If $x_1(t)$ and $x_2(t)$ are solutions to the differential equation
\[
x'' + bx' +cx = 0
\]
is $x=x_1+x_2+k$ for a constant $k$ always a solution? Is the function $y=tx_1$ a solution? Explain.
\end{problem}


\begin{problem}
Consider the following initial value problem:
\begin{align*}
    x''+4x'+3x&=0
\end{align*}
with initial data $x(0)=1$, $x'(0)=0$.  
\begin{enumerate}[(a)]
    \item Find the solution.
    \item Sketch a plot of the solution.
    \item Explain in words what is happening to the solution as time goes on. What happens as $t\to \infty$?
\end{enumerate}
\end{problem}

\begin{problem}
Write down a homogeneous second-order linear differential equation where the system displays a decaying oscillation.
\end{problem}

\begin{problem}
Consider the following differential equation:
\[
x''+2x'+x=\sin(t)
\]
\begin{enumerate}[(a)]
    \item Find the homogeneous solution $x_H(t)$.
    \item Find the particular integral $x_P(t)$.
    \item Find the specific solution corresponding to the initial data $x(0)=0$, $x'(0)=0$.
    \item Plot the curve $(x(t),x'(t))$ in the plane (we often call this \emph{phase space}). Use this link here: \url{https://www.desmos.com/calculator/ouqwcxj2xz}. Use the time range $t\in [0,10\pi]$.
    \item Describe what happens with this system over time. Does it seem to approach some kind of stable solution? Note that this stable solution could be periodic.
\end{enumerate}
\end{problem}


\end{document}
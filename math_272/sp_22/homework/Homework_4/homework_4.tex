%%%%%%%%%%%%%%%%%%%%%%%%%%%%%%%%%%%%%%%%%%%%%%%%%%%%%%%%%%%%%%%%%%%%%%%%%%%%%%%%%%%%
% Document data
%%%%%%%%%%%%%%%%%%%%%%%%%%%%%%%%%%%%%%%%%%%%%%%%%%%%%%%%%%%%%%%%%%%%%%%%%%%%%%%%%%%%
\documentclass[12pt]{article} %report allows for chapters
%%%%%%%%%%%%%%%%%%%%%%%%%%%%%%%%%%%%%%%%%%%%%%%%%%%%%%%%%%%%%%%%%%%%%%%%%%%%%%%%%%%%
\usepackage{preamble}
\newcommand{\grad}{\boldsymbol{\vec{\nabla}}}
\newcommand{\vecfieldB}{\boldsymbol{\vec{B}}}
\newcommand{\vecfieldE}{\boldsymbol{\vec{E}}}
\newcommand{\rhat}{\boldsymbol{\hat{r}}}
\newcommand{\thetahat}{\boldsymbol{\hat{\theta}}}
\newcommand{\phihat}{\boldsymbol{\hat{\phi}}}
\newcommand{\rhohat}{\boldsymbol{\hat{\rho}}}
\newcommand{\vecfieldV}{\boldsymbol{\vec{V}}}

\begin{document}

\begin{center}
   \textsc{\large MATH 272, Homework 4}\\
   \textsc{Due February 26$^\textrm{th}$}
\end{center}
\vspace{.5cm}

\begin{problem}
\textbf{(7 pts.)} Let $\vecfieldV$ be a vector field in the plane $\R^2$ defined by
\[
\vecfieldV(x,y) = \begin{pmatrix} \frac{1}{2}x-y \\ x + \frac{1}{2}y \end{pmatrix},
\]
and let $\vecx(t) = \begin{pmatrix}  e^{\frac{1}{2}t} (-c_1 \sin(t) + c_2 \cos(t) ) \\ e^{\frac{1}{2}t} (c_1 \cos(t) + c_2 \sin(t)) \end{pmatrix}$ for $t\in [0,\pi]$ where $c_1$ and $c_2$ are yet undetermined constants.
\begin{enumerate}[(a)]
    \item \textbf{(2 pts.)} Show that a flow of $\vecfieldV$ yields a linear system of equations.
    \item \textbf{(2 pts.)} Show that $\vecx(t)$ is a flow of the vector field $\vecfieldV$.
    \item \textbf{(1 pts.)} Let $\vecx(0)=\begin{pmatrix} 1 \\ 0 \end{pmatrix}$. Determine the particular solution to the initial value problem.
    \item \textbf{(2 pts.)} Plot the $\vecfieldV$ and your particular solution $\vecx$ simultaneously. Choose good bounds for your plot so that the whole curve is visible.
\end{enumerate}
\end{problem}

\begin{problem}
\textbf{(7 pts.)} Consider our model for a molecular crystal potential for which we took the scalar field
\[
u(x,y)=\cos^2(x)+\cos^2(y).
\]
\begin{enumerate}[(a)]
\item \textbf{(1 pts.)} Plot the graph of of $u(x,y)$ and the level curves. Feel free to use your old work.
\item \textbf{(1 pts.)} Write down the differential equation for a curve given by gradient descent of the system. That is, the negative of the gradient flow. 
\item \textbf{(2 pts.)} Without solving the problem, where would particles end up if they follow gradient descent? What if particles start on a peak?
\item \textbf{(3 pts.)} Recall the matrix $[J]$ given by
\[
[J] = \begin{pmatrix} 0 & -1 \\ 1 & 0 \end{pmatrix}.
\]
The \emph{Hamiltonian flow} is given by
\[
\boldsymbol{\dot{\vec{x}}}(t) = [J] \grad u(\vecx(t)).
\]
The \emph{Hamiltonian vector field} is the right hand side,
\[
[J] \grad u(x,y).
\]
Plot the Hamiltonian vector field and explain why the level curves to $u(x,y)$ correspond to the Hamiltonian flows.
\end{enumerate}
\end{problem}

\begin{problem}
\textbf{(10 pts.)} Let us consider the discrete heat equation for $n$ equally spaced particles on a line segment for which we have the following picture
\begin{figure}[H]
	\centering
	\resizebox{.9\columnwidth}{!}{\input{figures/equally_spaced_line_points.pdf_tex}}
\end{figure}
Let $u_j(t) \coloneqq u(x_j,t)$ denote the temperature of particle $j$ at time $t$, let $k_j$ be the thermal transport coefficient between particles $j$ and $j+1$, and let $f_j(t)=f(x_j,t)$ be the thermal energy source on particle $j$.
\begin{enumerate}[(a)]
    \item \textbf{(2 pts.)} For the boundary particles $x_1$ and $x_n$, we have
    \[
    \dot{u}_1 = -k_1 u_1 + k_1 u_{2} + f_1 \qquad \textrm{and} \qquad \dot{u}_n = -k_n u_{n} + k_{n-1} u_{n-1} +f_n,
    \]
    which correspond to \emph{Neumann type boundary conditions}. Explain each term in the above equations.
    \item \textbf{(2 pts.)} If we attached $x_1$ to $x_n$ with a material with a thermal transport coefficent of $k_0$ the above equations would need modification. Write these new equations. These are the \emph{periodic boundary conditions}. 
    \item \textbf{(1 pts.)} Explain why periodic boundary conditions are the same as working with a circular domain.
    \item \textbf{(1 pts.)} If we force $u_1$ and $u_n$ to be constant, what will the equations for the boundary particles be? These would be the \emph{Dirichlet type boundary conditions}.
    \item \textbf{(2 pts.)} For the interior particles, we have the relationship
    \[
    \dot{u}_j = -k_{j-1}u_j - k_{j} u_j + k_{j-1}u_{j-1} + k_{j} u_{j+1} +f_j \qquad \textrm{for $j=2,\dots,n-1$}.
    \]
    Explain what each term describes in the above equation.
    \item \textbf{(2 pts.)} In the limit as $n\to \infty$, we then have that $k$ is described as a function of position, $x$. The source free heat equation then reads
    \[
    \frac{\partial}{\partial t}u(x,t) = \frac{\partial}{\partial x} \left( k(x)\frac{\partial}{\partial x} u(x,t) \right) + f(x,t).
    \]
    Explain how this equation differs from the equation
    \[
    \frac{\partial}{\partial t}u(x,t) = k(x) \frac{\partial^2}{\partial x^2} u(x,t)+f(x,t).
    \]
\end{enumerate}
\end{problem}



\begin{problem}
    \textbf{(8 pts.)} Consider the 1-dimensional homogeneous Laplace equation given by 
    \[
    \frac{\partial^2}{\partial x^2} u_E(x) = 0,
    \]
    with the domain $\Omega$ as the unit interval on the $x$-axis.  Take the Dirichlet boundary conditions $u_E(0)=T_0$ and $u_E(L)=T_L$.  Think of these values as the ambient temperature at the endpoints of the rod.  These temperatures are constant since the ambient environment is so large.
    \begin{enumerate}[(a)]
        \item \textbf{(2 pts.)} Find the particular solution to this Laplace equation.
        \item \textbf{(2 pts.)} Suppose that $v(x,t)$ is a solution to the 1-dimensional source free isotropic heat equation with zero Dirichlet boundary values. Show that 
        \[
        u(x,t)=v(x,t)+u_E(x),
        \]  
        is a solution to the 1-dimensional source free isotropic heat equation with Dirichlet boundary values $u(0,t)=T_0$ and $u(L,t)=T_L$.
        \item \textbf{(2 pts.)} From Problem 1, we know that $\lim_{t\to \infty} v(x,t) = 0$.  Hence, show that the long time limit of a solution to the source free heat equation yields a solution to the Laplace equation.
        \item \textbf{(2 pts.)} Argue why the equilibrium temperature profile of a rod can be found without solving the heat equation.
    \end{enumerate}
\end{problem}

\begin{problem}
    \textbf{(3 pts.)} Using intuition from the previous problem, explain how one could solve the heat equation with a nonzero source term that only depends on $x$. In other words, how could one try to solve
    \[
    \left( -k \frac{\partial^2}{\partial x^2} + \frac{\partial}{\partial t} \right) u(x,t) = f(x),
    \]
\end{problem}

\begin{problem}
    \textbf{(13 pts.)} Consider the 2-dimensional source free isotropic heat equation given by
    \[
    \left( -k \Delta + \frac{\partial}{\partial t} \right) u(x,y,t) = 0,
    \]
    with the domain $\Omega$ as the unit square in the $xy$-plane. Take as well the Dirichlet boundary conditions $u(x,y,t)=0$ for $x$ and $y$ on the boundary of $\Omega$.
    \begin{enumerate}[(a)]
        \item \textbf{(2 pts.)} Show that $u_{mn}(x,y,t)=\sin(m\pi x)\sin(n\pi y)e^{-k(n^2+m^2)\pi^2 t}$ is a solution to the PDE and Dirichlet boundary conditions for any non-negative integers $m$ and $n$.
        \item \textbf{(2 pts.)} Show that a linear combination of solutions $u_{mn}$ and $u_{pq}$ is also a solution.
        \item \textbf{(3 pts.)} For $m=n=1$ and $k=1$, plot the solution for the values $t=0$, $t=0.01$, $t=0.1$ and $t=1$.  Explain what is physically happening as time moves forward.
        \item \textbf{(2 pts.)} Explain what varying the value for the conductivity $k$ does to the solution.  Feel free to use plots to support your hypothesis.
        \item \textbf{(2 pts.)} Explain the mathematical reason why increasing $m$ and $n$ causes the solution to converge to zero more quickly.
        \item \textbf{(2 pts.)} Explain the physical reason why increasing $m$ and $n$ causes the solution to converge to zero more quickly. Plots may help support your reasoning.
    \end{enumerate}
\end{problem}
\begin{problem}
\textbf{(11 pts.)} Consider the 1-dimensional wave equation given by
\[
\left(-\frac{\partial^2}{\partial x^2} +\frac{1}{c^2} \frac{\partial^2}{\partial  t^2} \right) u(x,t) = 0.
\]
We'll consider two distinct scenarios. First, we'll take an infinitely long elastic rod and second we'll take a rod of finite length with Dirichlet boundary conditions.
\begin{enumerate}[(a)]
    \item \textbf{(2 pts.)} For a rod of infinite length, consider the initial conditions
    \[
    u(x,0) = \begin{cases} x+1 & -1\leq x \leq 0 \\ 1-x & 0\leq x \leq 1 \\ 0 & \textrm{otherwise} \end{cases} \qquad \textrm{and} \qquad \frac{\partial}{\partial t} u(x,0) = 0.
    \]
    Find and plot the portion of the wave that moves to the right with $c=1$.
    \item \textbf{(2 pts.)} Let $u_R(x,t)$ be your solution from (a), show that this satisfies the right-moving wave equation
    \[
    \left(\frac{\partial}{\partial x} + \frac{1}{c} \frac{\partial}{\partial t} \right)u_R(x,t) = 0.
    \]
    \item \textbf{(1 pts.)} Why is it that we can ignore the points where your function $u_R(x,t)$ is not differentiable even though we are considering this as a solution to a PDE?
    \item \textbf{(2 pts.)} For an elastic rod $\Omega$ of finite length, $\Omega = [0,1]$, assume that we take the Dirichlet conditions $u(0,t)=0=u(1,t)$.  With the initial conditions
    \[
    u(x,0) = \sin(\pi x) \qquad \textrm{and} \qquad \frac{\partial}{\partial t} u(x,0)=0,
    \]
    find the solution using d'Alembert's formula.
    \item \textbf{(2 pts.)} Let $w(x,t)$ be your solution for (d), show that it is indeed equal to
    \[
    w(x,t) = \sin(\pi x)\cos(\pi c t).
    \]
    \item \textbf{(2 pts.)} With your result from (e), explain how we can decompose a standing wave into a linear combination of two waves; one moving towards the left and one moving towards the right and both reflecting off the boundary.
\end{enumerate}
\end{problem}

\begin{problem}
    \textbf{(8 pts.)} Consider the wave problem on the region $\Omega=[0,1]$ given by
    \[
    \begin{cases}
    \left( - \frac{\partial^2}{\partial x^2} +\frac{1}{c^2} \frac{\partial^2}{\partial t^2} \right) u(x,t) =0, & \textrm{in $(0,1)$},\\
    u(0,t)=0 \textrm{~and~} u(1,t)=0, &\textrm{as boundary conditions},\\
    u(x,0)=\sin(\pi x), &\textrm{as the initial condition}.
    \end{cases}
    \]
    This problem corresponds to taking a plucked elastic string fixed at the endpoints.
    \begin{enumerate}[(a)]
        \item \textbf{(1 pts.)} Use the separation of variables ansatz $u(x,t)=X(x)T(t)$ to get a new separation constant. This will give two ODEs: one will be in terms of $X(x)$ and the other will be in terms of $T(t)$.
        \item \textbf{(2 pts.)} Use the boundary conditions and solve the ODE that is in terms of $X(x)$ which will simultaneously find the allowed values for the separation constant.
        \item \textbf{(2 pts.)} Using these allowed values for the separation constant, find the solution for $T(t)$.
        \item \textbf{(1 pts.)} Find the particular solution for $u(x,t)$ by matching the initial condition.
        \item \textbf{(2 pts.)} Plot your solution for $x\in [0,1]$ and $t\in [0,\infty)$ (i.e., just plot up to a large value of $t$). In this case, compare your plots for $c=1/2$ and $c=1$.
    \end{enumerate}
\end{problem}

\begin{problem}
\textbf{(9 pts.)} Consider the heat flow problem on the region $\Omega=[0,1]$ given by
\[
\begin{cases}
\frac{\partial}{\partial t} u(x,t) = \frac{\partial^2}{\partial x^2} u(x,t) - 1, & \textrm{in $(0,1)$},\\
u(0,t)=0 \textrm{~and~} u(1,t)=1, & \text{as boundary conditions},\\
u(x,0) = \sin\left(\pi x\right) + \frac{1}{2}(x^2+x), & \textrm{as the initial condition}.
\end{cases}
\]
This corresponds to a rod kept at fixed temperatures at the endpoints that starts with a warm center initially.
\begin{enumerate}[(a)]
    \item \textbf{(3 pts.)} As with the previous problem, take an ansatz
    \[
    u(x,t) = v(x,t) + u_E(x)
    \]
    where $v(x,t)$ solves the following problem
    \[
\begin{cases}
\frac{\partial}{\partial t} v(x,t) = \frac{\partial^2}{\partial x^2} v(x,t), & \textrm{in $(0,1)$},\\
v(0,t)=0 \textrm{~and~} v(1,t)=0, & \text{as boundary conditions}.
\end{cases}
    \]
    Find the general solution $v(x,t)$ using separation of variables. \emph{Hint: feel free to use the work in the notes (Example ``Solving the Heat Equation" and Example ``Particular Solution to the 1D Heat Equation").}

    \item \textbf{(2 pts.)} Show that for $u(x,t)$ to be a solution that
    \[
    \frac{\partial^2}{\partial x^2} u_E(x) = 1.
    \]

    \item \textbf{(2 pts.)} Find the solution $u_E(x)$ to the following problem
    \[
    \begin{cases}
    \frac{\partial^2}{\partial x^2} u_E(x) = 1, & \textrm{in $(0,1)$},\\
    u_E(0)=0 \textrm{~and~} u_E(1)=1, & \text{as boundary conditions}.
    \end{cases}
    \]
    
    \item \textbf{(2 pts.)} All is left in determining the function $u(x,t)$ is to determine the particular solution that satisfies the initial condition. Using our ansatz $u(x,t)=v(x,t)+u_E(x)$, determine the particular solution.
\end{enumerate}
\end{problem}



\end{document}
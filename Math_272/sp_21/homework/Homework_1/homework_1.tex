%%%%%%%%%%%%%%%%%%%%%%%%%%%%%%%%%%%%%%%%%%%%%%%%%%%%%%%%%%%%%%%%%%%%%%%%%%%%%%%%%%%%
% Document data
%%%%%%%%%%%%%%%%%%%%%%%%%%%%%%%%%%%%%%%%%%%%%%%%%%%%%%%%%%%%%%%%%%%%%%%%%%%%%%%%%%%%
\documentclass[12pt]{article} %report allows for chapters
%%%%%%%%%%%%%%%%%%%%%%%%%%%%%%%%%%%%%%%%%%%%%%%%%%%%%%%%%%%%%%%%%%%%%%%%%%%%%%%%%%%%
\usepackage{preamble}

\begin{document}

\begin{center}
   \textsc{\large MATH 272, Homework 1}\\
   \textsc{Due January 29$^\textrm{nd}$}
\end{center}
\vspace{.5cm}
\textcolor{red}{When doing problem 3 and looking at slices the students need to actually find what the function is...}
\begin{problem}
Plot the following curves and print pictures of each using GeoGebra.  
\begin{enumerate}[(a)]
	\item (Helix) $\curvegamma_1(t) = \begin{pmatrix} 3\cos(t) \\ 3\sin(t) \\ t\end{pmatrix}$, from $t=0$ to $t=2\pi$. Where might this show up? If you think about the Earth moving through space and Moon orbiting Earth, then the Moon follows a (locally) helical path.
	
	\item (Falling Ball) $\curvegamma_2(t) = \begin{pmatrix} t\\  0.5t \\ 9-t^2 \end{pmatrix}$ from $t=0$ to $t=3$.
	
	\item (Trefoil knot) $\curvegamma_3(t) = \begin{pmatrix} \sin(t)+2\sin(2t) \\ \cos(t)-2\cos(2t) \\ -\sin(3t) \end{pmatrix}$ from $t=0$ to $t=2\pi$. (Note that this is the simplest nontrivial \emph{knot}. See: \url{https://en.wikipedia.org/wiki/Trefoil_knot} to learn more.)
	
	\item Create your own curve.
\end{enumerate}
\end{problem}

\begin{problem}
The length of a curve is an important notion. In fact, the length of a curve is often related to the energy of some configuration. We can compute the length of a curve over the time $t=t_0$ to $t=t_1$ by integrating the \emph{speed} of the curve over that time.  That is,
\[
\ell(\curvegamma) = \int_{t_0}^{t_1} \left|\tangentgamma(t)\right| \dif t.
\]
We can compute the \emph{energy} of a curve by taking
\[
E(\curvegamma) = \int_{t_0}^{t_1} \frac{1}{2} \left| \tangentgamma(t)\right|^2 \dif t.
\]
Find the length and energy of the Helix from Problem 1 (a).
\end{problem}

\begin{problem}
Given a scalar field of two variables $f(x,y)$, we can create an object called the \emph{graph} of $f(x,y)$ by plotting the set of points $(x,y,f(x,y))$. In fact, you have done this many times in your life. For example, you have consistently plotted the graph of a function $f(x)$ by plotting $(x,f(x))$ in the plane!  

Using GeoGebra, plot the graph of the following functions. Print these off and include them.  Also, describe the what the graph of the function does as we move along the $x$-axis, the $y$-axis, and along the line $y=x$. For each, use the range $-3\leq x \leq 3$ and $-3\leq y \leq 3$.
\begin{enumerate}[(a)]
	\item $f(x,y) = \frac{4xy}{1+x^2+y^2}$.
	\item $g(x,y) = \sin(xy)$.
	\item $h(x,y) = \frac{-x^2-y^2}{5}$.
\end{enumerate}
\end{problem}

\begin{problem} 
Let us visualize vector fields using GeoGebra (specifically \url{https://www.geogebra.org/m/u3xregNW}). Plot the following vector fields and print them out. 
\begin{enumerate}[(a)]
    \item (Constant wind from the northwest) $\vecfieldV(x,y)=\begin{pmatrix} 1 \\ -1 \\ 0\end{pmatrix}$.
    \item (Two wind fronts meeting) $\vecfieldU(x,y,z)=\begin{pmatrix} y \\ x \\ 0 \end{pmatrix}$.
    \item (Source) $\vecfieldE(x,y,z) = \begin{pmatrix} \frac{x}{(x^2+y^2+z^2)^{3/2}} \\ \frac{y}{(x^2+y^2+z^2)^{3/2}} \\ \frac{z}{(x^2+y^2+z^2)^{3/2}} \end{pmatrix}$.
    \item (Vortex) $\boldsymbol{\vec{S}}(x,y,z)=\begin{pmatrix} \frac{-y}{x^2+y^2+z^2} \\ \frac{x}{x^2+y^2+z^2} \\ 0\end{pmatrix}.$           
\end{enumerate}
\end{problem}

\begin{problem}
Compute the divergence and curl of the Source and Vortex fields from Problem 4 (c) and (d).  What can we say about the divergence and curl of these fields at the origin?
\end{problem}

\begin{problem}
Consider the following scalar field and vector field
\[
f(x,y,z) = x^2+y^2-z^2 \qquad \textrm{and} \qquad \vecfieldV(x,y,z) = \begin{pmatrix} -y \\ x \\ z \end{pmatrix}.
\]
\begin{enumerate}[(a)]
    \item Compute all first order partial derivatives of $f$.
    \item Show that $\frac{\dif^2 f}{\dif x \dif y} = \frac{\dif^2 f}{\dif y \dif x}$.
    \item Explain why $\vecfieldV$ has nonzero curl when $x$ and $y$ are nonzero using a physical argument. (\emph{Hint: What plane would a rod start rotating in?})
    \item Explain why $\vecfieldV$ has nonzero divergence when $z\neq 0$ using a physical argument. (\emph{Hint: In what direction would a particle start accelerating towards infinity?})
    \item Compute the directional derivative of $f$ at $\vecx_0 = (1,2,-3)$ in the direction of $\vecfieldV(\vecx_0)$.
\end{enumerate}
\end{problem}

\begin{problem}
For this problem, let us consider a family of scalar fields of varying dimensionality. In the previous problem, we plotted the graph of a scalar field with two inputs, but when there are more than two inputs we must resort to other methods of visualization.

In particular, we will seek out an understanding of the \emph{level sets} and how to relate these to the gradient of a scalar field. For each part, compute the set of points such that $f(\vecx)=1$, $f(\vecx)=2$, and $f(\vecx)=3$ and plot these sets (including all the different level sets in one plot per function). 

Then for each field, compute the gradient (row) vector
\[
\nablavec f(\vecx) = \begin{pmatrix} \frac{\partial f}{\partial x_1} & \frac{\partial f}{\partial x_2} & \cdots & \frac{\partial f}{\partial x_n}\end{pmatrix}.
\]
Finally, draw an approximation of the the gradient vector field on your plots at a three different points for each part. (\emph{Hint: think of how the gradient relates to level sets of functions!})
\begin{enumerate}[(a)]
	\item Consider the 1-dimensional scalar field 
	\[
	f(x) = |x| = \sqrt{x^2}.
	\]
	Here each level set will be made up of distinct points.
	\item Consider the 2-dimensional scalar field
	\[
	f(x,y) = |\vecx| = \sqrt{x^2+y^2}.
	\]
	Here each level set will be a curve.
	\item Consider the 3-dimensional scalar field
		\[
		f(x,y,z) = |\vecx| = \sqrt{x^2+y^2+z^2}.
		\]
		Here each level set will be a surface.
\end{enumerate}
\end{problem}



\end{document}